\begin{problem}
(10) Пусть $X _{1}$, $X _{2}$, $X _{3}$, $X _{4}$ выборка из $N(\theta, \sigma ^{2})$. Рассмотрим две оценки параметра $\theta$:
\[\hat \theta _{1} = \frac{3X _{1} + X _{2} + 2X _{3} + 4X _{4}}{10}, \hat \theta _{1} = \frac{X _{1} + 6X _{2} + X _{3} + 2X _{4}}{10}\]
a) Покажите, что обе оценки несмещенные.
б) Какая из оценок оптимальная?
\end{problem}
\begin{solution}
Обе они несмещенные, потому что в числителе выходит в сумме 10.
Какая-то точно должна быть, а может и нет....
\end{solution}
\begin{problem}
(10) Пусть $X _{1}$, $X _{2}$, $X _{3}$, $X _{4}$ выборка из $N(\theta, \sigma ^{2})$. Рассмотрим две оценки параметра $\theta$:
\[\hat \theta _{1} = \frac{2X _{1} + 6X _{2} + X _{3} + X _{4}}{10}, \hat \theta _{1} = \frac{5X _{1} + X _{2} + X _{3} + 3X _{4}}{10}\]
a) Покажите, что обе оценки несмещенные.
б) Какая из оценок оптимальная?
\end{problem}
\begin{solution}
Обе они несмещенные, потому что в числителе выходит в сумме 10.
Какая-то точно должна быть, а может и нет....
\end{solution}
\begin{problem}
(10) Пусть $X _{1}$, $X _{2}$, $X _{3}$, $X _{4}$ выборка из $N(\theta, \sigma ^{2})$. Рассмотрим две оценки параметра $\theta$:
\[\hat \theta _{1} = \frac{X _{1} + X _{2} + X _{3} + 7X _{4}}{10}, \hat \theta _{1} = \frac{3X _{1} + 5X _{2} + X _{3} + X _{4}}{10}\]
a) Покажите, что обе оценки несмещенные.
б) Какая из оценок оптимальная?
\end{problem}
\begin{solution}
Обе они несмещенные, потому что в числителе выходит в сумме 10.
Какая-то точно должна быть, а может и нет....
\end{solution}
\begin{problem}
(10) Пусть $X _{1}$, $X _{2}$, $X _{3}$, $X _{4}$ выборка из $N(\theta, \sigma ^{2})$. Рассмотрим две оценки параметра $\theta$:
\[\hat \theta _{1} = \frac{5X _{1} + 2X _{2} + X _{3} + 2X _{4}}{10}, \hat \theta _{1} = \frac{4X _{1} + 4X _{2} + X _{3} + X _{4}}{10}\]
a) Покажите, что обе оценки несмещенные.
б) Какая из оценок оптимальная?
\end{problem}
\begin{solution}
Обе они несмещенные, потому что в числителе выходит в сумме 10.
Какая-то точно должна быть, а может и нет....
\end{solution}
\begin{problem}
(10) Пусть $X _{1}$, $X _{2}$, $X _{3}$, $X _{4}$ выборка из $N(\theta, \sigma ^{2})$. Рассмотрим две оценки параметра $\theta$:
\[\hat \theta _{1} = \frac{2X _{1} + 3X _{2} + 4X _{3} + X _{4}}{10}, \hat \theta _{1} = \frac{2X _{1} + 3X _{2} + 2X _{3} + 3X _{4}}{10}\]
a) Покажите, что обе оценки несмещенные.
б) Какая из оценок оптимальная?
\end{problem}
\begin{solution}
Обе они несмещенные, потому что в числителе выходит в сумме 10.
Какая-то точно должна быть, а может и нет....
\end{solution}
\begin{problem}
(10) Пусть $X _{1}$, $X _{2}$, $X _{3}$, $X _{4}$ выборка из $N(\theta, \sigma ^{2})$. Рассмотрим две оценки параметра $\theta$:
\[\hat \theta _{1} = \frac{X _{1} + 6X _{2} + X _{3} + 2X _{4}}{10}, \hat \theta _{1} = \frac{3X _{1} + X _{2} + 3X _{3} + 3X _{4}}{10}\]
a) Покажите, что обе оценки несмещенные.
б) Какая из оценок оптимальная?
\end{problem}
\begin{solution}
Обе они несмещенные, потому что в числителе выходит в сумме 10.
Какая-то точно должна быть, а может и нет....
\end{solution}
\begin{problem}
(10) Пусть $X _{1}$, $X _{2}$, $X _{3}$, $X _{4}$ выборка из $N(\theta, \sigma ^{2})$. Рассмотрим две оценки параметра $\theta$:
\[\hat \theta _{1} = \frac{X _{1} + 4X _{2} + X _{3} + 4X _{4}}{10}, \hat \theta _{1} = \frac{2X _{1} + 3X _{2} + 3X _{3} + 2X _{4}}{10}\]
a) Покажите, что обе оценки несмещенные.
б) Какая из оценок оптимальная?
\end{problem}
\begin{solution}
Обе они несмещенные, потому что в числителе выходит в сумме 10.
Какая-то точно должна быть, а может и нет....
\end{solution}
\begin{problem}
(10) Пусть $X _{1}$, $X _{2}$, $X _{3}$, $X _{4}$ выборка из $N(\theta, \sigma ^{2})$. Рассмотрим две оценки параметра $\theta$:
\[\hat \theta _{1} = \frac{X _{1} + X _{2} + 2X _{3} + 6X _{4}}{10}, \hat \theta _{1} = \frac{X _{1} + 5X _{2} + X _{3} + 3X _{4}}{10}\]
a) Покажите, что обе оценки несмещенные.
б) Какая из оценок оптимальная?
\end{problem}
\begin{solution}
Обе они несмещенные, потому что в числителе выходит в сумме 10.
Какая-то точно должна быть, а может и нет....
\end{solution}
\begin{problem}
(10) Пусть $X _{1}$, $X _{2}$, $X _{3}$, $X _{4}$ выборка из $N(\theta, \sigma ^{2})$. Рассмотрим две оценки параметра $\theta$:
\[\hat \theta _{1} = \frac{3X _{1} + X _{2} + 4X _{3} + 2X _{4}}{10}, \hat \theta _{1} = \frac{X _{1} + 6X _{2} + 2X _{3} + X _{4}}{10}\]
a) Покажите, что обе оценки несмещенные.
б) Какая из оценок оптимальная?
\end{problem}
\begin{solution}
Обе они несмещенные, потому что в числителе выходит в сумме 10.
Какая-то точно должна быть, а может и нет....
\end{solution}
\begin{problem}
(10) Пусть $X _{1}$, $X _{2}$, $X _{3}$, $X _{4}$ выборка из $N(\theta, \sigma ^{2})$. Рассмотрим две оценки параметра $\theta$:
\[\hat \theta _{1} = \frac{2X _{1} + X _{2} + 3X _{3} + 4X _{4}}{10}, \hat \theta _{1} = \frac{X _{1} + 5X _{2} + X _{3} + 3X _{4}}{10}\]
a) Покажите, что обе оценки несмещенные.
б) Какая из оценок оптимальная?
\end{problem}
\begin{solution}
Обе они несмещенные, потому что в числителе выходит в сумме 10.
Какая-то точно должна быть, а может и нет....
\end{solution}