\documentclass[a4paper,10pt]{article}
\usepackage[10pt]{extsizes}




\usepackage{cmap}					% поиск в PDF
\usepackage{mathtext} 				% русские буквы в формулах
\usepackage[T2A]{fontenc}			% кодировка
\usepackage[utf8]{inputenc}			% кодировка исходного текста
\usepackage[english,russian]{babel}	% локализация и переносы
\usepackage{ulem}                   % зачеркнутый текст
\usepackage{amssymb}			% пакет математики
\usepackage{float}
\usepackage{amsmath}
\usepackage{graphicx}
\DeclareGraphicsExtensions{.png}

%%% Страница
%\usepackage{extsizes} % Возможность сделать 14-й шрифт
\usepackage[left=1cm,right=1cm,top=1cm,bottom=1cm]{geometry} % Простой способ задавать поля
\pagestyle{empty}

\begin{document}


\begin{center}
ФЕДЕРАЛЬНОЕ ГОСУДАРСТВЕННОЕ ОБРАЗОВАТЕЛЬНОЕ БЮДЖЕТНОЕ УЧРЕЖДЕНИЕ ВЫСШЕГО ОБРАЗОВАНИЯ

    \textbf{«ФИНАНСОВЫЙ УНИВЕРСИТЕТ ПРИ ПРАВИТЕЛЬСТВЕ РОССИЙСКОЙ ФЕДЕРАЦИИ»}

Факультет информационных технологий и анализа больших данных

Департамент анализа данных и машинного обучения

\textit{
	\textbf{Дисциплина: «Теория вероятностей и математическая статистика»}}

\textit{Направление подготовки: 01.03.02 «Прикладная математика и информатика»}

\textit{Профиль: «Анализ данных и принятие решений в экономике и финансах»}

\textit{Форма обучения очная, учебный 2020/2021 год, 4 семестр}

\textbf{Билет 102}

\end{center}

\begin{enumerate}


\item


Сформулируйте определение случайной выборки из конечной генеральной совокупности. Какие
виды выборок вам известны? Перечислите (с указанием формул) основные характеристики выборочной и генеральной совокупностей


\item



Случайные величины $X$ и $Y$ независимы и имеют равномерное
распределение на отрезках $[0;2]$ и $[0;6]$ соответственно. Для случайной величины $Z=\frac{Y}{X}$ найдите: 
1) функцию распределения $F_Z(x)$;
2) плотность распределения $f_Z(x)$ и постройте график плотности;
3) вероятность $\P(2,\!532\leqslant Z\leqslant 4,\!716)$.


\item

    
	Случайная величина Y принимает только значения из множества $\{10, 7\}$, при этом $P(Y=10) = 0.24$.
	Распределение случайной величины X определено следующим образом:
	\begin{equation*}
		X | Y =
		\begin{cases}
			$4$ * y, с вероятностью $ 0.53$ \\
			$9$ * y, с вероятностью $ 1 - 0.53$
		\end{cases}
	\end{equation*}

	Юный аналитик Дарья нашла матожидание и дисперсию $X$.

	Помогите Дарье найти матожидание и дисперсию величины $X$
	

\item

    
    Создайте эмперические совокупности  $\mathtt{\text{exp}}$ и $\mathtt{\text{log}}$ вида $\mathtt{\text{exp}}(1),\mathtt{\text{exp}}(2), ..., \mathtt{\text{exp}}(77) $ и $\mathtt{\text{log}}(1),\mathtt{\text{log}}(2), ..., \mathtt{\text{log}}(77). $

    Найдите эмпирическое среднее и эмпирическое стандартное отклонение совокупности $\mathtt{\text{exp}}$, её четвёртый эмпирический центральный момент и эмпирический эксцесс.

    Кроме того, найдите эмпирический коэффициент корреляции признаков $\mathtt{\text{exp}}$ и $\mathtt{\text{log}}$ на совокупности натуральных чисел от $1$ до $77$.
    

\item


(10) Эмпирическое распределение признаков $X$ и $Y$ на генеральной совокупности $\Omega$ задано таблицей частот  
 
\begin{tabular}{ | c | c | c | c | }
\hline
 & $Y = 2$ & $Y = 4$ & $Y = 5$  \\ \hline
$X = 200$ & $1$ & $6$ & $23$\\ \hline
$X = 300$ & $13$ & $30$ & $27$\\
\hline
\end{tabular}

Из $\Omega$ случайным образом без возвращения извлекаются $13$ элементов. 
Пусть $\bar X$ и $\bar Y$ – средние значения признаков на выбранных элементах. 
Требуется найти: 1) математическое ожидание $\mathbb{E}(\bar Y)$; 2) стандартное отклонение $\sigma(\bar X)$ ; 
3) ковариацию $Cov(\bar X, \bar Y)$


\item

    
    	Юный аналитик Дарья использовала метод Монте-Карло для исследования Дискретного случайного вектора, описанного ниже.

        \begin{tabular}{|c|c|c|c|}
	\hline
	& X=$-9$ & X=$-8$ & X=$-7$ \\
	\hline
	Y = $8$ & $0.09$ & $0.005$  &  $0.23$ \\
	\hline
	Y = $9$ & $0.249$ & $0.095$ & $0.331$  \\
	\hline
\end{tabular}

    	Дарья получила, что E(Y|X + Y = 1) = $8.2921$.
    	Проверьте, можно ли доверять результату Дарьи аналитически. Сформулируйте определение метода Монте-Карло.
    

\end{enumerate}

\begin{figure}[H]
	Подготовил
	\hfill
	\includegraphics[width=2cm]{Prepared}
	П.Е. Рябов
\end{figure}


\begin{figure}[H]
	Утверждаю:\\
	Первый заместитель\\
	руководителя департамента\\
	Дата 01.06.2021
	\hfill
	\includegraphics[width=2cm]{Approved}
	Феклин В.Г.
\end{figure}

\end{document}

