\documentclass[a4paper,12pt]{article}
\usepackage[12pt]{extsizes}




\usepackage{cmap}					% поиск в PDF
\usepackage{mathtext} 				% русские буквы в формулах
\usepackage[T2A]{fontenc}			% кодировка
\usepackage[utf8]{inputenc}			% кодировка исходного текста
\usepackage[english,russian]{babel}	% локализация и переносы
\usepackage{ulem}                   % зачеркнутый текст
\usepackage{amssymb}			% пакет математики
\usepackage{float}
\usepackage{amsmath}
\usepackage{graphicx}
\DeclareGraphicsExtensions{.png}

%%% Страница
%\usepackage{extsizes} % Возможность сделать 14-й шрифт
\usepackage[left=1cm,right=1cm,top=1cm,bottom=1cm]{geometry} % Простой способ задавать поля
\pagestyle{empty}

\begin{document}


\begin{center}
ФЕДЕРАЛЬНОЕ ГОСУДАРСТВЕННОЕ ОБРАЗОВАТЕЛЬНОЕ БЮДЖЕТНОЕ УЧРЕЖДЕНИЕ ВЫСШЕГО ОБРАЗОВАНИЯ

    \textbf{«ФИНАНСОВЫЙ УНИВЕРСИТЕТ ПРИ ПРАВИТЕЛЬСТВЕ РОССИЙСКОЙ ФЕДЕРАЦИИ»}

Факультет информационных технологий и анализа больших данных

Департамент анализа данных и машинного обучения

\textit{
	\textbf{Дисциплина: «Теория вероятностей и математическая статистика»}}

\textit{Направление подготовки: 01.03.02 «Прикладная математика и информатика»}

\textit{Профиль: «Анализ данных и принятие решений в экономике и финансах»}

\textit{Форма обучения очная, учебный 2020/2021 год, 4 семестр}



\end{center}

\begin{enumerate}

\item

Дайте определение случайной величины, которая имеет гамма-распределение $\Gamma(\alpha,  \lambda)$, и выведите основные свойства гамма-расределения. Запишите формулы для математичсекого ожидания
$\mathbb{E}(X)$ и дисперсии $\mathbb{V}ar(X)$ гамма-распределения



\item


Дайте определение случайной величины, которая имеет $\chi ^{2}$-распределение с n степенями свободы.
Запишите плотность $\chi ^{2}$- распределения. Выведите формулы для математического ожидания $\mathbb{E}(X)$ и дисперсии $\mathbb{V}ar(X)$ $\chi ^{2}$-распределение с n степенями свободы. Найдите а) $\mathbb{P}(\chi _{20}^{2} > 10.9)$, где $\chi _{20}^{2}$–случайная величина, которая имеет $\chi ^{2}$– распределение с 20 степенями свободы; б) найдите 93\%
(верхнюю) точку $\chi _{0.93}^{2} (5)$ хи-квадрат распределения с 5 степенями свободы



\item


Сформулируйте определение случайной выборки из конечной генеральной совокупности. Какие
виды выборок вам известны? Перечислите (с указанием формул) основные характеристики выборочной и генеральной совокупностей



\item

Случайные величины $X$ и $Y$ независимы и имеют равномерное
распределение на отрезках $[0;10]$ и $[0;3]$ соответственно. Для случайной величины $Z=\frac{Y}{X}$ найдите: 
1) функцию распределения $F_Z(x)$;
2) плотность распределения $f_Z(x)$ и постройте график плотности;
3) вероятность $\P(0,\!057\leqslant Z\leqslant 0,\!556)$.



\item



Случайные величины $X$ и $Y$ независимы и имеют равномерное
распределение на отрезках $[0;2]$ и $[0;1]$ соответственно. Для случайной величины $Z=\frac{Y}{X}$ найдите: 
1) функцию распределения $F_Z(x)$;
2) плотность распределения $f_Z(x)$ и постройте график плотности;
3) вероятность $\P(0,\!093\leqslant Z\leqslant 0,\!551)$.



\item



Случайные величины $X$ и $Y$ независимы и имеют равномерное
распределение на отрезках $[0;1]$ и $[0;9]$ соответственно. Для случайной величины $Z=\frac{Y}{X}$ найдите: 
1) функцию распределения $F_Z(x)$;
2) плотность распределения $f_Z(x)$ и постройте график плотности;
3) вероятность $\P(1,\!683\leqslant Z\leqslant 13,\!185)$.



\item



Случайные величины $X$ и $Y$ независимы и имеют равномерное
распределение на отрезках $[0;8]$ и $[0;2]$ соответственно. Для случайной величины $Z=\frac{Y}{X}$ найдите: 
1) функцию распределения $F_Z(x)$;
2) плотность распределения $f_Z(x)$ и постройте график плотности;
3) вероятность $\P(0,\!094\leqslant Z\leqslant 0,\!294)$.



\item



Случайные величины $X$ и $Y$ независимы и имеют равномерное
распределение на отрезках $[0;9]$ и $[0;10]$ соответственно. Для случайной величины $Z=\frac{Y}{X}$ найдите: 
1) функцию распределения $F_Z(x)$;
2) плотность распределения $f_Z(x)$ и постройте график плотности;
3) вероятность $\P(0,\!277\leqslant Z\leqslant 1,\!26)$.



\item



Случайные величины $X$ и $Y$ независимы и имеют равномерное
распределение на отрезках $[0;6]$ и $[0;8]$ соответственно. Для случайной величины $Z=\frac{Y}{X}$ найдите: 
1) функцию распределения $F_Z(x)$;
2) плотность распределения $f_Z(x)$ и постройте график плотности;
3) вероятность $\P(0,\!729\leqslant Z\leqslant 1,\!912)$.



\item



Случайные величины $X$ и $Y$ независимы и имеют равномерное
распределение на отрезках $[0;6]$ и $[0;1]$ соответственно. Для случайной величины $Z=\frac{Y}{X}$ найдите: 
1) функцию распределения $F_Z(x)$;
2) плотность распределения $f_Z(x)$ и постройте график плотности;
3) вероятность $\P(0,\!087\leqslant Z\leqslant 0,\!235)$.



\item



Случайные величины $X$ и $Y$ независимы и имеют равномерное
распределение на отрезках $[0;3]$ и $[0;8]$ соответственно. Для случайной величины $Z=\frac{Y}{X}$ найдите: 
1) функцию распределения $F_Z(x)$;
2) плотность распределения $f_Z(x)$ и постройте график плотности;
3) вероятность $\P(2,\!475\leqslant Z\leqslant 4,\!811)$.



\item



Случайные величины $X$ и $Y$ независимы и имеют равномерное
распределение на отрезках $[0;2]$ и $[0;4]$ соответственно. Для случайной величины $Z=\frac{Y}{X}$ найдите: 
1) функцию распределения $F_Z(x)$;
2) плотность распределения $f_Z(x)$ и постройте график плотности;
3) вероятность $\P(0,\!588\leqslant Z\leqslant 3,\!842)$.



\item



Случайные величины $X$ и $Y$ независимы и имеют равномерное
распределение на отрезках $[0;1]$ и $[0;8]$ соответственно. Для случайной величины $Z=\frac{Y}{X}$ найдите: 
1) функцию распределения $F_Z(x)$;
2) плотность распределения $f_Z(x)$ и постройте график плотности;
3) вероятность $\P(0,\!16\leqslant Z\leqslant 11,\!592)$.



\item



Случайные величины $X$ и $Y$ независимы и имеют равномерное
распределение на отрезках $[0;4]$ и $[0;3]$ соответственно. Для случайной величины $Z=\frac{Y}{X}$ найдите: 
1) функцию распределения $F_Z(x)$;
2) плотность распределения $f_Z(x)$ и постройте график плотности;
3) вероятность $\P(0,\!182\leqslant Z\leqslant 1,\!21)$.



\item



Случайные величины $X$ и $Y$ независимы и имеют равномерное
распределение на отрезках $[0;4]$ и $[0;7]$ соответственно. Для случайной величины $Z=\frac{Y}{X}$ найдите: 
1) функцию распределения $F_Z(x)$;
2) плотность распределения $f_Z(x)$ и постройте график плотности;
3) вероятность $\P(0,\!035\leqslant Z\leqslant 2,\!775)$.



\item



Случайные величины $X$ и $Y$ независимы и имеют равномерное
распределение на отрезках $[0;1]$ и $[0;10]$ соответственно. Для случайной величины $Z=\frac{Y}{X}$ найдите: 
1) функцию распределения $F_Z(x)$;
2) плотность распределения $f_Z(x)$ и постройте график плотности;
3) вероятность $\P(2,\!96\leqslant Z\leqslant 17,\!91)$.



\item



Случайные величины $X$ и $Y$ независимы и имеют равномерное
распределение на отрезках $[0;10]$ и $[0;9]$ соответственно. Для случайной величины $Z=\frac{Y}{X}$ найдите: 
1) функцию распределения $F_Z(x)$;
2) плотность распределения $f_Z(x)$ и постройте график плотности;
3) вероятность $\P(0,\!719\leqslant Z\leqslant 1,\!005)$.



\item



Случайные величины $X$ и $Y$ независимы и имеют равномерное
распределение на отрезках $[0;2]$ и $[0;7]$ соответственно. Для случайной величины $Z=\frac{Y}{X}$ найдите: 
1) функцию распределения $F_Z(x)$;
2) плотность распределения $f_Z(x)$ и постройте график плотности;
3) вероятность $\P(2,\!019\leqslant Z\leqslant 3,\!843)$.



\item



Случайные величины $X$ и $Y$ независимы и имеют равномерное
распределение на отрезках $[0;5]$ и $[0;10]$ соответственно. Для случайной величины $Z=\frac{Y}{X}$ найдите: 
1) функцию распределения $F_Z(x)$;
2) плотность распределения $f_Z(x)$ и постройте график плотности;
3) вероятность $\P(0,\!1\leqslant Z\leqslant 3,\!714)$.



\item



Случайные величины $X$ и $Y$ независимы и имеют равномерное
распределение на отрезках $[0;6]$ и $[0;9]$ соответственно. Для случайной величины $Z=\frac{Y}{X}$ найдите: 
1) функцию распределения $F_Z(x)$;
2) плотность распределения $f_Z(x)$ и постройте график плотности;
3) вероятность $\P(1,\!179\leqslant Z\leqslant 2,\!754)$.



\item



Случайные величины $X$ и $Y$ независимы и имеют равномерное
распределение на отрезках $[0;7]$ и $[0;3]$ соответственно. Для случайной величины $Z=\frac{Y}{X}$ найдите: 
1) функцию распределения $F_Z(x)$;
2) плотность распределения $f_Z(x)$ и постройте график плотности;
3) вероятность $\P(0,\!006\leqslant Z\leqslant 0,\!519)$.



\item



Случайные величины $X$ и $Y$ независимы и имеют равномерное
распределение на отрезках $[0;9]$ и $[0;3]$ соответственно. Для случайной величины $Z=\frac{Y}{X}$ найдите: 
1) функцию распределения $F_Z(x)$;
2) плотность распределения $f_Z(x)$ и постройте график плотности;
3) вероятность $\P(0,\!059\leqslant Z\leqslant 0,\!348)$.



\item



Случайные величины $X$ и $Y$ независимы и имеют равномерное
распределение на отрезках $[0;2]$ и $[0;6]$ соответственно. Для случайной величины $Z=\frac{Y}{X}$ найдите: 
1) функцию распределения $F_Z(x)$;
2) плотность распределения $f_Z(x)$ и постройте график плотности;
3) вероятность $\P(2,\!532\leqslant Z\leqslant 4,\!716)$.



\item



Случайные величины $X$ и $Y$ независимы и имеют равномерное
распределение на отрезках $[0;1]$ и $[0;3]$ соответственно. Для случайной величины $Z=\frac{Y}{X}$ найдите: 
1) функцию распределения $F_Z(x)$;
2) плотность распределения $f_Z(x)$ и постройте график плотности;
3) вероятность $\P(0,\!039\leqslant Z\leqslant 5,\!208)$.



\item



Случайные величины $X$ и $Y$ независимы и имеют равномерное
распределение на отрезках $[0;7]$ и $[0;8]$ соответственно. Для случайной величины $Z=\frac{Y}{X}$ найдите: 
1) функцию распределения $F_Z(x)$;
2) плотность распределения $f_Z(x)$ и постройте график плотности;
3) вероятность $\P(1,\!072\leqslant Z\leqslant 1,\!953)$.



\item



Случайные величины $X$ и $Y$ независимы и имеют равномерное
распределение на отрезках $[0;3]$ и $[0;10]$ соответственно. Для случайной величины $Z=\frac{Y}{X}$ найдите: 
1) функцию распределения $F_Z(x)$;
2) плотность распределения $f_Z(x)$ и постройте график плотности;
3) вероятность $\P(3,\!263\leqslant Z\leqslant 5,\!35)$.



\item



Случайные величины $X$ и $Y$ независимы и имеют равномерное
распределение на отрезках $[0;2]$ и $[0;8]$ соответственно. Для случайной величины $Z=\frac{Y}{X}$ найдите: 
1) функцию распределения $F_Z(x)$;
2) плотность распределения $f_Z(x)$ и постройте график плотности;
3) вероятность $\P(2,\!016\leqslant Z\leqslant 6,\!716)$.



\item



Случайные величины $X$ и $Y$ независимы и имеют равномерное
распределение на отрезках $[0;3]$ и $[0;5]$ соответственно. Для случайной величины $Z=\frac{Y}{X}$ найдите: 
1) функцию распределения $F_Z(x)$;
2) плотность распределения $f_Z(x)$ и постройте график плотности;
3) вероятность $\P(0,\!915\leqslant Z\leqslant 2,\!783)$.



\item

(10) Сформулируйте лемму Неймана-Пирсона в случае проверки двух простых гипотез. Приведите
пример построения наиболее мощного критерия.



\item


(10) По выборке $X _{1}, X _{2}...X _{n}$ объема $n$ из нормального закона распределения $N(\mu, \sigma^{2})$, когда $\sigma^{2} = \mathbb{V}ar(X)$
– неизвестна, проверяется на уровне значимости $\alpha$ основная гипотеза $H _{0} : \mu = \mu _{0}$ против альтернативной $H _{1} : \mu > \mu _{0}$
гипотезы . 1) Приведите необходимую статистику критерия и критическое множество для проверки $H _{0}$
против $H _{1}$. 2) Приведите (с доказательством) основные свойства критерия. 3) Приведите (с выводом)
выражение для $P$-значения критерия. 4) Приведите (с выводом и необходимыми пояснениями в обозначениях)
выражения для вероятности ошибки второго рода $\beta$ и мощности критерия $W$. 5) Является ли критерий: а)
состоятельным; б) несмещенным? Ответ обосновать



\item


(10) Сформулируйте критерий независимости $\chi ^ {2}$ – Пирсона. Приведите (с выводом и
необходимыми пояснениями в обозначениях) явный вид статистики критерия в случае, когда 
таблица сопряженности двух признаков $X$ и $Y$ имеет вид

\begin{tabular}[b]{ | c | c | c | }
\hline
$ $ & $Y = y _{1}$ & $Y = y _{2}$  \\ \hline
$X = x _{1}$ & $a$ & $b$ \\ \hline
$X = x _{2}$ & $c$ & $d$ \\
\hline
\end{tabular}



\item

%\folder 1.pdf
(10) Известно, что доля возвратов по кредитам в банке имеет распределение $F(x) = x ^{\beta}, 0 \leqslant x \leqslant 1$.
Наблюдения показали, что в среднем она составляет $91,6667\%$. Методом моментов оцените параметр $\beta$ и
вероятность того, что она опуститься ниже $59\%$



\item


%\folder 2.pdf
(10) Известно, что доля возвратов по кредитам в банке имеет распределение $F(x) = x ^{\beta}, 0 \leqslant x \leqslant 1$.
Наблюдения показали, что в среднем она составляет $75,0\%$. Методом моментов оцените параметр $\beta$ и
вероятность того, что она опуститься ниже $20\%$



\item


(10) Известно, что доля возвратов по кредитам в банке имеет распределение $F(x) = x ^{\beta}, 0 \leqslant x \leqslant 1$.
Наблюдения показали, что в среднем она составляет $91,6667\%$. Методом моментов оцените параметр $\beta$ и
вероятность того, что она опуститься ниже $59\%$



\item


(10) Известно, что доля возвратов по кредитам в банке имеет распределение $F(x) = x ^{\beta}, 0 \leqslant x \leqslant 1$.
Наблюдения показали, что в среднем она составляет $75,0\%$. Методом моментов оцените параметр $\beta$ и
вероятность того, что она опуститься ниже $20\%$



\item


(10) Известно, что доля возвратов по кредитам в банке имеет распределение $F(x) = x ^{\beta}, 0 \leqslant x \leqslant 1$.
Наблюдения показали, что в среднем она составляет $87,5\%$. Методом моментов оцените параметр $\beta$ и
вероятность того, что она опуститься ниже $53\%$



\item


(10) Известно, что доля возвратов по кредитам в банке имеет распределение $F(x) = x ^{\beta}, 0 \leqslant x \leqslant 1$.
Наблюдения показали, что в среднем она составляет $87,5\%$. Методом моментов оцените параметр $\beta$ и
вероятность того, что она опуститься ниже $17\%$



\item


(10) Известно, что доля возвратов по кредитам в банке имеет распределение $F(x) = x ^{\beta}, 0 \leqslant x \leqslant 1$.
Наблюдения показали, что в среднем она составляет $75,0\%$. Методом моментов оцените параметр $\beta$ и
вероятность того, что она опуститься ниже $52\%$



\item


(10) Известно, что доля возвратов по кредитам в банке имеет распределение $F(x) = x ^{\beta}, 0 \leqslant x \leqslant 1$.
Наблюдения показали, что в среднем она составляет $85,7143\%$. Методом моментов оцените параметр $\beta$ и
вероятность того, что она опуститься ниже $26\%$



\item


(10) Известно, что доля возвратов по кредитам в банке имеет распределение $F(x) = x ^{\beta}, 0 \leqslant x \leqslant 1$.
Наблюдения показали, что в среднем она составляет $85,7143\%$. Методом моментов оцените параметр $\beta$ и
вероятность того, что она опуститься ниже $96\%$



\item


(10) Известно, что доля возвратов по кредитам в банке имеет распределение $F(x) = x ^{\beta}, 0 \leqslant x \leqslant 1$.
Наблюдения показали, что в среднем она составляет $88,8889\%$. Методом моментов оцените параметр $\beta$ и
вероятность того, что она опуститься ниже $89\%$



\item


(10) Известно, что доля возвратов по кредитам в банке имеет распределение $F(x) = x ^{\beta}, 0 \leqslant x \leqslant 1$.
Наблюдения показали, что в среднем она составляет $93,3333\%$. Методом моментов оцените параметр $\beta$ и
вероятность того, что она опуститься ниже $5\%$



\item


(10) Известно, что доля возвратов по кредитам в банке имеет распределение $F(x) = x ^{\beta}, 0 \leqslant x \leqslant 1$.
Наблюдения показали, что в среднем она составляет $93,3333\%$. Методом моментов оцените параметр $\beta$ и
вероятность того, что она опуститься ниже $19\%$



\item

    
	Случайная величина Y принимает только значения из множества $\{10, 7\}$, при этом $P(Y=10) = 0.24$.
	Распределение случайной величины X определено следующим образом:
	\begin{equation*}
		X | Y =
		\begin{cases}
			$4$ * y, с вероятностью $ 0.53$ \\
			$9$ * y, с вероятностью $ 1 - 0.53$
		\end{cases}
	\end{equation*}

	Юный аналитик Дарья нашла матожидание и дисперсию $X$.

	Помогите Дарье найти матожидание и дисперсию величины $X$
	


\item

    
	Случайная величина Y принимает только значения из множества $\{1, 10\}$, при этом $P(Y=1) = 0.7$.
	Распределение случайной величины X определено следующим образом:
	\begin{equation*}
		X | Y =
		\begin{cases}
			$5$ * y, с вероятностью $ 0.11$ \\
			$8$ * y, с вероятностью $ 1 - 0.11$
		\end{cases}
	\end{equation*}

	Юный аналитик Дарья нашла матожидание и дисперсию $X$.

	Помогите Дарье найти матожидание и дисперсию величины $X$
	


\item

    
	Случайная величина Y принимает только значения из множества $\{7, 5\}$, при этом $P(Y=7) = 0.08$.
	Распределение случайной величины X определено следующим образом:
	\begin{equation*}
		X | Y =
		\begin{cases}
			$9$ * y, с вероятностью $ 0.24$ \\
			$8$ * y, с вероятностью $ 1 - 0.24$
		\end{cases}
	\end{equation*}

	Юный аналитик Дарья нашла матожидание и дисперсию $X$.

	Помогите Дарье найти матожидание и дисперсию величины $X$
	


\item

    
	Случайная величина Y принимает только значения из множества $\{2, 1\}$, при этом $P(Y=2) = 0.61$.
	Распределение случайной величины X определено следующим образом:
	\begin{equation*}
		X | Y =
		\begin{cases}
			$8$ * y, с вероятностью $ 0.15$ \\
			$6$ * y, с вероятностью $ 1 - 0.15$
		\end{cases}
	\end{equation*}

	Юный аналитик Дарья нашла матожидание и дисперсию $X$.

	Помогите Дарье найти матожидание и дисперсию величины $X$
	


\item

    
	Случайная величина Y принимает только значения из множества $\{3, 4\}$, при этом $P(Y=3) = 0.33$.
	Распределение случайной величины X определено следующим образом:
	\begin{equation*}
		X | Y =
		\begin{cases}
			$9$ * y, с вероятностью $ 0.34$ \\
			$7$ * y, с вероятностью $ 1 - 0.34$
		\end{cases}
	\end{equation*}

	Юный аналитик Дарья нашла матожидание и дисперсию $X$.

	Помогите Дарье найти матожидание и дисперсию величины $X$
	


\item

    
    Создайте эмперические совокупности  $\mathtt{\text{sin}}$ и $\mathtt{\text{cos}}$ вида $\mathtt{\text{sin}}(1),\mathtt{\text{sin}}(2), ..., \mathtt{\text{sin}}(60) $ и $\mathtt{\text{cos}}(1),\mathtt{\text{cos}}(2), ..., \mathtt{\text{cos}}(60). $

    Найдите эмпирическое среднее и эмпирическое стандартное отклонение совокупности $\mathtt{\text{sin}}$, её четвёртый эмпирический центральный момент и эмпирический эксцесс.

    Кроме того, найдите эмпирический коэффициент корреляции признаков $\mathtt{\text{sin}}$ и $\mathtt{\text{cos}}$ на совокупности натуральных чисел от $1$ до $60$.
    


\item

    
    Создайте эмперические совокупности  $\mathtt{\text{exp}}$ и $\mathtt{\text{cos}}$ вида $\mathtt{\text{exp}}(1),\mathtt{\text{exp}}(2), ..., \mathtt{\text{exp}}(57) $ и $\mathtt{\text{cos}}(1),\mathtt{\text{cos}}(2), ..., \mathtt{\text{cos}}(57). $

    Найдите эмпирическое среднее и эмпирическое стандартное отклонение совокупности $\mathtt{\text{exp}}$, её четвёртый эмпирический центральный момент и эмпирический эксцесс.

    Кроме того, найдите эмпирический коэффициент корреляции признаков $\mathtt{\text{exp}}$ и $\mathtt{\text{cos}}$ на совокупности натуральных чисел от $1$ до $57$.
    


\item

    
    Создайте эмперические совокупности  $\mathtt{\text{log}}$ и $\mathtt{\text{cos}}$ вида $\mathtt{\text{log}}(1),\mathtt{\text{log}}(2), ..., \mathtt{\text{log}}(61) $ и $\mathtt{\text{cos}}(1),\mathtt{\text{cos}}(2), ..., \mathtt{\text{cos}}(61). $

    Найдите эмпирическое среднее и эмпирическое стандартное отклонение совокупности $\mathtt{\text{log}}$, её четвёртый эмпирический центральный момент и эмпирический эксцесс.

    Кроме того, найдите эмпирический коэффициент корреляции признаков $\mathtt{\text{log}}$ и $\mathtt{\text{cos}}$ на совокупности натуральных чисел от $1$ до $61$.
    


\item

    
    Создайте эмперические совокупности  $\mathtt{\text{exp}}$ и $\mathtt{\text{sin}}$ вида $\mathtt{\text{exp}}(1),\mathtt{\text{exp}}(2), ..., \mathtt{\text{exp}}(85) $ и $\mathtt{\text{sin}}(1),\mathtt{\text{sin}}(2), ..., \mathtt{\text{sin}}(85). $

    Найдите эмпирическое среднее и эмпирическое стандартное отклонение совокупности $\mathtt{\text{exp}}$, её четвёртый эмпирический центральный момент и эмпирический эксцесс.

    Кроме того, найдите эмпирический коэффициент корреляции признаков $\mathtt{\text{exp}}$ и $\mathtt{\text{sin}}$ на совокупности натуральных чисел от $1$ до $85$.
    


\item

    
    Создайте эмперические совокупности  $\mathtt{\text{cos}}$ и $\mathtt{\text{log}}$ вида $\mathtt{\text{cos}}(1),\mathtt{\text{cos}}(2), ..., \mathtt{\text{cos}}(98) $ и $\mathtt{\text{log}}(1),\mathtt{\text{log}}(2), ..., \mathtt{\text{log}}(98). $

    Найдите эмпирическое среднее и эмпирическое стандартное отклонение совокупности $\mathtt{\text{cos}}$, её четвёртый эмпирический центральный момент и эмпирический эксцесс.

    Кроме того, найдите эмпирический коэффициент корреляции признаков $\mathtt{\text{cos}}$ и $\mathtt{\text{log}}$ на совокупности натуральных чисел от $1$ до $98$.
    


\item

    
    Создайте эмперические совокупности  $\mathtt{\text{exp}}$ и $\mathtt{\text{log}}$ вида $\mathtt{\text{exp}}(1),\mathtt{\text{exp}}(2), ..., \mathtt{\text{exp}}(100) $ и $\mathtt{\text{log}}(1),\mathtt{\text{log}}(2), ..., \mathtt{\text{log}}(100). $

    Найдите эмпирическое среднее и эмпирическое стандартное отклонение совокупности $\mathtt{\text{exp}}$, её четвёртый эмпирический центральный момент и эмпирический эксцесс.

    Кроме того, найдите эмпирический коэффициент корреляции признаков $\mathtt{\text{exp}}$ и $\mathtt{\text{log}}$ на совокупности натуральных чисел от $1$ до $100$.
    


\item

    
    Создайте эмперические совокупности  $\mathtt{\text{exp}}$ и $\mathtt{\text{log}}$ вида $\mathtt{\text{exp}}(1),\mathtt{\text{exp}}(2), ..., \mathtt{\text{exp}}(77) $ и $\mathtt{\text{log}}(1),\mathtt{\text{log}}(2), ..., \mathtt{\text{log}}(77). $

    Найдите эмпирическое среднее и эмпирическое стандартное отклонение совокупности $\mathtt{\text{exp}}$, её четвёртый эмпирический центральный момент и эмпирический эксцесс.

    Кроме того, найдите эмпирический коэффициент корреляции признаков $\mathtt{\text{exp}}$ и $\mathtt{\text{log}}$ на совокупности натуральных чисел от $1$ до $77$.
    


\item


(10) В группе $\Omega$ учатся студенты:$\omega _{1}...\omega _{25}$ . Пусть $X$ и $Y$ – 100-балльные экзаменационные оценки по
математическому анализу и теории вероятностей. Оценки $\omega _{i}$ студента обозначаются: $x _{i} = X(\omega _{i})$ и $y _{i} = Y(\omega _{i})$, $i = 1...25$. Все оценки известны
$x _{0} = 55, y _{0} = 54$, $x _{1} = 64, y _{1} = 68$, $x _{2} = 34, y _{2} = 51$, $x _{3} = 48, y _{3} = 73$, $x _{4} = 81, y _{4} = 69$, $x _{5} = 62, y _{5} = 69$, $x _{6} = 76, y _{6} = 59$, $x _{7} = 84, y _{7} = 45$, $x _{8} = 97, y _{8} = 77$, $x _{9} = 76, y _{9} = 87$, $x _{10} = 43, y _{10} = 67$, $x _{11} = 33, y _{11} = 55$, $x _{12} = 71, y _{12} = 96$, $x _{13} = 62, y _{13} = 97$, $x _{14} = 84, y _{14} = 37$, $x _{15} = 41, y _{15} = 70$, $x _{16} = 92, y _{16} = 41$, $x _{17} = 60, y _{17} = 54$, $x _{18} = 71, y _{18} = 44$, $x _{19} = 39, y _{19} = 70$, $x _{20} = 98, y _{20} = 75$, $x _{21} = 99, y _{21} = 32$, $x _{22} = 58, y _{22} = 42$, $x _{23} = 61, y _{23} = 92$, $x _{24} = 58, y _{24} = 32$
Требуется
найти следующие условные эмпирические характеристики: 1) ковариацию $X$ и $Y$ при условии, что одновременно $X \geqslant 50$
 и $Y \geqslant 50$; 2) коэффициент корреляции $X$ и $Y$ при том же условии.



\item


(10) В группе $\Omega$ учатся студенты:$\omega _{1}...\omega _{25}$ . Пусть $X$ и $Y$ – 100-балльные экзаменационные оценки по
математическому анализу и теории вероятностей. Оценки $\omega _{i}$ студента обозначаются: $x _{i} = X(\omega _{i})$ и $y _{i} = Y(\omega _{i})$, $i = 1...25$. Все оценки известны
$x _{0} = 37, y _{0} = 77$, $x _{1} = 59, y _{1} = 94$, $x _{2} = 40, y _{2} = 37$, $x _{3} = 41, y _{3} = 52$, $x _{4} = 96, y _{4} = 55$, $x _{5} = 52, y _{5} = 55$, $x _{6} = 70, y _{6} = 77$, $x _{7} = 38, y _{7} = 83$, $x _{8} = 70, y _{8} = 73$, $x _{9} = 31, y _{9} = 89$, $x _{10} = 67, y _{10} = 93$, $x _{11} = 47, y _{11} = 41$, $x _{12} = 46, y _{12} = 51$, $x _{13} = 91, y _{13} = 45$, $x _{14} = 33, y _{14} = 44$, $x _{15} = 86, y _{15} = 83$, $x _{16} = 30, y _{16} = 57$, $x _{17} = 54, y _{17} = 97$, $x _{18} = 54, y _{18} = 85$, $x _{19} = 81, y _{19} = 95$, $x _{20} = 48, y _{20} = 67$, $x _{21} = 54, y _{21} = 75$, $x _{22} = 61, y _{22} = 92$, $x _{23} = 64, y _{23} = 34$, $x _{24} = 38, y _{24} = 88$
Требуется
найти следующие условные эмпирические характеристики: 1) ковариацию $X$ и $Y$ при условии, что одновременно $X \geqslant 50$
 и $Y \geqslant 50$; 2) коэффициент корреляции $X$ и $Y$ при том же условии.



\item


(10) В группе $\Omega$ учатся студенты:$\omega _{1}...\omega _{25}$ . Пусть $X$ и $Y$ – 100-балльные экзаменационные оценки по
математическому анализу и теории вероятностей. Оценки $\omega _{i}$ студента обозначаются: $x _{i} = X(\omega _{i})$ и $y _{i} = Y(\omega _{i})$, $i = 1...25$. Все оценки известны
$x _{0} = 73, y _{0} = 44$, $x _{1} = 44, y _{1} = 83$, $x _{2} = 49, y _{2} = 41$, $x _{3} = 36, y _{3} = 32$, $x _{4} = 48, y _{4} = 60$, $x _{5} = 53, y _{5} = 37$, $x _{6} = 70, y _{6} = 86$, $x _{7} = 61, y _{7} = 82$, $x _{8} = 42, y _{8} = 57$, $x _{9} = 94, y _{9} = 40$, $x _{10} = 44, y _{10} = 78$, $x _{11} = 85, y _{11} = 78$, $x _{12} = 48, y _{12} = 66$, $x _{13} = 88, y _{13} = 82$, $x _{14} = 31, y _{14} = 39$, $x _{15} = 84, y _{15} = 68$, $x _{16} = 49, y _{16} = 51$, $x _{17} = 84, y _{17} = 55$, $x _{18} = 65, y _{18} = 67$, $x _{19} = 37, y _{19} = 99$, $x _{20} = 46, y _{20} = 31$, $x _{21} = 84, y _{21} = 46$, $x _{22} = 40, y _{22} = 67$, $x _{23} = 86, y _{23} = 54$, $x _{24} = 89, y _{24} = 32$
Требуется
найти следующие условные эмпирические характеристики: 1) ковариацию $X$ и $Y$ при условии, что одновременно $X \geqslant 50$
 и $Y \geqslant 50$; 2) коэффициент корреляции $X$ и $Y$ при том же условии.



\item


(10) В группе $\Omega$ учатся студенты:$\omega _{1}...\omega _{25}$ . Пусть $X$ и $Y$ – 100-балльные экзаменационные оценки по
математическому анализу и теории вероятностей. Оценки $\omega _{i}$ студента обозначаются: $x _{i} = X(\omega _{i})$ и $y _{i} = Y(\omega _{i})$, $i = 1...25$. Все оценки известны
$x _{0} = 33, y _{0} = 72$, $x _{1} = 94, y _{1} = 94$, $x _{2} = 91, y _{2} = 52$, $x _{3} = 47, y _{3} = 59$, $x _{4} = 53, y _{4} = 45$, $x _{5} = 96, y _{5} = 54$, $x _{6} = 60, y _{6} = 99$, $x _{7} = 70, y _{7} = 44$, $x _{8} = 50, y _{8} = 81$, $x _{9} = 57, y _{9} = 40$, $x _{10} = 99, y _{10} = 61$, $x _{11} = 94, y _{11} = 43$, $x _{12} = 85, y _{12} = 96$, $x _{13} = 30, y _{13} = 91$, $x _{14} = 57, y _{14} = 37$, $x _{15} = 42, y _{15} = 35$, $x _{16} = 84, y _{16} = 75$, $x _{17} = 96, y _{17} = 97$, $x _{18} = 69, y _{18} = 92$, $x _{19} = 91, y _{19} = 93$, $x _{20} = 45, y _{20} = 30$, $x _{21} = 35, y _{21} = 94$, $x _{22} = 83, y _{22} = 53$, $x _{23} = 53, y _{23} = 60$, $x _{24} = 36, y _{24} = 69$
Требуется
найти следующие условные эмпирические характеристики: 1) ковариацию $X$ и $Y$ при условии, что одновременно $X \geqslant 50$
 и $Y \geqslant 50$; 2) коэффициент корреляции $X$ и $Y$ при том же условии.



\item


(10) В группе $\Omega$ учатся студенты:$\omega _{1}...\omega _{25}$ . Пусть $X$ и $Y$ – 100-балльные экзаменационные оценки по
математическому анализу и теории вероятностей. Оценки $\omega _{i}$ студента обозначаются: $x _{i} = X(\omega _{i})$ и $y _{i} = Y(\omega _{i})$, $i = 1...25$. Все оценки известны
$x _{0} = 55, y _{0} = 55$, $x _{1} = 88, y _{1} = 86$, $x _{2} = 42, y _{2} = 96$, $x _{3} = 69, y _{3} = 93$, $x _{4} = 43, y _{4} = 64$, $x _{5} = 42, y _{5} = 86$, $x _{6} = 35, y _{6} = 45$, $x _{7} = 60, y _{7} = 55$, $x _{8} = 41, y _{8} = 90$, $x _{9} = 62, y _{9} = 57$, $x _{10} = 52, y _{10} = 53$, $x _{11} = 67, y _{11} = 32$, $x _{12} = 72, y _{12} = 98$, $x _{13} = 42, y _{13} = 84$, $x _{14} = 97, y _{14} = 51$, $x _{15} = 32, y _{15} = 89$, $x _{16} = 38, y _{16} = 84$, $x _{17} = 42, y _{17} = 84$, $x _{18} = 61, y _{18} = 94$, $x _{19} = 96, y _{19} = 31$, $x _{20} = 67, y _{20} = 56$, $x _{21} = 66, y _{21} = 67$, $x _{22} = 41, y _{22} = 95$, $x _{23} = 54, y _{23} = 95$, $x _{24} = 36, y _{24} = 80$
Требуется
найти следующие условные эмпирические характеристики: 1) ковариацию $X$ и $Y$ при условии, что одновременно $X \geqslant 50$
 и $Y \geqslant 50$; 2) коэффициент корреляции $X$ и $Y$ при том же условии.



\item


(10) В группе $\Omega$ учатся студенты:$\omega _{1}...\omega _{25}$ . Пусть $X$ и $Y$ – 100-балльные экзаменационные оценки по
математическому анализу и теории вероятностей. Оценки $\omega _{i}$ студента обозначаются: $x _{i} = X(\omega _{i})$ и $y _{i} = Y(\omega _{i})$, $i = 1...25$. Все оценки известны
$x _{0} = 64, y _{0} = 84$, $x _{1} = 82, y _{1} = 42$, $x _{2} = 51, y _{2} = 99$, $x _{3} = 68, y _{3} = 57$, $x _{4} = 90, y _{4} = 71$, $x _{5} = 89, y _{5} = 55$, $x _{6} = 55, y _{6} = 55$, $x _{7} = 90, y _{7} = 58$, $x _{8} = 61, y _{8} = 78$, $x _{9} = 38, y _{9} = 84$, $x _{10} = 56, y _{10} = 95$, $x _{11} = 86, y _{11} = 69$, $x _{12} = 71, y _{12} = 72$, $x _{13} = 35, y _{13} = 99$, $x _{14} = 82, y _{14} = 67$, $x _{15} = 79, y _{15} = 59$, $x _{16} = 83, y _{16} = 88$, $x _{17} = 45, y _{17} = 75$, $x _{18} = 70, y _{18} = 79$, $x _{19} = 89, y _{19} = 80$, $x _{20} = 33, y _{20} = 30$, $x _{21} = 63, y _{21} = 73$, $x _{22} = 55, y _{22} = 53$, $x _{23} = 31, y _{23} = 78$, $x _{24} = 50, y _{24} = 90$
Требуется
найти следующие условные эмпирические характеристики: 1) ковариацию $X$ и $Y$ при условии, что одновременно $X \geqslant 50$
 и $Y \geqslant 50$; 2) коэффициент корреляции $X$ и $Y$ при том же условии.



\item


(10) В группе $\Omega$ учатся студенты:$\omega _{1}...\omega _{25}$ . Пусть $X$ и $Y$ – 100-балльные экзаменационные оценки по
математическому анализу и теории вероятностей. Оценки $\omega _{i}$ студента обозначаются: $x _{i} = X(\omega _{i})$ и $y _{i} = Y(\omega _{i})$, $i = 1...25$. Все оценки известны
$x _{0} = 60, y _{0} = 64$, $x _{1} = 80, y _{1} = 79$, $x _{2} = 99, y _{2} = 66$, $x _{3} = 30, y _{3} = 82$, $x _{4} = 34, y _{4} = 38$, $x _{5} = 69, y _{5} = 32$, $x _{6} = 58, y _{6} = 79$, $x _{7} = 99, y _{7} = 66$, $x _{8} = 33, y _{8} = 49$, $x _{9} = 56, y _{9} = 91$, $x _{10} = 64, y _{10} = 89$, $x _{11} = 95, y _{11} = 80$, $x _{12} = 97, y _{12} = 85$, $x _{13} = 33, y _{13} = 39$, $x _{14} = 34, y _{14} = 68$, $x _{15} = 90, y _{15} = 61$, $x _{16} = 30, y _{16} = 94$, $x _{17} = 59, y _{17} = 53$, $x _{18} = 45, y _{18} = 90$, $x _{19} = 61, y _{19} = 71$, $x _{20} = 85, y _{20} = 87$, $x _{21} = 44, y _{21} = 46$, $x _{22} = 79, y _{22} = 36$, $x _{23} = 36, y _{23} = 47$, $x _{24} = 70, y _{24} = 36$
Требуется
найти следующие условные эмпирические характеристики: 1) ковариацию $X$ и $Y$ при условии, что одновременно $X \geqslant 50$
 и $Y \geqslant 50$; 2) коэффициент корреляции $X$ и $Y$ при том же условии.



\item


(10) В группе $\Omega$ учатся студенты:$\omega _{1}...\omega _{25}$ . Пусть $X$ и $Y$ – 100-балльные экзаменационные оценки по
математическому анализу и теории вероятностей. Оценки $\omega _{i}$ студента обозначаются: $x _{i} = X(\omega _{i})$ и $y _{i} = Y(\omega _{i})$, $i = 1...25$. Все оценки известны
$x _{0} = 40, y _{0} = 84$, $x _{1} = 83, y _{1} = 71$, $x _{2} = 85, y _{2} = 64$, $x _{3} = 77, y _{3} = 32$, $x _{4} = 86, y _{4} = 59$, $x _{5} = 99, y _{5} = 77$, $x _{6} = 91, y _{6} = 74$, $x _{7} = 46, y _{7} = 48$, $x _{8} = 73, y _{8} = 42$, $x _{9} = 82, y _{9} = 89$, $x _{10} = 40, y _{10} = 43$, $x _{11} = 60, y _{11} = 31$, $x _{12} = 81, y _{12} = 57$, $x _{13} = 88, y _{13} = 50$, $x _{14} = 34, y _{14} = 31$, $x _{15} = 45, y _{15} = 63$, $x _{16} = 38, y _{16} = 45$, $x _{17} = 34, y _{17} = 92$, $x _{18} = 92, y _{18} = 83$, $x _{19} = 88, y _{19} = 56$, $x _{20} = 60, y _{20} = 36$, $x _{21} = 85, y _{21} = 59$, $x _{22} = 60, y _{22} = 87$, $x _{23} = 30, y _{23} = 53$, $x _{24} = 56, y _{24} = 73$
Требуется
найти следующие условные эмпирические характеристики: 1) ковариацию $X$ и $Y$ при условии, что одновременно $X \geqslant 50$
 и $Y \geqslant 50$; 2) коэффициент корреляции $X$ и $Y$ при том же условии.



\item


(10) В группе $\Omega$ учатся студенты:$\omega _{1}...\omega _{25}$ . Пусть $X$ и $Y$ – 100-балльные экзаменационные оценки по
математическому анализу и теории вероятностей. Оценки $\omega _{i}$ студента обозначаются: $x _{i} = X(\omega _{i})$ и $y _{i} = Y(\omega _{i})$, $i = 1...25$. Все оценки известны
$x _{0} = 32, y _{0} = 89$, $x _{1} = 61, y _{1} = 91$, $x _{2} = 64, y _{2} = 88$, $x _{3} = 97, y _{3} = 55$, $x _{4} = 66, y _{4} = 84$, $x _{5} = 78, y _{5} = 56$, $x _{6} = 62, y _{6} = 60$, $x _{7} = 73, y _{7} = 42$, $x _{8} = 40, y _{8} = 59$, $x _{9} = 86, y _{9} = 80$, $x _{10} = 76, y _{10} = 33$, $x _{11} = 56, y _{11} = 64$, $x _{12} = 87, y _{12} = 86$, $x _{13} = 70, y _{13} = 38$, $x _{14} = 87, y _{14} = 76$, $x _{15} = 72, y _{15} = 63$, $x _{16} = 79, y _{16} = 41$, $x _{17} = 33, y _{17} = 74$, $x _{18} = 67, y _{18} = 71$, $x _{19} = 65, y _{19} = 34$, $x _{20} = 57, y _{20} = 56$, $x _{21} = 63, y _{21} = 87$, $x _{22} = 68, y _{22} = 95$, $x _{23} = 46, y _{23} = 94$, $x _{24} = 50, y _{24} = 73$
Требуется
найти следующие условные эмпирические характеристики: 1) ковариацию $X$ и $Y$ при условии, что одновременно $X \geqslant 50$
 и $Y \geqslant 50$; 2) коэффициент корреляции $X$ и $Y$ при том же условии.



\item


(10) В группе $\Omega$ учатся студенты:$\omega _{1}...\omega _{25}$ . Пусть $X$ и $Y$ – 100-балльные экзаменационные оценки по
математическому анализу и теории вероятностей. Оценки $\omega _{i}$ студента обозначаются: $x _{i} = X(\omega _{i})$ и $y _{i} = Y(\omega _{i})$, $i = 1...25$. Все оценки известны
$x _{0} = 97, y _{0} = 80$, $x _{1} = 45, y _{1} = 92$, $x _{2} = 41, y _{2} = 62$, $x _{3} = 56, y _{3} = 75$, $x _{4} = 88, y _{4} = 53$, $x _{5} = 45, y _{5} = 93$, $x _{6} = 91, y _{6} = 71$, $x _{7} = 31, y _{7} = 62$, $x _{8} = 57, y _{8} = 69$, $x _{9} = 48, y _{9} = 84$, $x _{10} = 33, y _{10} = 82$, $x _{11} = 95, y _{11} = 34$, $x _{12} = 94, y _{12} = 40$, $x _{13} = 58, y _{13} = 78$, $x _{14} = 64, y _{14} = 60$, $x _{15} = 81, y _{15} = 47$, $x _{16} = 57, y _{16} = 55$, $x _{17} = 30, y _{17} = 93$, $x _{18} = 51, y _{18} = 52$, $x _{19} = 99, y _{19} = 88$, $x _{20} = 47, y _{20} = 60$, $x _{21} = 78, y _{21} = 31$, $x _{22} = 61, y _{22} = 37$, $x _{23} = 91, y _{23} = 81$, $x _{24} = 39, y _{24} = 98$
Требуется
найти следующие условные эмпирические характеристики: 1) ковариацию $X$ и $Y$ при условии, что одновременно $X \geqslant 50$
 и $Y \geqslant 50$; 2) коэффициент корреляции $X$ и $Y$ при том же условии.



\item

    
    	Распределение результатов экзамена в некоторой стране с $14$-балльной системой оценивания задано следующим образом:
    	$\left\{ 1 : 3, \  2 : 7, \  3 : 5, \  4 : 2, \  5 : 11, \  6 : 9, \  7 : 2, \  8 : 19, \  9 : 23, \  10 : 26, \  11 : 15, \  12 : 9, \  13 : 20, \  14 : 41\right\}$

	Работы будут перепроверять $16$ преподавателей, которые разделили все имеющиеся работы между собой случайным образом. Пусть $\overline{X}$ - средний балл (по перепроверки) работ, попавших к одному преподавателю.

	Требуется найти матожидание и стандартное отклонение среднего балла работ, попавших к одному преподавателю, до перепроверки.
    


\item

    
    	Распределение результатов экзамена в некоторой стране с $11$-балльной системой оценивания задано следующим образом:
    	$\left\{ 1 : 13, \  2 : 3, \  3 : 14, \  4 : 9, \  5 : 6, \  6 : 15, \  7 : 1, \  8 : 22, \  9 : 17, \  10 : 10, \  11 : 16\right\}$

	Работы будут перепроверять $6$ преподавателей, которые разделили все имеющиеся работы между собой случайным образом. Пусть $\overline{X}$ - средний балл (по перепроверки) работ, попавших к одному преподавателю.

	Требуется найти матожидание и стандартное отклонение среднего балла работ, попавших к одному преподавателю, до перепроверки.
    


\item

    
    	Распределение результатов экзамена в некоторой стране с $10$-балльной системой оценивания задано следующим образом:
    	$\left\{ 1 : 6, \  2 : 16, \  3 : 9, \  4 : 16, \  5 : 14, \  6 : 4, \  7 : 25, \  8 : 26, \  9 : 24, \  10 : 10\right\}$

	Работы будут перепроверять $10$ преподавателей, которые разделили все имеющиеся работы между собой случайным образом. Пусть $\overline{X}$ - средний балл (по перепроверки) работ, попавших к одному преподавателю.

	Требуется найти матожидание и стандартное отклонение среднего балла работ, попавших к одному преподавателю, до перепроверки.
    


\item


(10) Эмпирическое распределение признаков $X$ и $Y$ на генеральной совокупности $\Omega$ задано таблицей частот  
 
\begin{tabular}{ | c | c | c | c | }
\hline
 & $Y = 2$ & $Y = 4$ & $Y = 5$  \\ \hline
$X = 200$ & $28$ & $13$ & $10$\\ \hline
$X = 300$ & $1$ & $12$ & $35$\\
\hline
\end{tabular}

Из $\Omega$ случайным образом без возвращения извлекаются $7$ элементов. 
Пусть $\bar X$ и $\bar Y$ – средние значения признаков на выбранных элементах. 
Требуется найти: 1) математическое ожидание $\mathbb{E}(\bar Y)$; 2) стандартное отклонение $\sigma(\bar X)$ ; 
3) ковариацию $Cov(\bar X, \bar Y)$



\item


(10) Эмпирическое распределение признаков $X$ и $Y$ на генеральной совокупности $\Omega$ задано таблицей частот  
 
\begin{tabular}{ | c | c | c | c | }
\hline
 & $Y = 2$ & $Y = 4$ & $Y = 5$  \\ \hline
$X = 200$ & $1$ & $18$ & $12$\\ \hline
$X = 300$ & $31$ & $26$ & $12$\\
\hline
\end{tabular}

Из $\Omega$ случайным образом без возвращения извлекаются $12$ элементов. 
Пусть $\bar X$ и $\bar Y$ – средние значения признаков на выбранных элементах. 
Требуется найти: 1) математическое ожидание $\mathbb{E}(\bar Y)$; 2) стандартное отклонение $\sigma(\bar X)$ ; 
3) ковариацию $Cov(\bar X, \bar Y)$



\item


(10) Эмпирическое распределение признаков $X$ и $Y$ на генеральной совокупности $\Omega$ задано таблицей частот  
 
\begin{tabular}{ | c | c | c | c | }
\hline
 & $Y = 2$ & $Y = 4$ & $Y = 5$  \\ \hline
$X = 200$ & $17$ & $3$ & $13$\\ \hline
$X = 300$ & $21$ & $23$ & $23$\\
\hline
\end{tabular}

Из $\Omega$ случайным образом без возвращения извлекаются $10$ элементов. 
Пусть $\bar X$ и $\bar Y$ – средние значения признаков на выбранных элементах. 
Требуется найти: 1) математическое ожидание $\mathbb{E}(\bar Y)$; 2) стандартное отклонение $\sigma(\bar X)$ ; 
3) ковариацию $Cov(\bar X, \bar Y)$



\item


(10) Эмпирическое распределение признаков $X$ и $Y$ на генеральной совокупности $\Omega$ задано таблицей частот  
 
\begin{tabular}{ | c | c | c | c | }
\hline
 & $Y = 2$ & $Y = 4$ & $Y = 5$  \\ \hline
$X = 200$ & $28$ & $23$ & $3$\\ \hline
$X = 300$ & $2$ & $12$ & $32$\\
\hline
\end{tabular}

Из $\Omega$ случайным образом без возвращения извлекаются $5$ элементов. 
Пусть $\bar X$ и $\bar Y$ – средние значения признаков на выбранных элементах. 
Требуется найти: 1) математическое ожидание $\mathbb{E}(\bar Y)$; 2) стандартное отклонение $\sigma(\bar X)$ ; 
3) ковариацию $Cov(\bar X, \bar Y)$



\item


(10) Эмпирическое распределение признаков $X$ и $Y$ на генеральной совокупности $\Omega$ задано таблицей частот  
 
\begin{tabular}{ | c | c | c | c | }
\hline
 & $Y = 2$ & $Y = 4$ & $Y = 5$  \\ \hline
$X = 200$ & $11$ & $26$ & $27$\\ \hline
$X = 300$ & $5$ & $10$ & $21$\\
\hline
\end{tabular}

Из $\Omega$ случайным образом без возвращения извлекаются $6$ элементов. 
Пусть $\bar X$ и $\bar Y$ – средние значения признаков на выбранных элементах. 
Требуется найти: 1) математическое ожидание $\mathbb{E}(\bar Y)$; 2) стандартное отклонение $\sigma(\bar X)$ ; 
3) ковариацию $Cov(\bar X, \bar Y)$



\item


(10) Эмпирическое распределение признаков $X$ и $Y$ на генеральной совокупности $\Omega$ задано таблицей частот  
 
\begin{tabular}{ | c | c | c | c | }
\hline
 & $Y = 2$ & $Y = 4$ & $Y = 5$  \\ \hline
$X = 200$ & $16$ & $16$ & $22$\\ \hline
$X = 300$ & $7$ & $26$ & $13$\\
\hline
\end{tabular}

Из $\Omega$ случайным образом без возвращения извлекаются $9$ элементов. 
Пусть $\bar X$ и $\bar Y$ – средние значения признаков на выбранных элементах. 
Требуется найти: 1) математическое ожидание $\mathbb{E}(\bar Y)$; 2) стандартное отклонение $\sigma(\bar X)$ ; 
3) ковариацию $Cov(\bar X, \bar Y)$



\item


(10) Эмпирическое распределение признаков $X$ и $Y$ на генеральной совокупности $\Omega$ задано таблицей частот  
 
\begin{tabular}{ | c | c | c | c | }
\hline
 & $Y = 2$ & $Y = 4$ & $Y = 5$  \\ \hline
$X = 200$ & $16$ & $19$ & $5$\\ \hline
$X = 300$ & $25$ & $10$ & $25$\\
\hline
\end{tabular}

Из $\Omega$ случайным образом без возвращения извлекаются $6$ элементов. 
Пусть $\bar X$ и $\bar Y$ – средние значения признаков на выбранных элементах. 
Требуется найти: 1) математическое ожидание $\mathbb{E}(\bar Y)$; 2) стандартное отклонение $\sigma(\bar X)$ ; 
3) ковариацию $Cov(\bar X, \bar Y)$



\item


(10) Эмпирическое распределение признаков $X$ и $Y$ на генеральной совокупности $\Omega$ задано таблицей частот  
 
\begin{tabular}{ | c | c | c | c | }
\hline
 & $Y = 2$ & $Y = 4$ & $Y = 5$  \\ \hline
$X = 200$ & $24$ & $17$ & $3$\\ \hline
$X = 300$ & $13$ & $24$ & $19$\\
\hline
\end{tabular}

Из $\Omega$ случайным образом без возвращения извлекаются $9$ элементов. 
Пусть $\bar X$ и $\bar Y$ – средние значения признаков на выбранных элементах. 
Требуется найти: 1) математическое ожидание $\mathbb{E}(\bar Y)$; 2) стандартное отклонение $\sigma(\bar X)$ ; 
3) ковариацию $Cov(\bar X, \bar Y)$



\item


(10) Эмпирическое распределение признаков $X$ и $Y$ на генеральной совокупности $\Omega$ задано таблицей частот  
 
\begin{tabular}{ | c | c | c | c | }
\hline
 & $Y = 2$ & $Y = 4$ & $Y = 5$  \\ \hline
$X = 200$ & $1$ & $6$ & $23$\\ \hline
$X = 300$ & $13$ & $30$ & $27$\\
\hline
\end{tabular}

Из $\Omega$ случайным образом без возвращения извлекаются $13$ элементов. 
Пусть $\bar X$ и $\bar Y$ – средние значения признаков на выбранных элементах. 
Требуется найти: 1) математическое ожидание $\mathbb{E}(\bar Y)$; 2) стандартное отклонение $\sigma(\bar X)$ ; 
3) ковариацию $Cov(\bar X, \bar Y)$



\item


(10) Эмпирическое распределение признаков $X$ и $Y$ на генеральной совокупности $\Omega$ задано таблицей частот  
 
\begin{tabular}{ | c | c | c | c | }
\hline
 & $Y = 2$ & $Y = 4$ & $Y = 5$  \\ \hline
$X = 200$ & $25$ & $26$ & $10$\\ \hline
$X = 300$ & $10$ & $10$ & $19$\\
\hline
\end{tabular}

Из $\Omega$ случайным образом без возвращения извлекаются $12$ элементов. 
Пусть $\bar X$ и $\bar Y$ – средние значения признаков на выбранных элементах. 
Требуется найти: 1) математическое ожидание $\mathbb{E}(\bar Y)$; 2) стандартное отклонение $\sigma(\bar X)$ ; 
3) ковариацию $Cov(\bar X, \bar Y)$



\item

    
    	Юный аналитик Дарья использовала метод Монте-Карло для исследования Дискретного случайного вектора, описанного ниже.

        \begin{tabular}{|c|c|c|c|}
	\hline
	& X=$-8$ & X=$-7$ & X=$-6$ \\
	\hline
	Y = $7$ & $0.304$ & $0.245$  &  $0.32$ \\
	\hline
	Y = $8$ & $0.005$ & $0.029$ & $0.097$  \\
	\hline
\end{tabular}

    	Дарья получила, что E(Y|X + Y = 1) = $7.0893$.
    	Проверьте, можно ли доверять результату Дарьи аналитически. Сформулируйте определение метода Монте-Карло.
    


\item

    
    	Юный аналитик Дарья использовала метод Монте-Карло для исследования Дискретного случайного вектора, описанного ниже.

        \begin{tabular}{|c|c|c|c|}
	\hline
	& X=$-3$ & X=$-2$ & X=$-1$ \\
	\hline
	Y = $2$ & $0.29$ & $0.298$  &  $0.234$ \\
	\hline
	Y = $3$ & $0.066$ & $0.03$ & $0.082$  \\
	\hline
\end{tabular}

    	Дарья получила, что E(Y|X + Y = 1) = $2.10982$.
    	Проверьте, можно ли доверять результату Дарьи аналитически. Сформулируйте определение метода Монте-Карло.
    


\item

    
    	Юный аналитик Дарья использовала метод Монте-Карло для исследования Дискретного случайного вектора, описанного ниже.

        \begin{tabular}{|c|c|c|c|}
	\hline
	& X=$-4$ & X=$-3$ & X=$-2$ \\
	\hline
	Y = $3$ & $0.07$ & $0.084$  &  $0.205$ \\
	\hline
	Y = $4$ & $0.011$ & $0.201$ & $0.429$  \\
	\hline
\end{tabular}

    	Дарья получила, что E(Y|X + Y = 1) = $3.49618$.
    	Проверьте, можно ли доверять результату Дарьи аналитически. Сформулируйте определение метода Монте-Карло.
    


\item

    
    	Юный аналитик Дарья использовала метод Монте-Карло для исследования Дискретного случайного вектора, описанного ниже.

        \begin{tabular}{|c|c|c|c|}
	\hline
	& X=$-9$ & X=$-8$ & X=$-7$ \\
	\hline
	Y = $8$ & $0.09$ & $0.005$  &  $0.23$ \\
	\hline
	Y = $9$ & $0.249$ & $0.095$ & $0.331$  \\
	\hline
\end{tabular}

    	Дарья получила, что E(Y|X + Y = 1) = $8.2921$.
    	Проверьте, можно ли доверять результату Дарьи аналитически. Сформулируйте определение метода Монте-Карло.
    


\item

    
    	Юный аналитик Дарья использовала метод Монте-Карло для исследования Дискретного случайного вектора, описанного ниже.

        \begin{tabular}{|c|c|c|c|}
	\hline
	& X=$-6$ & X=$-5$ & X=$-4$ \\
	\hline
	Y = $5$ & $0.039$ & $0.207$  &  $0.054$ \\
	\hline
	Y = $6$ & $0.035$ & $0.255$ & $0.41$  \\
	\hline
\end{tabular}

    	Дарья получила, что E(Y|X + Y = 1) = $5.82286$.
    	Проверьте, можно ли доверять результату Дарьи аналитически. Сформулируйте определение метода Монте-Карло.
    


\item

    
    	Юный аналитик Дарья использовала метод Монте-Карло для исследования Дискретного случайного вектора, описанного ниже.

        \begin{tabular}{|c|c|c|c|}
	\hline
	& X=$-5$ & X=$-4$ & X=$-3$ \\
	\hline
	Y = $4$ & $0.216$ & $0.277$  &  $0.141$ \\
	\hline
	Y = $5$ & $0.153$ & $0.025$ & $0.188$  \\
	\hline
\end{tabular}

    	Дарья получила, что E(Y|X + Y = 1) = $4.15479$.
    	Проверьте, можно ли доверять результату Дарьи аналитически. Сформулируйте определение метода Монте-Карло.
    


\item


(10) Пусть $X _{1}$, $X _{2}$, $X _{3}$, $X _{4}$ выборка из $N(\theta, \sigma ^{2})$. Рассмотрим две оценки параметра $\theta$:
\[\hat \theta _{1} = \frac{3X _{1} + X _{2} + 2X _{3} + 4X _{4}}{10}, \hat \theta _{1} = \frac{X _{1} + 6X _{2} + X _{3} + 2X _{4}}{10}\]
a) Покажите, что обе оценки несмещенные.
б) Какая из оценок оптимальная?



\item


(10) Пусть $X _{1}$, $X _{2}$, $X _{3}$, $X _{4}$ выборка из $N(\theta, \sigma ^{2})$. Рассмотрим две оценки параметра $\theta$:
\[\hat \theta _{1} = \frac{2X _{1} + 6X _{2} + X _{3} + X _{4}}{10}, \hat \theta _{1} = \frac{5X _{1} + X _{2} + X _{3} + 3X _{4}}{10}\]
a) Покажите, что обе оценки несмещенные.
б) Какая из оценок оптимальная?



\item


(10) Пусть $X _{1}$, $X _{2}$, $X _{3}$, $X _{4}$ выборка из $N(\theta, \sigma ^{2})$. Рассмотрим две оценки параметра $\theta$:
\[\hat \theta _{1} = \frac{X _{1} + X _{2} + X _{3} + 7X _{4}}{10}, \hat \theta _{1} = \frac{3X _{1} + 5X _{2} + X _{3} + X _{4}}{10}\]
a) Покажите, что обе оценки несмещенные.
б) Какая из оценок оптимальная?



\item


(10) Пусть $X _{1}$, $X _{2}$, $X _{3}$, $X _{4}$ выборка из $N(\theta, \sigma ^{2})$. Рассмотрим две оценки параметра $\theta$:
\[\hat \theta _{1} = \frac{5X _{1} + 2X _{2} + X _{3} + 2X _{4}}{10}, \hat \theta _{1} = \frac{4X _{1} + 4X _{2} + X _{3} + X _{4}}{10}\]
a) Покажите, что обе оценки несмещенные.
б) Какая из оценок оптимальная?



\item


(10) Пусть $X _{1}$, $X _{2}$, $X _{3}$, $X _{4}$ выборка из $N(\theta, \sigma ^{2})$. Рассмотрим две оценки параметра $\theta$:
\[\hat \theta _{1} = \frac{2X _{1} + 3X _{2} + 4X _{3} + X _{4}}{10}, \hat \theta _{1} = \frac{2X _{1} + 3X _{2} + 2X _{3} + 3X _{4}}{10}\]
a) Покажите, что обе оценки несмещенные.
б) Какая из оценок оптимальная?



\item


(10) Пусть $X _{1}$, $X _{2}$, $X _{3}$, $X _{4}$ выборка из $N(\theta, \sigma ^{2})$. Рассмотрим две оценки параметра $\theta$:
\[\hat \theta _{1} = \frac{X _{1} + 6X _{2} + X _{3} + 2X _{4}}{10}, \hat \theta _{1} = \frac{3X _{1} + X _{2} + 3X _{3} + 3X _{4}}{10}\]
a) Покажите, что обе оценки несмещенные.
б) Какая из оценок оптимальная?



\item


(10) Пусть $X _{1}$, $X _{2}$, $X _{3}$, $X _{4}$ выборка из $N(\theta, \sigma ^{2})$. Рассмотрим две оценки параметра $\theta$:
\[\hat \theta _{1} = \frac{X _{1} + 4X _{2} + X _{3} + 4X _{4}}{10}, \hat \theta _{1} = \frac{2X _{1} + 3X _{2} + 3X _{3} + 2X _{4}}{10}\]
a) Покажите, что обе оценки несмещенные.
б) Какая из оценок оптимальная?



\item


(10) Пусть $X _{1}$, $X _{2}$, $X _{3}$, $X _{4}$ выборка из $N(\theta, \sigma ^{2})$. Рассмотрим две оценки параметра $\theta$:
\[\hat \theta _{1} = \frac{X _{1} + X _{2} + 2X _{3} + 6X _{4}}{10}, \hat \theta _{1} = \frac{X _{1} + 5X _{2} + X _{3} + 3X _{4}}{10}\]
a) Покажите, что обе оценки несмещенные.
б) Какая из оценок оптимальная?



\item


(10) Пусть $X _{1}$, $X _{2}$, $X _{3}$, $X _{4}$ выборка из $N(\theta, \sigma ^{2})$. Рассмотрим две оценки параметра $\theta$:
\[\hat \theta _{1} = \frac{3X _{1} + X _{2} + 4X _{3} + 2X _{4}}{10}, \hat \theta _{1} = \frac{X _{1} + 6X _{2} + 2X _{3} + X _{4}}{10}\]
a) Покажите, что обе оценки несмещенные.
б) Какая из оценок оптимальная?



\item


(10) Пусть $X _{1}$, $X _{2}$, $X _{3}$, $X _{4}$ выборка из $N(\theta, \sigma ^{2})$. Рассмотрим две оценки параметра $\theta$:
\[\hat \theta _{1} = \frac{2X _{1} + X _{2} + 3X _{3} + 4X _{4}}{10}, \hat \theta _{1} = \frac{X _{1} + 5X _{2} + X _{3} + 3X _{4}}{10}\]
a) Покажите, что обе оценки несмещенные.
б) Какая из оценок оптимальная?



\item

    
	Известно, что доля возвратов по кредитам в банке имеет распределение $F(x) = x^{\beta}, 0 \le x \le 1$. Наблюдения показали, что в среднем она составила $60.0$\%. Методом моментов оцените параметр $\beta$ и вероятность того, что она опуститься ниже $52.0$\%.
	


\item

    
	Известно, что доля возвратов по кредитам в банке имеет распределение $F(x) = x^{\beta}, 0 \le x \le 1$. Наблюдения показали, что в среднем она составила $71.0$\%. Методом моментов оцените параметр $\beta$ и вероятность того, что она опуститься ниже $62.0$\%.
	


\item

    
	Известно, что доля возвратов по кредитам в банке имеет распределение $F(x) = x^{\beta}, 0 \le x \le 1$. Наблюдения показали, что в среднем она составила $67.0$\%. Методом моментов оцените параметр $\beta$ и вероятность того, что она опуститься ниже $52.0$\%.
	


\item

    
	Известно, что доля возвратов по кредитам в банке имеет распределение $F(x) = x^{\beta}, 0 \le x \le 1$. Наблюдения показали, что в среднем она составила $55.0$\%. Методом моментов оцените параметр $\beta$ и вероятность того, что она опуститься ниже $50.0$\%.
	


\item

    
	Известно, что доля возвратов по кредитам в банке имеет распределение $F(x) = x^{\beta}, 0 \le x \le 1$. Наблюдения показали, что в среднем она составила $62.0$\%. Методом моментов оцените параметр $\beta$ и вероятность того, что она опуститься ниже $59.0$\%.
	


\item

    
	Известно, что доля возвратов по кредитам в банке имеет распределение $F(x) = x^{\beta}, 0 \le x \le 1$. Наблюдения показали, что в среднем она составила $74.0$\%. Методом моментов оцените параметр $\beta$ и вероятность того, что она опуститься ниже $73.0$\%.
	


\item

    
	Известно, что доля возвратов по кредитам в банке имеет распределение $F(x) = x^{\beta}, 0 \le x \le 1$. Наблюдения показали, что в среднем она составила $76.0$\%. Методом моментов оцените параметр $\beta$ и вероятность того, что она опуститься ниже $74.0$\%.
	


\item

    
	Известно, что доля возвратов по кредитам в банке имеет распределение $F(x) = x^{\beta}, 0 \le x \le 1$. Наблюдения показали, что в среднем она составила $57.0$\%. Методом моментов оцените параметр $\beta$ и вероятность того, что она опуститься ниже $51.0$\%.
	


\end{enumerate}
\end{document}