\documentclass[a4paper,10pt]{article}
\usepackage[10pt]{extsizes}




\usepackage{cmap}					% поиск в PDF
\usepackage{mathtext} 				% русские буквы в формулах
\usepackage[T2A]{fontenc}			% кодировка
\usepackage[utf8]{inputenc}			% кодировка исходного текста
\usepackage[english,russian]{babel}	% локализация и переносы
\usepackage{ulem}                   % зачеркнутый текст
\usepackage{amssymb}			% пакет математики
\usepackage{float}
\usepackage{amsmath}
\usepackage{graphicx}
\DeclareGraphicsExtensions{.png}

%%% Страница
%\usepackage{extsizes} % Возможность сделать 14-й шрифт
\usepackage[left=1cm,right=1cm,top=1cm,bottom=1cm]{geometry} % Простой способ задавать поля
\pagestyle{empty}

\begin{document}


\begin{center}
ФЕДЕРАЛЬНОЕ ГОСУДАРСТВЕННОЕ ОБРАЗОВАТЕЛЬНОЕ БЮДЖЕТНОЕ УЧРЕЖДЕНИЕ ВЫСШЕГО ОБРАЗОВАНИЯ

    \textbf{«ФИНАНСОВЫЙ УНИВЕРСИТЕТ ПРИ ПРАВИТЕЛЬСТВЕ РОССИЙСКОЙ ФЕДЕРАЦИИ»}

Факультет информационных технологий и анализа больших данных

Департамент анализа данных и машинного обучения

\textit{
	\textbf{Дисциплина: «Теория вероятностей и математическая статистика»}}

\textit{Направление подготовки: 01.03.02 «Прикладная математика и информатика»}

\textit{Профиль: «Анализ данных и принятие решений в экономике и финансах»}

\textit{Форма обучения очная, учебный 2020/2021 год, 4 семестр}

\textbf{Билет 120}

\end{center}

\begin{enumerate}


\item


Дайте определение случайной величины, которая имеет $\chi ^{2}$-распределение с n степенями свободы.
Запишите плотность $\chi ^{2}$- распределения. Выведите формулы для математического ожидания $\mathbb{E}(X)$ и дисперсии $\mathbb{V}ar(X)$ $\chi ^{2}$-распределение с n степенями свободы. Найдите а) $\mathbb{P}(\chi _{20}^{2} > 10.9)$, где $\chi _{20}^{2}$–случайная величина, которая имеет $\chi ^{2}$– распределение с 20 степенями свободы; б) найдите 93\%
(верхнюю) точку $\chi _{0.93}^{2} (5)$ хи-квадрат распределения с 5 степенями свободы


\item



Случайные величины $X$ и $Y$ независимы и имеют равномерное
распределение на отрезках $[0;4]$ и $[0;7]$ соответственно. Для случайной величины $Z=\frac{Y}{X}$ найдите: 
1) функцию распределения $F_Z(x)$;
2) плотность распределения $f_Z(x)$ и постройте график плотности;
3) вероятность $\P(0,\!035\leqslant Z\leqslant 2,\!775)$.


\item

    
	Случайная величина Y принимает только значения из множества $\{2, 1\}$, при этом $P(Y=2) = 0.61$.
	Распределение случайной величины X определено следующим образом:
	\begin{equation*}
		X | Y =
		\begin{cases}
			$8$ * y, с вероятностью $ 0.15$ \\
			$6$ * y, с вероятностью $ 1 - 0.15$
		\end{cases}
	\end{equation*}

	Юный аналитик Дарья нашла матожидание и дисперсию $X$.

	Помогите Дарье найти матожидание и дисперсию величины $X$
	

\item


(10) В группе $\Omega$ учатся студенты:$\omega _{1}...\omega _{25}$ . Пусть $X$ и $Y$ – 100-балльные экзаменационные оценки по
математическому анализу и теории вероятностей. Оценки $\omega _{i}$ студента обозначаются: $x _{i} = X(\omega _{i})$ и $y _{i} = Y(\omega _{i})$, $i = 1...25$. Все оценки известны
$x _{0} = 40, y _{0} = 84$, $x _{1} = 83, y _{1} = 71$, $x _{2} = 85, y _{2} = 64$, $x _{3} = 77, y _{3} = 32$, $x _{4} = 86, y _{4} = 59$, $x _{5} = 99, y _{5} = 77$, $x _{6} = 91, y _{6} = 74$, $x _{7} = 46, y _{7} = 48$, $x _{8} = 73, y _{8} = 42$, $x _{9} = 82, y _{9} = 89$, $x _{10} = 40, y _{10} = 43$, $x _{11} = 60, y _{11} = 31$, $x _{12} = 81, y _{12} = 57$, $x _{13} = 88, y _{13} = 50$, $x _{14} = 34, y _{14} = 31$, $x _{15} = 45, y _{15} = 63$, $x _{16} = 38, y _{16} = 45$, $x _{17} = 34, y _{17} = 92$, $x _{18} = 92, y _{18} = 83$, $x _{19} = 88, y _{19} = 56$, $x _{20} = 60, y _{20} = 36$, $x _{21} = 85, y _{21} = 59$, $x _{22} = 60, y _{22} = 87$, $x _{23} = 30, y _{23} = 53$, $x _{24} = 56, y _{24} = 73$
Требуется
найти следующие условные эмпирические характеристики: 1) ковариацию $X$ и $Y$ при условии, что одновременно $X \geqslant 50$
 и $Y \geqslant 50$; 2) коэффициент корреляции $X$ и $Y$ при том же условии.


\item


(10) Эмпирическое распределение признаков $X$ и $Y$ на генеральной совокупности $\Omega$ задано таблицей частот  
 
\begin{tabular}{ | c | c | c | c | }
\hline
 & $Y = 2$ & $Y = 4$ & $Y = 5$  \\ \hline
$X = 200$ & $28$ & $13$ & $10$\\ \hline
$X = 300$ & $1$ & $12$ & $35$\\
\hline
\end{tabular}

Из $\Omega$ случайным образом без возвращения извлекаются $7$ элементов. 
Пусть $\bar X$ и $\bar Y$ – средние значения признаков на выбранных элементах. 
Требуется найти: 1) математическое ожидание $\mathbb{E}(\bar Y)$; 2) стандартное отклонение $\sigma(\bar X)$ ; 
3) ковариацию $Cov(\bar X, \bar Y)$


\item


(10) Пусть $X _{1}$, $X _{2}$, $X _{3}$, $X _{4}$ выборка из $N(\theta, \sigma ^{2})$. Рассмотрим две оценки параметра $\theta$:
\[\hat \theta _{1} = \frac{2X _{1} + 6X _{2} + X _{3} + X _{4}}{10}, \hat \theta _{1} = \frac{5X _{1} + X _{2} + X _{3} + 3X _{4}}{10}\]
a) Покажите, что обе оценки несмещенные.
б) Какая из оценок оптимальная?


\end{enumerate}

\begin{figure}[H]
	Подготовил
	\hfill
	\includegraphics[width=2cm]{Prepared}
	П.Е. Рябов
\end{figure}


\begin{figure}[H]
	Утверждаю:\\
	Первый заместитель\\
	руководителя департамента\\
	Дата 01.06.2021
	\hfill
	\includegraphics[width=2cm]{Approved}
	Феклин В.Г.
\end{figure}

\end{document}

