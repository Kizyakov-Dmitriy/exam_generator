\documentclass[a4paper,12pt]{article}
\usepackage[12pt]{extsizes}




\usepackage{cmap}					% поиск в PDF
\usepackage{mathtext} 				% русские буквы в формулах
\usepackage[T2A]{fontenc}			% кодировка
\usepackage[utf8]{inputenc}			% кодировка исходного текста
\usepackage[english,russian]{babel}	% локализация и переносы
\usepackage{ulem}                   % зачеркнутый текст
\usepackage{amssymb}			% пакет математики
\usepackage{float}
\usepackage{amsmath}
\usepackage{graphicx}
\DeclareGraphicsExtensions{.png}

%%% Страница
%\usepackage{extsizes} % Возможность сделать 14-й шрифт
\usepackage[left=1cm,right=1cm,top=1cm,bottom=1cm]{geometry} % Простой способ задавать поля
\pagestyle{empty}

\begin{document}


\begin{center}
ФЕДЕРАЛЬНОЕ ГОСУДАРСТВЕННОЕ ОБРАЗОВАТЕЛЬНОЕ БЮДЖЕТНОЕ УЧРЕЖДЕНИЕ ВЫСШЕГО ОБРАЗОВАНИЯ

    \textbf{«ФИНАНСОВЫЙ УНИВЕРСИТЕТ ПРИ ПРАВИТЕЛЬСТВЕ РОССИЙСКОЙ ФЕДЕРАЦИИ»}

Факультет информационных технологий и анализа больших данных

Департамент анализа данных и машинного обучения

\textit{
	\textbf{Дисциплина: «Теория вероятностей и математическая статистика»}}

\textit{Направление подготовки: 01.03.02 «Прикладная математика и информатика»}

\textit{Профиль: «Анализ данных и принятие решений в экономике и финансах»}

\textit{Форма обучения очная, учебный 2020/2021 год, 4 семестр}



\end{center}


\section{Билет 101}

\begin{enumerate}


\item


Сформулируйте определение случайной выборки из конечной генеральной совокупности. Какие
виды выборок вам известны? Перечислите (с указанием формул) основные характеристики выборочной и генеральной совокупностей




Здесь очень много исчерпывающей информации о выборках из генеральной совокупности и про различные виды выборок


\item



Случайные величины $X$ и $Y$ независимы и имеют равномерное
распределение на отрезках $[0;3]$ и $[0;5]$ соответственно. Для случайной величины $Z=\frac{Y}{X}$ найдите: 
1) функцию распределения $F_Z(x)$;
2) плотность распределения $f_Z(x)$ и постройте график плотности;
3) вероятность $\P(0,\!915\leqslant Z\leqslant 2,\!783)$.




%\folder 2_53d25.png
1) Функция распределения $F_Z(x)$ имеет вид:
$
F_Z(x)=\left\{
\begin{array}{l}
0, x\leqslant 0;\\
\frac{3 x}{10}, 0\leqslant x\leqslant \frac{5}{3}\approx 1,\!667;\\
1 - \frac{5}{6 x}, x\geqslant\frac{5}{3};
\end{array}.
\right.
$
2) Плотность распределения $f_Z(x)$ имеет вид:
$
f_Z(x)=\left\{
\begin{array}{l}
0, x<0;\\
\frac{3}{10}, 0\leqslant x\leqslant \frac{5}{3}\approx 1,\!667;\\
\frac{5}{6 x^{2}}, x\geqslant\frac{5}{3};
\end{array}.
\right.
$


\begin{figure}[H]
    \includegraphics[width=0.9\textwidth]{2_53d25}
\end{figure}


3) вероятность равна:
$
\P(0,\!915\leqslant Z\leqslant 2,\!783)=
0,\!4261.
$


\item


(10) Известно, что доля возвратов по кредитам в банке имеет распределение $F(x) = x ^{\beta}, 0 \leqslant x \leqslant 1$.
Наблюдения показали, что в среднем она составляет $87,5\%$. Методом моментов оцените параметр $\beta$ и
вероятность того, что она опуститься ниже $17\%$




Найдём плотность рапределения как интеграл от ФР, а дальше всё и вовсе простою Ответ: $410338673$


\item

    
    Создайте эмперические совокупности  $\mathtt{\text{exp}}$ и $\mathtt{\text{log}}$ вида $\mathtt{\text{exp}}(1),\mathtt{\text{exp}}(2), ..., \mathtt{\text{exp}}(100) $ и $\mathtt{\text{log}}(1),\mathtt{\text{log}}(2), ..., \mathtt{\text{log}}(100). $

    Найдите эмпирическое среднее и эмпирическое стандартное отклонение совокупности $\mathtt{\text{exp}}$, её четвёртый эмпирический центральный момент и эмпирический эксцесс.

    Кроме того, найдите эмпирический коэффициент корреляции признаков $\mathtt{\text{exp}}$ и $\mathtt{\text{log}}$ на совокупности натуральных чисел от $1$ до $100$.
    


    
    Используя

	$E(X) = sum(X) / n$

	$Var(X) = E(X^2) - [E(X)]^2$

	$\mu_4(X) = E((X-E(X))^4)$

	$Ex = \frac{\mu_4(X)}{[\sigma(X)]^4} - 3$

	$r_{xy} = \frac{E(XY) - E(X) * E(Y)}{\sigma(X) * \sigma(Y)}$

    рассчитаем искомые значения.

    Ответы: $4.25253870368928 \cdot 10^{41}, 2.85939246949767 \cdot 10^{42}, 4.98013632124489 \cdot 10^{171}, 71.49826, 0.00038$.

    

\item


(10) Эмпирическое распределение признаков $X$ и $Y$ на генеральной совокупности $\Omega$ задано таблицей частот  
 
\begin{tabular}{ | c | c | c | c | }
\hline
 & $Y = 2$ & $Y = 4$ & $Y = 5$  \\ \hline
$X = 200$ & $24$ & $17$ & $3$\\ \hline
$X = 300$ & $13$ & $24$ & $19$\\
\hline
\end{tabular}

Из $\Omega$ случайным образом без возвращения извлекаются $9$ элементов. 
Пусть $\bar X$ и $\bar Y$ – средние значения признаков на выбранных элементах. 
Требуется найти: 1) математическое ожидание $\mathbb{E}(\bar Y)$; 2) стандартное отклонение $\sigma(\bar X)$ ; 
3) ковариацию $Cov(\bar X, \bar Y)$




1) математическое ожидание $\mathbb{E}(\bar Y)$: $3.48$ 
2) стандартное отклонение $\sigma(\bar X)$: $248.8024$
3) ковариацию $Cov(\bar X, \bar Y)$: $2.0333$


\item


(10) Пусть $X _{1}$, $X _{2}$, $X _{3}$, $X _{4}$ выборка из $N(\theta, \sigma ^{2})$. Рассмотрим две оценки параметра $\theta$:
\[\hat \theta _{1} = \frac{X _{1} + X _{2} + 2X _{3} + 6X _{4}}{10}, \hat \theta _{1} = \frac{X _{1} + 5X _{2} + X _{3} + 3X _{4}}{10}\]
a) Покажите, что обе оценки несмещенные.
б) Какая из оценок оптимальная?




Обе они несмещенные, потому что в числителе выходит в сумме 10.
Какая-то точно должна быть, а может и нет....



\end{enumerate}

\section{Билет 102}

\begin{enumerate}


\item


Сформулируйте определение случайной выборки из конечной генеральной совокупности. Какие
виды выборок вам известны? Перечислите (с указанием формул) основные характеристики выборочной и генеральной совокупностей




Здесь очень много исчерпывающей информации о выборках из генеральной совокупности и про различные виды выборок


\item



Случайные величины $X$ и $Y$ независимы и имеют равномерное
распределение на отрезках $[0;3]$ и $[0;8]$ соответственно. Для случайной величины $Z=\frac{Y}{X}$ найдите: 
1) функцию распределения $F_Z(x)$;
2) плотность распределения $f_Z(x)$ и постройте график плотности;
3) вероятность $\P(2,\!475\leqslant Z\leqslant 4,\!811)$.




%\folder 2_53d8.png
1) Функция распределения $F_Z(x)$ имеет вид:
$
F_Z(x)=\left\{
\begin{array}{l}
0, x\leqslant 0;\\
\frac{3 x}{16}, 0\leqslant x\leqslant \frac{8}{3}\approx 2,\!667;\\
1 - \frac{4}{3 x}, x\geqslant\frac{8}{3};
\end{array}.
\right.
$
2) Плотность распределения $f_Z(x)$ имеет вид:
$
f_Z(x)=\left\{
\begin{array}{l}
0, x<0;\\
\frac{3}{16}, 0\leqslant x\leqslant \frac{8}{3}\approx 2,\!667;\\
\frac{4}{3 x^{2}}, x\geqslant\frac{8}{3};
\end{array}.
\right.
$


\begin{figure}[H]
    \includegraphics[width=0.9\textwidth]{2_53d8}
\end{figure}


3) вероятность равна:
$
\P(2,\!475\leqslant Z\leqslant 4,\!811)=
0,\!25884.
$


\item

    
	Случайная величина Y принимает только значения из множества $\{1, 10\}$, при этом $P(Y=1) = 0.7$.
	Распределение случайной величины X определено следующим образом:
	\begin{equation*}
		X | Y =
		\begin{cases}
			$5$ * y, с вероятностью $ 0.11$ \\
			$8$ * y, с вероятностью $ 1 - 0.11$
		\end{cases}
	\end{equation*}

	Юный аналитик Дарья нашла матожидание и дисперсию $X$.

	Помогите Дарье найти матожидание и дисперсию величины $X$
	


	

	Первым этапом надо найти характеристики случайной величины $Y$

	$E(Y) = 1 * 0.7 + 10 * (1 - 0.7)$

	$Var(Y) = E(Y^2) - [E(Y)]^2 = 1^2 * 0.7 + 10^2 * (1 - 0.7) - [E(Y)]^2$


	Перейдем к рассмотрению характеристик условной случайно величины X

	$E(X) = E(E(X|Y)) = E[E(5 * Y) * 0.11 + E(8 * Y) * (1 - 0.11)] = E(Y) * (5 * 0.11 + 8 * (1 - 0.11)) = 28.379$

	$E(Var(X|Y)) = E[b * Var(c3 * Y) + (1 - b) * Var(c4 * Y)] = Var(Y) * (c3^2 * b + c4^2 * (1- b)) $

	$Var(E(X|Y)) = E(X^2|Y) - [E(X)]^2 = [E(Y)]^2 * (b * c3^2 + (1-b)*c4^2) - E(X)]^2$

	$Var(X) = E(Var(X|Y)) + Var(E(X|Y)) = 1027.72936$
	

\item


(10) В группе $\Omega$ учатся студенты:$\omega _{1}...\omega _{25}$ . Пусть $X$ и $Y$ – 100-балльные экзаменационные оценки по
математическому анализу и теории вероятностей. Оценки $\omega _{i}$ студента обозначаются: $x _{i} = X(\omega _{i})$ и $y _{i} = Y(\omega _{i})$, $i = 1...25$. Все оценки известны
$x _{0} = 33, y _{0} = 72$, $x _{1} = 94, y _{1} = 94$, $x _{2} = 91, y _{2} = 52$, $x _{3} = 47, y _{3} = 59$, $x _{4} = 53, y _{4} = 45$, $x _{5} = 96, y _{5} = 54$, $x _{6} = 60, y _{6} = 99$, $x _{7} = 70, y _{7} = 44$, $x _{8} = 50, y _{8} = 81$, $x _{9} = 57, y _{9} = 40$, $x _{10} = 99, y _{10} = 61$, $x _{11} = 94, y _{11} = 43$, $x _{12} = 85, y _{12} = 96$, $x _{13} = 30, y _{13} = 91$, $x _{14} = 57, y _{14} = 37$, $x _{15} = 42, y _{15} = 35$, $x _{16} = 84, y _{16} = 75$, $x _{17} = 96, y _{17} = 97$, $x _{18} = 69, y _{18} = 92$, $x _{19} = 91, y _{19} = 93$, $x _{20} = 45, y _{20} = 30$, $x _{21} = 35, y _{21} = 94$, $x _{22} = 83, y _{22} = 53$, $x _{23} = 53, y _{23} = 60$, $x _{24} = 36, y _{24} = 69$
Требуется
найти следующие условные эмпирические характеристики: 1) ковариацию $X$ и $Y$ при условии, что одновременно $X \geqslant 50$
 и $Y \geqslant 50$; 2) коэффициент корреляции $X$ и $Y$ при том же условии.




1) Ковариация = $-350.8333$
2) Коэффициент корреляции = $-1.2925$


\item


(10) Эмпирическое распределение признаков $X$ и $Y$ на генеральной совокупности $\Omega$ задано таблицей частот  
 
\begin{tabular}{ | c | c | c | c | }
\hline
 & $Y = 2$ & $Y = 4$ & $Y = 5$  \\ \hline
$X = 200$ & $25$ & $26$ & $10$\\ \hline
$X = 300$ & $10$ & $10$ & $19$\\
\hline
\end{tabular}

Из $\Omega$ случайным образом без возвращения извлекаются $12$ элементов. 
Пусть $\bar X$ и $\bar Y$ – средние значения признаков на выбранных элементах. 
Требуется найти: 1) математическое ожидание $\mathbb{E}(\bar Y)$; 2) стандартное отклонение $\sigma(\bar X)$ ; 
3) ковариацию $Cov(\bar X, \bar Y)$




1) математическое ожидание $\mathbb{E}(\bar Y)$: $3.59$ 
2) стандартное отклонение $\sigma(\bar X)$: $228.8693$
3) ковариацию $Cov(\bar X, \bar Y)$: $1.3324$


\item

    
	Известно, что доля возвратов по кредитам в банке имеет распределение $F(x) = x^{\beta}, 0 \le x \le 1$. Наблюдения показали, что в среднем она составила $71.0$\%. Методом моментов оцените параметр $\beta$ и вероятность того, что она опуститься ниже $62.0$\%.
	


	

	$f(x) = F'(x) = \beta \cdot x^{\beta - 1}$

	$\mu_{1} = E(X) = \int_{-\inf}^{\inf}x \cdot f(x) = \int_{-\inf}^{\inf} \beta \cdot x^{\beta} = \beta \cdot \frac{x^{\beta + 1}}{\beta + 1}\bigg|_0^1 = \frac{\beta}{\beta + 1}$

	$\beta = (\beta + 1) \cdot 71.0$

	$\beta = \frac{71.0}{1 - 71.0}$

	$ P(x \le 62.0) = F(62.0) = 62.0^{2.45} $

    Ответ: $2.45, 0.31$
	


\end{enumerate}

\section{Билет 103}

\begin{enumerate}


\item


Сформулируйте определение случайной выборки из конечной генеральной совокупности. Какие
виды выборок вам известны? Перечислите (с указанием формул) основные характеристики выборочной и генеральной совокупностей




Здесь очень много исчерпывающей информации о выборках из генеральной совокупности и про различные виды выборок


\item



Случайные величины $X$ и $Y$ независимы и имеют равномерное
распределение на отрезках $[0;1]$ и $[0;10]$ соответственно. Для случайной величины $Z=\frac{Y}{X}$ найдите: 
1) функцию распределения $F_Z(x)$;
2) плотность распределения $f_Z(x)$ и постройте график плотности;
3) вероятность $\P(2,\!96\leqslant Z\leqslant 17,\!91)$.




%\folder 2_53d13.png
1) Функция распределения $F_Z(x)$ имеет вид:
$
F_Z(x)=\left\{
\begin{array}{l}
0, x\leqslant 0;\\
\frac{x}{20}, 0\leqslant x\leqslant 10\approx 10,\!0;\\
1 - \frac{5}{x}, x\geqslant10;
\end{array}.
\right.
$
2) Плотность распределения $f_Z(x)$ имеет вид:
$
f_Z(x)=\left\{
\begin{array}{l}
0, x<0;\\
\frac{1}{20}, 0\leqslant x\leqslant 10\approx 10,\!0;\\
\frac{5}{x^{2}}, x\geqslant10;
\end{array}.
\right.
$


\begin{figure}[H]
    \includegraphics[width=0.9\textwidth]{2_53d13}
\end{figure}


3) вероятность равна:
$
\P(2,\!96\leqslant Z\leqslant 17,\!91)=
0,\!57283.
$


\item

%\folder 1.pdf
(10) Известно, что доля возвратов по кредитам в банке имеет распределение $F(x) = x ^{\beta}, 0 \leqslant x \leqslant 1$.
Наблюдения показали, что в среднем она составляет $91,6667\%$. Методом моментов оцените параметр $\beta$ и
вероятность того, что она опуститься ниже $59\%$




Найдём плотность рапределения как интеграл от ФР, а дальше всё и вовсе простою Ответ: $30155888444737842659$


\item


(10) В группе $\Omega$ учатся студенты:$\omega _{1}...\omega _{25}$ . Пусть $X$ и $Y$ – 100-балльные экзаменационные оценки по
математическому анализу и теории вероятностей. Оценки $\omega _{i}$ студента обозначаются: $x _{i} = X(\omega _{i})$ и $y _{i} = Y(\omega _{i})$, $i = 1...25$. Все оценки известны
$x _{0} = 73, y _{0} = 44$, $x _{1} = 44, y _{1} = 83$, $x _{2} = 49, y _{2} = 41$, $x _{3} = 36, y _{3} = 32$, $x _{4} = 48, y _{4} = 60$, $x _{5} = 53, y _{5} = 37$, $x _{6} = 70, y _{6} = 86$, $x _{7} = 61, y _{7} = 82$, $x _{8} = 42, y _{8} = 57$, $x _{9} = 94, y _{9} = 40$, $x _{10} = 44, y _{10} = 78$, $x _{11} = 85, y _{11} = 78$, $x _{12} = 48, y _{12} = 66$, $x _{13} = 88, y _{13} = 82$, $x _{14} = 31, y _{14} = 39$, $x _{15} = 84, y _{15} = 68$, $x _{16} = 49, y _{16} = 51$, $x _{17} = 84, y _{17} = 55$, $x _{18} = 65, y _{18} = 67$, $x _{19} = 37, y _{19} = 99$, $x _{20} = 46, y _{20} = 31$, $x _{21} = 84, y _{21} = 46$, $x _{22} = 40, y _{22} = 67$, $x _{23} = 86, y _{23} = 54$, $x _{24} = 89, y _{24} = 32$
Требуется
найти следующие условные эмпирические характеристики: 1) ковариацию $X$ и $Y$ при условии, что одновременно $X \geqslant 50$
 и $Y \geqslant 50$; 2) коэффициент корреляции $X$ и $Y$ при том же условии.




1) Ковариация = $-345.5$
2) Коэффициент корреляции = $-2.9554$


\item

    
    	Распределение результатов экзамена в некоторой стране с $10$-балльной системой оценивания задано следующим образом:
    	$\left\{ 1 : 6, \  2 : 16, \  3 : 9, \  4 : 16, \  5 : 14, \  6 : 4, \  7 : 25, \  8 : 26, \  9 : 24, \  10 : 10\right\}$

	Работы будут перепроверять $10$ преподавателей, которые разделили все имеющиеся работы между собой случайным образом. Пусть $\overline{X}$ - средний балл (по перепроверки) работ, попавших к одному преподавателю.

	Требуется найти матожидание и стандартное отклонение среднего балла работ, попавших к одному преподавателю, до перепроверки.
    


    


    k = len(marks) // k

    ex = np.sum([marks[m] * m for m in marks]) / n

    varx = np.var([ m for m in marks for temp in range(marks[m])]) / k * (n - k) / (n - 1)

    sigmax = varx**(0.5)
    Ответы: $6.14667, 0.65542$.

    

\item

    
	Известно, что доля возвратов по кредитам в банке имеет распределение $F(x) = x^{\beta}, 0 \le x \le 1$. Наблюдения показали, что в среднем она составила $60.0$\%. Методом моментов оцените параметр $\beta$ и вероятность того, что она опуститься ниже $52.0$\%.
	


	

	$f(x) = F'(x) = \beta \cdot x^{\beta - 1}$

	$\mu_{1} = E(X) = \int_{-\inf}^{\inf}x \cdot f(x) = \int_{-\inf}^{\inf} \beta \cdot x^{\beta} = \beta \cdot \frac{x^{\beta + 1}}{\beta + 1}\bigg|_0^1 = \frac{\beta}{\beta + 1}$

	$\beta = (\beta + 1) \cdot 60.0$

	$\beta = \frac{60.0}{1 - 60.0}$

	$ P(x \le 52.0) = F(52.0) = 52.0^{1.5} $

    Ответ: $1.5, 0.37$
	


\end{enumerate}

\section{Билет 104}

\begin{enumerate}


\item

Дайте определение случайной величины, которая имеет гамма-распределение $\Gamma(\alpha,  \lambda)$, и выведите основные свойства гамма-расределения. Запишите формулы для математичсекого ожидания
$\mathbb{E}(X)$ и дисперсии $\mathbb{V}ar(X)$ гамма-распределения




Здесь написанно много всего интересного и полезного о гамма-распределении


\item



Случайные величины $X$ и $Y$ независимы и имеют равномерное
распределение на отрезках $[0;5]$ и $[0;10]$ соответственно. Для случайной величины $Z=\frac{Y}{X}$ найдите: 
1) функцию распределения $F_Z(x)$;
2) плотность распределения $f_Z(x)$ и постройте график плотности;
3) вероятность $\P(0,\!1\leqslant Z\leqslant 3,\!714)$.




%\folder 2_53d16.png
1) Функция распределения $F_Z(x)$ имеет вид:
$
F_Z(x)=\left\{
\begin{array}{l}
0, x\leqslant 0;\\
\frac{x}{4}, 0\leqslant x\leqslant 2\approx 2,\!0;\\
1 - \frac{1}{x}, x\geqslant2;
\end{array}.
\right.
$
2) Плотность распределения $f_Z(x)$ имеет вид:
$
f_Z(x)=\left\{
\begin{array}{l}
0, x<0;\\
\frac{1}{4}, 0\leqslant x\leqslant 2\approx 2,\!0;\\
\frac{1}{x^{2}}, x\geqslant2;
\end{array}.
\right.
$


\begin{figure}[H]
    \includegraphics[width=0.9\textwidth]{2_53d16}
\end{figure}


3) вероятность равна:
$
\P(0,\!1\leqslant Z\leqslant 3,\!714)=
0,\!70575.
$


\item

%\folder 1.pdf
(10) Известно, что доля возвратов по кредитам в банке имеет распределение $F(x) = x ^{\beta}, 0 \leqslant x \leqslant 1$.
Наблюдения показали, что в среднем она составляет $91,6667\%$. Методом моментов оцените параметр $\beta$ и
вероятность того, что она опуститься ниже $59\%$




Найдём плотность рапределения как интеграл от ФР, а дальше всё и вовсе простою Ответ: $30155888444737842659$


\item

    
    Создайте эмперические совокупности  $\mathtt{\text{log}}$ и $\mathtt{\text{cos}}$ вида $\mathtt{\text{log}}(1),\mathtt{\text{log}}(2), ..., \mathtt{\text{log}}(61) $ и $\mathtt{\text{cos}}(1),\mathtt{\text{cos}}(2), ..., \mathtt{\text{cos}}(61). $

    Найдите эмпирическое среднее и эмпирическое стандартное отклонение совокупности $\mathtt{\text{log}}$, её четвёртый эмпирический центральный момент и эмпирический эксцесс.

    Кроме того, найдите эмпирический коэффициент корреляции признаков $\mathtt{\text{log}}$ и $\mathtt{\text{cos}}$ на совокупности натуральных чисел от $1$ до $61$.
    


    
    Используя

	$E(X) = sum(X) / n$

	$Var(X) = E(X^2) - [E(X)]^2$

	$\mu_4(X) = E((X-E(X))^4)$

	$Ex = \frac{\mu_4(X)}{[\sigma(X)]^4} - 3$

	$r_{xy} = \frac{E(XY) - E(X) * E(Y)}{\sigma(X) * \sigma(Y)}$

    рассчитаем искомые значения.

    Ответы: $3.15966, 0.89438, 3.08587, 1.82265, -1.0 \cdot 10^{-5}$.

    

\item


(10) Эмпирическое распределение признаков $X$ и $Y$ на генеральной совокупности $\Omega$ задано таблицей частот  
 
\begin{tabular}{ | c | c | c | c | }
\hline
 & $Y = 2$ & $Y = 4$ & $Y = 5$  \\ \hline
$X = 200$ & $25$ & $26$ & $10$\\ \hline
$X = 300$ & $10$ & $10$ & $19$\\
\hline
\end{tabular}

Из $\Omega$ случайным образом без возвращения извлекаются $12$ элементов. 
Пусть $\bar X$ и $\bar Y$ – средние значения признаков на выбранных элементах. 
Требуется найти: 1) математическое ожидание $\mathbb{E}(\bar Y)$; 2) стандартное отклонение $\sigma(\bar X)$ ; 
3) ковариацию $Cov(\bar X, \bar Y)$




1) математическое ожидание $\mathbb{E}(\bar Y)$: $3.59$ 
2) стандартное отклонение $\sigma(\bar X)$: $228.8693$
3) ковариацию $Cov(\bar X, \bar Y)$: $1.3324$


\item

    
    	Юный аналитик Дарья использовала метод Монте-Карло для исследования Дискретного случайного вектора, описанного ниже.

        \begin{tabular}{|c|c|c|c|}
	\hline
	& X=$-6$ & X=$-5$ & X=$-4$ \\
	\hline
	Y = $5$ & $0.039$ & $0.207$  &  $0.054$ \\
	\hline
	Y = $6$ & $0.035$ & $0.255$ & $0.41$  \\
	\hline
\end{tabular}

    	Дарья получила, что E(Y|X + Y = 1) = $5.82286$.
    	Проверьте, можно ли доверять результату Дарьи аналитически. Сформулируйте определение метода Монте-Карло.
    


    
        $E(Y|X+Y=1) = \frac{\sum(P(X=1 - y_i, y=y_i) * y_i)}{\sum(P(X=1 - y_i, y=y_i)}$.

        Ответ: $5.82286$
    


\end{enumerate}

\section{Билет 105}

\begin{enumerate}


\item


Сформулируйте определение случайной выборки из конечной генеральной совокупности. Какие
виды выборок вам известны? Перечислите (с указанием формул) основные характеристики выборочной и генеральной совокупностей




Здесь очень много исчерпывающей информации о выборках из генеральной совокупности и про различные виды выборок


\item


(10) Сформулируйте критерий независимости $\chi ^ {2}$ – Пирсона. Приведите (с выводом и
необходимыми пояснениями в обозначениях) явный вид статистики критерия в случае, когда 
таблица сопряженности двух признаков $X$ и $Y$ имеет вид

\begin{tabular}[b]{ | c | c | c | }
\hline
$ $ & $Y = y _{1}$ & $Y = y _{2}$  \\ \hline
$X = x _{1}$ & $a$ & $b$ \\ \hline
$X = x _{2}$ & $c$ & $d$ \\
\hline
\end{tabular}




Здесь формулировки критерия независимости Пирсона и приводится пример


\item

    
	Случайная величина Y принимает только значения из множества $\{10, 7\}$, при этом $P(Y=10) = 0.24$.
	Распределение случайной величины X определено следующим образом:
	\begin{equation*}
		X | Y =
		\begin{cases}
			$4$ * y, с вероятностью $ 0.53$ \\
			$9$ * y, с вероятностью $ 1 - 0.53$
		\end{cases}
	\end{equation*}

	Юный аналитик Дарья нашла матожидание и дисперсию $X$.

	Помогите Дарье найти матожидание и дисперсию величины $X$
	


	

	Первым этапом надо найти характеристики случайной величины $Y$

	$E(Y) = 10 * 0.24 + 7 * (1 - 0.24)$

	$Var(Y) = E(Y^2) - [E(Y)]^2 = 10^2 * 0.24 + 7^2 * (1 - 0.24) - [E(Y)]^2$


	Перейдем к рассмотрению характеристик условной случайно величины X

	$E(X) = E(E(X|Y)) = E[E(4 * Y) * 0.53 + E(9 * Y) * (1 - 0.53)] = E(Y) * (4 * 0.53 + 9 * (1 - 0.53)) = 49.022$

	$E(Var(X|Y)) = E[b * Var(c3 * Y) + (1 - b) * Var(c4 * Y)] = Var(Y) * (c3^2 * b + c4^2 * (1- b)) $

	$Var(E(X|Y)) = E(X^2|Y) - [E(X)]^2 = [E(Y)]^2 * (b * c3^2 + (1-b)*c4^2) - E(X)]^2$

	$Var(X) = E(Var(X|Y)) + Var(E(X|Y)) = 447.56552$
	

\item


(10) В группе $\Omega$ учатся студенты:$\omega _{1}...\omega _{25}$ . Пусть $X$ и $Y$ – 100-балльные экзаменационные оценки по
математическому анализу и теории вероятностей. Оценки $\omega _{i}$ студента обозначаются: $x _{i} = X(\omega _{i})$ и $y _{i} = Y(\omega _{i})$, $i = 1...25$. Все оценки известны
$x _{0} = 55, y _{0} = 54$, $x _{1} = 64, y _{1} = 68$, $x _{2} = 34, y _{2} = 51$, $x _{3} = 48, y _{3} = 73$, $x _{4} = 81, y _{4} = 69$, $x _{5} = 62, y _{5} = 69$, $x _{6} = 76, y _{6} = 59$, $x _{7} = 84, y _{7} = 45$, $x _{8} = 97, y _{8} = 77$, $x _{9} = 76, y _{9} = 87$, $x _{10} = 43, y _{10} = 67$, $x _{11} = 33, y _{11} = 55$, $x _{12} = 71, y _{12} = 96$, $x _{13} = 62, y _{13} = 97$, $x _{14} = 84, y _{14} = 37$, $x _{15} = 41, y _{15} = 70$, $x _{16} = 92, y _{16} = 41$, $x _{17} = 60, y _{17} = 54$, $x _{18} = 71, y _{18} = 44$, $x _{19} = 39, y _{19} = 70$, $x _{20} = 98, y _{20} = 75$, $x _{21} = 99, y _{21} = 32$, $x _{22} = 58, y _{22} = 42$, $x _{23} = 61, y _{23} = 92$, $x _{24} = 58, y _{24} = 32$
Требуется
найти следующие условные эмпирические характеристики: 1) ковариацию $X$ и $Y$ при условии, что одновременно $X \geqslant 50$
 и $Y \geqslant 50$; 2) коэффициент корреляции $X$ и $Y$ при том же условии.




1) Ковариация = $276.75$
2) Коэффициент корреляции = $1.373$


\item

    
    	Распределение результатов экзамена в некоторой стране с $11$-балльной системой оценивания задано следующим образом:
    	$\left\{ 1 : 13, \  2 : 3, \  3 : 14, \  4 : 9, \  5 : 6, \  6 : 15, \  7 : 1, \  8 : 22, \  9 : 17, \  10 : 10, \  11 : 16\right\}$

	Работы будут перепроверять $6$ преподавателей, которые разделили все имеющиеся работы между собой случайным образом. Пусть $\overline{X}$ - средний балл (по перепроверки) работ, попавших к одному преподавателю.

	Требуется найти матожидание и стандартное отклонение среднего балла работ, попавших к одному преподавателю, до перепроверки.
    


    


    k = len(marks) // k

    ex = np.sum([marks[m] * m for m in marks]) / n

    varx = np.var([ m for m in marks for temp in range(marks[m])]) / k * (n - k) / (n - 1)

    sigmax = varx**(0.5)
    Ответы: $6.57937, 0.64259$.

    

\item

    
    	Юный аналитик Дарья использовала метод Монте-Карло для исследования Дискретного случайного вектора, описанного ниже.

        \begin{tabular}{|c|c|c|c|}
	\hline
	& X=$-3$ & X=$-2$ & X=$-1$ \\
	\hline
	Y = $2$ & $0.29$ & $0.298$  &  $0.234$ \\
	\hline
	Y = $3$ & $0.066$ & $0.03$ & $0.082$  \\
	\hline
\end{tabular}

    	Дарья получила, что E(Y|X + Y = 1) = $2.10982$.
    	Проверьте, можно ли доверять результату Дарьи аналитически. Сформулируйте определение метода Монте-Карло.
    


    
        $E(Y|X+Y=1) = \frac{\sum(P(X=1 - y_i, y=y_i) * y_i)}{\sum(P(X=1 - y_i, y=y_i)}$.

        Ответ: $2.10982$
    


\end{enumerate}

\section{Билет 106}

\begin{enumerate}


\item


Дайте определение случайной величины, которая имеет $\chi ^{2}$-распределение с n степенями свободы.
Запишите плотность $\chi ^{2}$- распределения. Выведите формулы для математического ожидания $\mathbb{E}(X)$ и дисперсии $\mathbb{V}ar(X)$ $\chi ^{2}$-распределение с n степенями свободы. Найдите а) $\mathbb{P}(\chi _{20}^{2} > 10.9)$, где $\chi _{20}^{2}$–случайная величина, которая имеет $\chi ^{2}$– распределение с 20 степенями свободы; б) найдите 93\%
(верхнюю) точку $\chi _{0.93}^{2} (5)$ хи-квадрат распределения с 5 степенями свободы




$\mathbb{P}(\chi _{20}^{2} > 10.9) =  0.948775$; $\chi _{0.93}^{2} (5) = 1.34721$.


\item



Случайные величины $X$ и $Y$ независимы и имеют равномерное
распределение на отрезках $[0;6]$ и $[0;1]$ соответственно. Для случайной величины $Z=\frac{Y}{X}$ найдите: 
1) функцию распределения $F_Z(x)$;
2) плотность распределения $f_Z(x)$ и постройте график плотности;
3) вероятность $\P(0,\!087\leqslant Z\leqslant 0,\!235)$.




%\folder 2_53d7.png
1) Функция распределения $F_Z(x)$ имеет вид:
$
F_Z(x)=\left\{
\begin{array}{l}
0, x\leqslant 0;\\
3 x, 0\leqslant x\leqslant \frac{1}{6}\approx 0,\!167;\\
1 - \frac{1}{12 x}, x\geqslant\frac{1}{6};
\end{array}.
\right.
$
2) Плотность распределения $f_Z(x)$ имеет вид:
$
f_Z(x)=\left\{
\begin{array}{l}
0, x<0;\\
3, 0\leqslant x\leqslant \frac{1}{6}\approx 0,\!167;\\
\frac{1}{12 x^{2}}, x\geqslant\frac{1}{6};
\end{array}.
\right.
$


\begin{figure}[H]
    \includegraphics[width=0.9\textwidth]{2_53d7}
\end{figure}


3) вероятность равна:
$
\P(0,\!087\leqslant Z\leqslant 0,\!235)=
0,\!38564.
$


\item


(10) Известно, что доля возвратов по кредитам в банке имеет распределение $F(x) = x ^{\beta}, 0 \leqslant x \leqslant 1$.
Наблюдения показали, что в среднем она составляет $87,5\%$. Методом моментов оцените параметр $\beta$ и
вероятность того, что она опуститься ниже $53\%$




Найдём плотность рапределения как интеграл от ФР, а дальше всё и вовсе простою Ответ: $1174711139837$


\item


(10) В группе $\Omega$ учатся студенты:$\omega _{1}...\omega _{25}$ . Пусть $X$ и $Y$ – 100-балльные экзаменационные оценки по
математическому анализу и теории вероятностей. Оценки $\omega _{i}$ студента обозначаются: $x _{i} = X(\omega _{i})$ и $y _{i} = Y(\omega _{i})$, $i = 1...25$. Все оценки известны
$x _{0} = 32, y _{0} = 89$, $x _{1} = 61, y _{1} = 91$, $x _{2} = 64, y _{2} = 88$, $x _{3} = 97, y _{3} = 55$, $x _{4} = 66, y _{4} = 84$, $x _{5} = 78, y _{5} = 56$, $x _{6} = 62, y _{6} = 60$, $x _{7} = 73, y _{7} = 42$, $x _{8} = 40, y _{8} = 59$, $x _{9} = 86, y _{9} = 80$, $x _{10} = 76, y _{10} = 33$, $x _{11} = 56, y _{11} = 64$, $x _{12} = 87, y _{12} = 86$, $x _{13} = 70, y _{13} = 38$, $x _{14} = 87, y _{14} = 76$, $x _{15} = 72, y _{15} = 63$, $x _{16} = 79, y _{16} = 41$, $x _{17} = 33, y _{17} = 74$, $x _{18} = 67, y _{18} = 71$, $x _{19} = 65, y _{19} = 34$, $x _{20} = 57, y _{20} = 56$, $x _{21} = 63, y _{21} = 87$, $x _{22} = 68, y _{22} = 95$, $x _{23} = 46, y _{23} = 94$, $x _{24} = 50, y _{24} = 73$
Требуется
найти следующие условные эмпирические характеристики: 1) ковариацию $X$ и $Y$ при условии, что одновременно $X \geqslant 50$
 и $Y \geqslant 50$; 2) коэффициент корреляции $X$ и $Y$ при том же условии.




1) Ковариация = $-262.8$
2) Коэффициент корреляции = $-1.5753$


\item


(10) Эмпирическое распределение признаков $X$ и $Y$ на генеральной совокупности $\Omega$ задано таблицей частот  
 
\begin{tabular}{ | c | c | c | c | }
\hline
 & $Y = 2$ & $Y = 4$ & $Y = 5$  \\ \hline
$X = 200$ & $1$ & $18$ & $12$\\ \hline
$X = 300$ & $31$ & $26$ & $12$\\
\hline
\end{tabular}

Из $\Omega$ случайным образом без возвращения извлекаются $12$ элементов. 
Пусть $\bar X$ и $\bar Y$ – средние значения признаков на выбранных элементах. 
Требуется найти: 1) математическое ожидание $\mathbb{E}(\bar Y)$; 2) стандартное отклонение $\sigma(\bar X)$ ; 
3) ковариацию $Cov(\bar X, \bar Y)$




1) математическое ожидание $\mathbb{E}(\bar Y)$: $3.6$ 
2) стандартное отклонение $\sigma(\bar X)$: $256.084$
3) ковариацию $Cov(\bar X, \bar Y)$: $-1.9911$


\item

    
    	Юный аналитик Дарья использовала метод Монте-Карло для исследования Дискретного случайного вектора, описанного ниже.

        \begin{tabular}{|c|c|c|c|}
	\hline
	& X=$-4$ & X=$-3$ & X=$-2$ \\
	\hline
	Y = $3$ & $0.07$ & $0.084$  &  $0.205$ \\
	\hline
	Y = $4$ & $0.011$ & $0.201$ & $0.429$  \\
	\hline
\end{tabular}

    	Дарья получила, что E(Y|X + Y = 1) = $3.49618$.
    	Проверьте, можно ли доверять результату Дарьи аналитически. Сформулируйте определение метода Монте-Карло.
    


    
        $E(Y|X+Y=1) = \frac{\sum(P(X=1 - y_i, y=y_i) * y_i)}{\sum(P(X=1 - y_i, y=y_i)}$.

        Ответ: $3.49618$
    


\end{enumerate}

\section{Билет 107}

\begin{enumerate}


\item


Сформулируйте определение случайной выборки из конечной генеральной совокупности. Какие
виды выборок вам известны? Перечислите (с указанием формул) основные характеристики выборочной и генеральной совокупностей




Здесь очень много исчерпывающей информации о выборках из генеральной совокупности и про различные виды выборок


\item



Случайные величины $X$ и $Y$ независимы и имеют равномерное
распределение на отрезках $[0;3]$ и $[0;8]$ соответственно. Для случайной величины $Z=\frac{Y}{X}$ найдите: 
1) функцию распределения $F_Z(x)$;
2) плотность распределения $f_Z(x)$ и постройте график плотности;
3) вероятность $\P(2,\!475\leqslant Z\leqslant 4,\!811)$.




%\folder 2_53d8.png
1) Функция распределения $F_Z(x)$ имеет вид:
$
F_Z(x)=\left\{
\begin{array}{l}
0, x\leqslant 0;\\
\frac{3 x}{16}, 0\leqslant x\leqslant \frac{8}{3}\approx 2,\!667;\\
1 - \frac{4}{3 x}, x\geqslant\frac{8}{3};
\end{array}.
\right.
$
2) Плотность распределения $f_Z(x)$ имеет вид:
$
f_Z(x)=\left\{
\begin{array}{l}
0, x<0;\\
\frac{3}{16}, 0\leqslant x\leqslant \frac{8}{3}\approx 2,\!667;\\
\frac{4}{3 x^{2}}, x\geqslant\frac{8}{3};
\end{array}.
\right.
$


\begin{figure}[H]
    \includegraphics[width=0.9\textwidth]{2_53d8}
\end{figure}


3) вероятность равна:
$
\P(2,\!475\leqslant Z\leqslant 4,\!811)=
0,\!25884.
$


\item


%\folder 2.pdf
(10) Известно, что доля возвратов по кредитам в банке имеет распределение $F(x) = x ^{\beta}, 0 \leqslant x \leqslant 1$.
Наблюдения показали, что в среднем она составляет $75,0\%$. Методом моментов оцените параметр $\beta$ и
вероятность того, что она опуститься ниже $20\%$




Найдём плотность рапределения как интеграл от ФР, а дальше всё и вовсе простою Ответ: $8000$


\item

    
    Создайте эмперические совокупности  $\mathtt{\text{exp}}$ и $\mathtt{\text{cos}}$ вида $\mathtt{\text{exp}}(1),\mathtt{\text{exp}}(2), ..., \mathtt{\text{exp}}(57) $ и $\mathtt{\text{cos}}(1),\mathtt{\text{cos}}(2), ..., \mathtt{\text{cos}}(57). $

    Найдите эмпирическое среднее и эмпирическое стандартное отклонение совокупности $\mathtt{\text{exp}}$, её четвёртый эмпирический центральный момент и эмпирический эксцесс.

    Кроме того, найдите эмпирический коэффициент корреляции признаков $\mathtt{\text{exp}}$ и $\mathtt{\text{cos}}$ на совокупности натуральных чисел от $1$ до $57$.
    


    
    Используя

	$E(X) = sum(X) / n$

	$Var(X) = E(X^2) - [E(X)]^2$

	$\mu_4(X) = E((X-E(X))^4)$

	$Ex = \frac{\mu_4(X)}{[\sigma(X)]^4} - 3$

	$r_{xy} = \frac{E(XY) - E(X) * E(Y)}{\sigma(X) * \sigma(Y)}$

    рассчитаем искомые значения.

    Ответы: $1.57801343872465 \cdot 10^{23}, 7.94364472492678 \cdot 10^{23}, 1.66305653632206 \cdot 10^{97}, 38.76647, 0.00352$.

    

\item


(10) Эмпирическое распределение признаков $X$ и $Y$ на генеральной совокупности $\Omega$ задано таблицей частот  
 
\begin{tabular}{ | c | c | c | c | }
\hline
 & $Y = 2$ & $Y = 4$ & $Y = 5$  \\ \hline
$X = 200$ & $1$ & $18$ & $12$\\ \hline
$X = 300$ & $31$ & $26$ & $12$\\
\hline
\end{tabular}

Из $\Omega$ случайным образом без возвращения извлекаются $12$ элементов. 
Пусть $\bar X$ и $\bar Y$ – средние значения признаков на выбранных элементах. 
Требуется найти: 1) математическое ожидание $\mathbb{E}(\bar Y)$; 2) стандартное отклонение $\sigma(\bar X)$ ; 
3) ковариацию $Cov(\bar X, \bar Y)$




1) математическое ожидание $\mathbb{E}(\bar Y)$: $3.6$ 
2) стандартное отклонение $\sigma(\bar X)$: $256.084$
3) ковариацию $Cov(\bar X, \bar Y)$: $-1.9911$


\item

    
	Известно, что доля возвратов по кредитам в банке имеет распределение $F(x) = x^{\beta}, 0 \le x \le 1$. Наблюдения показали, что в среднем она составила $55.0$\%. Методом моментов оцените параметр $\beta$ и вероятность того, что она опуститься ниже $50.0$\%.
	


	

	$f(x) = F'(x) = \beta \cdot x^{\beta - 1}$

	$\mu_{1} = E(X) = \int_{-\inf}^{\inf}x \cdot f(x) = \int_{-\inf}^{\inf} \beta \cdot x^{\beta} = \beta \cdot \frac{x^{\beta + 1}}{\beta + 1}\bigg|_0^1 = \frac{\beta}{\beta + 1}$

	$\beta = (\beta + 1) \cdot 55.0$

	$\beta = \frac{55.0}{1 - 55.0}$

	$ P(x \le 50.0) = F(50.0) = 50.0^{1.22} $

    Ответ: $1.22, 0.43$
	


\end{enumerate}

\section{Билет 108}

\begin{enumerate}


\item


Сформулируйте определение случайной выборки из конечной генеральной совокупности. Какие
виды выборок вам известны? Перечислите (с указанием формул) основные характеристики выборочной и генеральной совокупностей




Здесь очень много исчерпывающей информации о выборках из генеральной совокупности и про различные виды выборок


\item



Случайные величины $X$ и $Y$ независимы и имеют равномерное
распределение на отрезках $[0;7]$ и $[0;3]$ соответственно. Для случайной величины $Z=\frac{Y}{X}$ найдите: 
1) функцию распределения $F_Z(x)$;
2) плотность распределения $f_Z(x)$ и постройте график плотности;
3) вероятность $\P(0,\!006\leqslant Z\leqslant 0,\!519)$.




%\folder 2_53d18.png
1) Функция распределения $F_Z(x)$ имеет вид:
$
F_Z(x)=\left\{
\begin{array}{l}
0, x\leqslant 0;\\
\frac{7 x}{6}, 0\leqslant x\leqslant \frac{3}{7}\approx 0,\!429;\\
1 - \frac{3}{14 x}, x\geqslant\frac{3}{7};
\end{array}.
\right.
$
2) Плотность распределения $f_Z(x)$ имеет вид:
$
f_Z(x)=\left\{
\begin{array}{l}
0, x<0;\\
\frac{7}{6}, 0\leqslant x\leqslant \frac{3}{7}\approx 0,\!429;\\
\frac{3}{14 x^{2}}, x\geqslant\frac{3}{7};
\end{array}.
\right.
$


\begin{figure}[H]
    \includegraphics[width=0.9\textwidth]{2_53d18}
\end{figure}


3) вероятность равна:
$
\P(0,\!006\leqslant Z\leqslant 0,\!519)=
0,\!57962.
$


\item

    
	Случайная величина Y принимает только значения из множества $\{10, 7\}$, при этом $P(Y=10) = 0.24$.
	Распределение случайной величины X определено следующим образом:
	\begin{equation*}
		X | Y =
		\begin{cases}
			$4$ * y, с вероятностью $ 0.53$ \\
			$9$ * y, с вероятностью $ 1 - 0.53$
		\end{cases}
	\end{equation*}

	Юный аналитик Дарья нашла матожидание и дисперсию $X$.

	Помогите Дарье найти матожидание и дисперсию величины $X$
	


	

	Первым этапом надо найти характеристики случайной величины $Y$

	$E(Y) = 10 * 0.24 + 7 * (1 - 0.24)$

	$Var(Y) = E(Y^2) - [E(Y)]^2 = 10^2 * 0.24 + 7^2 * (1 - 0.24) - [E(Y)]^2$


	Перейдем к рассмотрению характеристик условной случайно величины X

	$E(X) = E(E(X|Y)) = E[E(4 * Y) * 0.53 + E(9 * Y) * (1 - 0.53)] = E(Y) * (4 * 0.53 + 9 * (1 - 0.53)) = 49.022$

	$E(Var(X|Y)) = E[b * Var(c3 * Y) + (1 - b) * Var(c4 * Y)] = Var(Y) * (c3^2 * b + c4^2 * (1- b)) $

	$Var(E(X|Y)) = E(X^2|Y) - [E(X)]^2 = [E(Y)]^2 * (b * c3^2 + (1-b)*c4^2) - E(X)]^2$

	$Var(X) = E(Var(X|Y)) + Var(E(X|Y)) = 447.56552$
	

\item

    
    Создайте эмперические совокупности  $\mathtt{\text{cos}}$ и $\mathtt{\text{log}}$ вида $\mathtt{\text{cos}}(1),\mathtt{\text{cos}}(2), ..., \mathtt{\text{cos}}(98) $ и $\mathtt{\text{log}}(1),\mathtt{\text{log}}(2), ..., \mathtt{\text{log}}(98). $

    Найдите эмпирическое среднее и эмпирическое стандартное отклонение совокупности $\mathtt{\text{cos}}$, её четвёртый эмпирический центральный момент и эмпирический эксцесс.

    Кроме того, найдите эмпирический коэффициент корреляции признаков $\mathtt{\text{cos}}$ и $\mathtt{\text{log}}$ на совокупности натуральных чисел от $1$ до $98$.
    


    
    Используя

	$E(X) = sum(X) / n$

	$Var(X) = E(X^2) - [E(X)]^2$

	$\mu_4(X) = E((X-E(X))^4)$

	$Ex = \frac{\mu_4(X)}{[\sigma(X)]^4} - 3$

	$r_{xy} = \frac{E(XY) - E(X) * E(Y)}{\sigma(X) * \sigma(Y)}$

    рассчитаем искомые значения.

    Ответы: $-0.01464, 0.70686, 0.37349, -1.50394, 1.0 \cdot 10^{-5}$.

    

\item


(10) Эмпирическое распределение признаков $X$ и $Y$ на генеральной совокупности $\Omega$ задано таблицей частот  
 
\begin{tabular}{ | c | c | c | c | }
\hline
 & $Y = 2$ & $Y = 4$ & $Y = 5$  \\ \hline
$X = 200$ & $28$ & $23$ & $3$\\ \hline
$X = 300$ & $2$ & $12$ & $32$\\
\hline
\end{tabular}

Из $\Omega$ случайным образом без возвращения извлекаются $5$ элементов. 
Пусть $\bar X$ и $\bar Y$ – средние значения признаков на выбранных элементах. 
Требуется найти: 1) математическое ожидание $\mathbb{E}(\bar Y)$; 2) стандартное отклонение $\sigma(\bar X)$ ; 
3) ковариацию $Cov(\bar X, \bar Y)$




1) математическое ожидание $\mathbb{E}(\bar Y)$: $3.75$ 
2) стандартное отклонение $\sigma(\bar X)$: $244.6913$
3) ковариацию $Cov(\bar X, \bar Y)$: $3.7904$


\item


(10) Пусть $X _{1}$, $X _{2}$, $X _{3}$, $X _{4}$ выборка из $N(\theta, \sigma ^{2})$. Рассмотрим две оценки параметра $\theta$:
\[\hat \theta _{1} = \frac{X _{1} + 4X _{2} + X _{3} + 4X _{4}}{10}, \hat \theta _{1} = \frac{2X _{1} + 3X _{2} + 3X _{3} + 2X _{4}}{10}\]
a) Покажите, что обе оценки несмещенные.
б) Какая из оценок оптимальная?




Обе они несмещенные, потому что в числителе выходит в сумме 10.
Какая-то точно должна быть, а может и нет....



\end{enumerate}

\section{Билет 109}

\begin{enumerate}


\item

Дайте определение случайной величины, которая имеет гамма-распределение $\Gamma(\alpha,  \lambda)$, и выведите основные свойства гамма-расределения. Запишите формулы для математичсекого ожидания
$\mathbb{E}(X)$ и дисперсии $\mathbb{V}ar(X)$ гамма-распределения




Здесь написанно много всего интересного и полезного о гамма-распределении


\item



Случайные величины $X$ и $Y$ независимы и имеют равномерное
распределение на отрезках $[0;4]$ и $[0;7]$ соответственно. Для случайной величины $Z=\frac{Y}{X}$ найдите: 
1) функцию распределения $F_Z(x)$;
2) плотность распределения $f_Z(x)$ и постройте график плотности;
3) вероятность $\P(0,\!035\leqslant Z\leqslant 2,\!775)$.




%\folder 2_53d12.png
1) Функция распределения $F_Z(x)$ имеет вид:
$
F_Z(x)=\left\{
\begin{array}{l}
0, x\leqslant 0;\\
\frac{2 x}{7}, 0\leqslant x\leqslant \frac{7}{4}\approx 1,\!75;\\
1 - \frac{7}{8 x}, x\geqslant\frac{7}{4};
\end{array}.
\right.
$
2) Плотность распределения $f_Z(x)$ имеет вид:
$
f_Z(x)=\left\{
\begin{array}{l}
0, x<0;\\
\frac{2}{7}, 0\leqslant x\leqslant \frac{7}{4}\approx 1,\!75;\\
\frac{7}{8 x^{2}}, x\geqslant\frac{7}{4};
\end{array}.
\right.
$


\begin{figure}[H]
    \includegraphics[width=0.9\textwidth]{2_53d12}
\end{figure}


3) вероятность равна:
$
\P(0,\!035\leqslant Z\leqslant 2,\!775)=
0,\!67474.
$


\item

    
	Случайная величина Y принимает только значения из множества $\{1, 10\}$, при этом $P(Y=1) = 0.7$.
	Распределение случайной величины X определено следующим образом:
	\begin{equation*}
		X | Y =
		\begin{cases}
			$5$ * y, с вероятностью $ 0.11$ \\
			$8$ * y, с вероятностью $ 1 - 0.11$
		\end{cases}
	\end{equation*}

	Юный аналитик Дарья нашла матожидание и дисперсию $X$.

	Помогите Дарье найти матожидание и дисперсию величины $X$
	


	

	Первым этапом надо найти характеристики случайной величины $Y$

	$E(Y) = 1 * 0.7 + 10 * (1 - 0.7)$

	$Var(Y) = E(Y^2) - [E(Y)]^2 = 1^2 * 0.7 + 10^2 * (1 - 0.7) - [E(Y)]^2$


	Перейдем к рассмотрению характеристик условной случайно величины X

	$E(X) = E(E(X|Y)) = E[E(5 * Y) * 0.11 + E(8 * Y) * (1 - 0.11)] = E(Y) * (5 * 0.11 + 8 * (1 - 0.11)) = 28.379$

	$E(Var(X|Y)) = E[b * Var(c3 * Y) + (1 - b) * Var(c4 * Y)] = Var(Y) * (c3^2 * b + c4^2 * (1- b)) $

	$Var(E(X|Y)) = E(X^2|Y) - [E(X)]^2 = [E(Y)]^2 * (b * c3^2 + (1-b)*c4^2) - E(X)]^2$

	$Var(X) = E(Var(X|Y)) + Var(E(X|Y)) = 1027.72936$
	

\item


(10) В группе $\Omega$ учатся студенты:$\omega _{1}...\omega _{25}$ . Пусть $X$ и $Y$ – 100-балльные экзаменационные оценки по
математическому анализу и теории вероятностей. Оценки $\omega _{i}$ студента обозначаются: $x _{i} = X(\omega _{i})$ и $y _{i} = Y(\omega _{i})$, $i = 1...25$. Все оценки известны
$x _{0} = 40, y _{0} = 84$, $x _{1} = 83, y _{1} = 71$, $x _{2} = 85, y _{2} = 64$, $x _{3} = 77, y _{3} = 32$, $x _{4} = 86, y _{4} = 59$, $x _{5} = 99, y _{5} = 77$, $x _{6} = 91, y _{6} = 74$, $x _{7} = 46, y _{7} = 48$, $x _{8} = 73, y _{8} = 42$, $x _{9} = 82, y _{9} = 89$, $x _{10} = 40, y _{10} = 43$, $x _{11} = 60, y _{11} = 31$, $x _{12} = 81, y _{12} = 57$, $x _{13} = 88, y _{13} = 50$, $x _{14} = 34, y _{14} = 31$, $x _{15} = 45, y _{15} = 63$, $x _{16} = 38, y _{16} = 45$, $x _{17} = 34, y _{17} = 92$, $x _{18} = 92, y _{18} = 83$, $x _{19} = 88, y _{19} = 56$, $x _{20} = 60, y _{20} = 36$, $x _{21} = 85, y _{21} = 59$, $x _{22} = 60, y _{22} = 87$, $x _{23} = 30, y _{23} = 53$, $x _{24} = 56, y _{24} = 73$
Требуется
найти следующие условные эмпирические характеристики: 1) ковариацию $X$ и $Y$ при условии, что одновременно $X \geqslant 50$
 и $Y \geqslant 50$; 2) коэффициент корреляции $X$ и $Y$ при том же условии.




1) Ковариация = $-335.0$
2) Коэффициент корреляции = $-2.4919$


\item


(10) Эмпирическое распределение признаков $X$ и $Y$ на генеральной совокупности $\Omega$ задано таблицей частот  
 
\begin{tabular}{ | c | c | c | c | }
\hline
 & $Y = 2$ & $Y = 4$ & $Y = 5$  \\ \hline
$X = 200$ & $24$ & $17$ & $3$\\ \hline
$X = 300$ & $13$ & $24$ & $19$\\
\hline
\end{tabular}

Из $\Omega$ случайным образом без возвращения извлекаются $9$ элементов. 
Пусть $\bar X$ и $\bar Y$ – средние значения признаков на выбранных элементах. 
Требуется найти: 1) математическое ожидание $\mathbb{E}(\bar Y)$; 2) стандартное отклонение $\sigma(\bar X)$ ; 
3) ковариацию $Cov(\bar X, \bar Y)$




1) математическое ожидание $\mathbb{E}(\bar Y)$: $3.48$ 
2) стандартное отклонение $\sigma(\bar X)$: $248.8024$
3) ковариацию $Cov(\bar X, \bar Y)$: $2.0333$


\item

    
	Известно, что доля возвратов по кредитам в банке имеет распределение $F(x) = x^{\beta}, 0 \le x \le 1$. Наблюдения показали, что в среднем она составила $67.0$\%. Методом моментов оцените параметр $\beta$ и вероятность того, что она опуститься ниже $52.0$\%.
	


	

	$f(x) = F'(x) = \beta \cdot x^{\beta - 1}$

	$\mu_{1} = E(X) = \int_{-\inf}^{\inf}x \cdot f(x) = \int_{-\inf}^{\inf} \beta \cdot x^{\beta} = \beta \cdot \frac{x^{\beta + 1}}{\beta + 1}\bigg|_0^1 = \frac{\beta}{\beta + 1}$

	$\beta = (\beta + 1) \cdot 67.0$

	$\beta = \frac{67.0}{1 - 67.0}$

	$ P(x \le 52.0) = F(52.0) = 52.0^{2.03} $

    Ответ: $2.03, 0.27$
	


\end{enumerate}

\section{Билет 110}

\begin{enumerate}


\item


Дайте определение случайной величины, которая имеет $\chi ^{2}$-распределение с n степенями свободы.
Запишите плотность $\chi ^{2}$- распределения. Выведите формулы для математического ожидания $\mathbb{E}(X)$ и дисперсии $\mathbb{V}ar(X)$ $\chi ^{2}$-распределение с n степенями свободы. Найдите а) $\mathbb{P}(\chi _{20}^{2} > 10.9)$, где $\chi _{20}^{2}$–случайная величина, которая имеет $\chi ^{2}$– распределение с 20 степенями свободы; б) найдите 93\%
(верхнюю) точку $\chi _{0.93}^{2} (5)$ хи-квадрат распределения с 5 степенями свободы




$\mathbb{P}(\chi _{20}^{2} > 10.9) =  0.948775$; $\chi _{0.93}^{2} (5) = 1.34721$.


\item

Случайные величины $X$ и $Y$ независимы и имеют равномерное
распределение на отрезках $[0;10]$ и $[0;3]$ соответственно. Для случайной величины $Z=\frac{Y}{X}$ найдите: 
1) функцию распределения $F_Z(x)$;
2) плотность распределения $f_Z(x)$ и постройте график плотности;
3) вероятность $\P(0,\!057\leqslant Z\leqslant 0,\!556)$.




%\folder 2_53d1.png
1) Функция распределения $F_Z(x)$ имеет вид:
$
F_Z(x)=\left\{
\begin{array}{l}
0, x\leqslant 0;\\
\frac{5 x}{3}, 0\leqslant x\leqslant \frac{3}{10}\approx 0,\!3;\\
1 - \frac{3}{20 x}, x\geqslant\frac{3}{10};
\end{array}.
\right.
$
2) Плотность распределения $f_Z(x)$ имеет вид:
$
f_Z(x)=\left\{
\begin{array}{l}
0, x<0;\\
\frac{5}{3}, 0\leqslant x\leqslant \frac{3}{10}\approx 0,\!3;\\
\frac{3}{20 x^{2}}, x\geqslant\frac{3}{10};
\end{array}.
\right.
$


\begin{figure}[H]
    \includegraphics[width=0.9\textwidth]{2_53d1}
\end{figure}


3) вероятность равна:
$
\P(0,\!057\leqslant Z\leqslant 0,\!556)=
0,\!63552.
$


\item


(10) Известно, что доля возвратов по кредитам в банке имеет распределение $F(x) = x ^{\beta}, 0 \leqslant x \leqslant 1$.
Наблюдения показали, что в среднем она составляет $75,0\%$. Методом моментов оцените параметр $\beta$ и
вероятность того, что она опуститься ниже $52\%$




Найдём плотность рапределения как интеграл от ФР, а дальше всё и вовсе простою Ответ: $140608$


\item


(10) В группе $\Omega$ учатся студенты:$\omega _{1}...\omega _{25}$ . Пусть $X$ и $Y$ – 100-балльные экзаменационные оценки по
математическому анализу и теории вероятностей. Оценки $\omega _{i}$ студента обозначаются: $x _{i} = X(\omega _{i})$ и $y _{i} = Y(\omega _{i})$, $i = 1...25$. Все оценки известны
$x _{0} = 32, y _{0} = 89$, $x _{1} = 61, y _{1} = 91$, $x _{2} = 64, y _{2} = 88$, $x _{3} = 97, y _{3} = 55$, $x _{4} = 66, y _{4} = 84$, $x _{5} = 78, y _{5} = 56$, $x _{6} = 62, y _{6} = 60$, $x _{7} = 73, y _{7} = 42$, $x _{8} = 40, y _{8} = 59$, $x _{9} = 86, y _{9} = 80$, $x _{10} = 76, y _{10} = 33$, $x _{11} = 56, y _{11} = 64$, $x _{12} = 87, y _{12} = 86$, $x _{13} = 70, y _{13} = 38$, $x _{14} = 87, y _{14} = 76$, $x _{15} = 72, y _{15} = 63$, $x _{16} = 79, y _{16} = 41$, $x _{17} = 33, y _{17} = 74$, $x _{18} = 67, y _{18} = 71$, $x _{19} = 65, y _{19} = 34$, $x _{20} = 57, y _{20} = 56$, $x _{21} = 63, y _{21} = 87$, $x _{22} = 68, y _{22} = 95$, $x _{23} = 46, y _{23} = 94$, $x _{24} = 50, y _{24} = 73$
Требуется
найти следующие условные эмпирические характеристики: 1) ковариацию $X$ и $Y$ при условии, что одновременно $X \geqslant 50$
 и $Y \geqslant 50$; 2) коэффициент корреляции $X$ и $Y$ при том же условии.




1) Ковариация = $-262.8$
2) Коэффициент корреляции = $-1.5753$


\item


(10) Эмпирическое распределение признаков $X$ и $Y$ на генеральной совокупности $\Omega$ задано таблицей частот  
 
\begin{tabular}{ | c | c | c | c | }
\hline
 & $Y = 2$ & $Y = 4$ & $Y = 5$  \\ \hline
$X = 200$ & $16$ & $19$ & $5$\\ \hline
$X = 300$ & $25$ & $10$ & $25$\\
\hline
\end{tabular}

Из $\Omega$ случайным образом без возвращения извлекаются $6$ элементов. 
Пусть $\bar X$ и $\bar Y$ – средние значения признаков на выбранных элементах. 
Требуется найти: 1) математическое ожидание $\mathbb{E}(\bar Y)$; 2) стандартное отклонение $\sigma(\bar X)$ ; 
3) ковариацию $Cov(\bar X, \bar Y)$




1) математическое ожидание $\mathbb{E}(\bar Y)$: $3.48$ 
2) стандартное отклонение $\sigma(\bar X)$: $256.5595$
3) ковариацию $Cov(\bar X, \bar Y)$: $0.5887$


\item

    
	Известно, что доля возвратов по кредитам в банке имеет распределение $F(x) = x^{\beta}, 0 \le x \le 1$. Наблюдения показали, что в среднем она составила $71.0$\%. Методом моментов оцените параметр $\beta$ и вероятность того, что она опуститься ниже $62.0$\%.
	


	

	$f(x) = F'(x) = \beta \cdot x^{\beta - 1}$

	$\mu_{1} = E(X) = \int_{-\inf}^{\inf}x \cdot f(x) = \int_{-\inf}^{\inf} \beta \cdot x^{\beta} = \beta \cdot \frac{x^{\beta + 1}}{\beta + 1}\bigg|_0^1 = \frac{\beta}{\beta + 1}$

	$\beta = (\beta + 1) \cdot 71.0$

	$\beta = \frac{71.0}{1 - 71.0}$

	$ P(x \le 62.0) = F(62.0) = 62.0^{2.45} $

    Ответ: $2.45, 0.31$
	


\end{enumerate}

\section{Билет 111}

\begin{enumerate}


\item

Дайте определение случайной величины, которая имеет гамма-распределение $\Gamma(\alpha,  \lambda)$, и выведите основные свойства гамма-расределения. Запишите формулы для математичсекого ожидания
$\mathbb{E}(X)$ и дисперсии $\mathbb{V}ar(X)$ гамма-распределения




Здесь написанно много всего интересного и полезного о гамма-распределении


\item



Случайные величины $X$ и $Y$ независимы и имеют равномерное
распределение на отрезках $[0;1]$ и $[0;10]$ соответственно. Для случайной величины $Z=\frac{Y}{X}$ найдите: 
1) функцию распределения $F_Z(x)$;
2) плотность распределения $f_Z(x)$ и постройте график плотности;
3) вероятность $\P(2,\!96\leqslant Z\leqslant 17,\!91)$.




%\folder 2_53d13.png
1) Функция распределения $F_Z(x)$ имеет вид:
$
F_Z(x)=\left\{
\begin{array}{l}
0, x\leqslant 0;\\
\frac{x}{20}, 0\leqslant x\leqslant 10\approx 10,\!0;\\
1 - \frac{5}{x}, x\geqslant10;
\end{array}.
\right.
$
2) Плотность распределения $f_Z(x)$ имеет вид:
$
f_Z(x)=\left\{
\begin{array}{l}
0, x<0;\\
\frac{1}{20}, 0\leqslant x\leqslant 10\approx 10,\!0;\\
\frac{5}{x^{2}}, x\geqslant10;
\end{array}.
\right.
$


\begin{figure}[H]
    \includegraphics[width=0.9\textwidth]{2_53d13}
\end{figure}


3) вероятность равна:
$
\P(2,\!96\leqslant Z\leqslant 17,\!91)=
0,\!57283.
$


\item


%\folder 2.pdf
(10) Известно, что доля возвратов по кредитам в банке имеет распределение $F(x) = x ^{\beta}, 0 \leqslant x \leqslant 1$.
Наблюдения показали, что в среднем она составляет $75,0\%$. Методом моментов оцените параметр $\beta$ и
вероятность того, что она опуститься ниже $20\%$




Найдём плотность рапределения как интеграл от ФР, а дальше всё и вовсе простою Ответ: $8000$


\item


(10) В группе $\Omega$ учатся студенты:$\omega _{1}...\omega _{25}$ . Пусть $X$ и $Y$ – 100-балльные экзаменационные оценки по
математическому анализу и теории вероятностей. Оценки $\omega _{i}$ студента обозначаются: $x _{i} = X(\omega _{i})$ и $y _{i} = Y(\omega _{i})$, $i = 1...25$. Все оценки известны
$x _{0} = 64, y _{0} = 84$, $x _{1} = 82, y _{1} = 42$, $x _{2} = 51, y _{2} = 99$, $x _{3} = 68, y _{3} = 57$, $x _{4} = 90, y _{4} = 71$, $x _{5} = 89, y _{5} = 55$, $x _{6} = 55, y _{6} = 55$, $x _{7} = 90, y _{7} = 58$, $x _{8} = 61, y _{8} = 78$, $x _{9} = 38, y _{9} = 84$, $x _{10} = 56, y _{10} = 95$, $x _{11} = 86, y _{11} = 69$, $x _{12} = 71, y _{12} = 72$, $x _{13} = 35, y _{13} = 99$, $x _{14} = 82, y _{14} = 67$, $x _{15} = 79, y _{15} = 59$, $x _{16} = 83, y _{16} = 88$, $x _{17} = 45, y _{17} = 75$, $x _{18} = 70, y _{18} = 79$, $x _{19} = 89, y _{19} = 80$, $x _{20} = 33, y _{20} = 30$, $x _{21} = 63, y _{21} = 73$, $x _{22} = 55, y _{22} = 53$, $x _{23} = 31, y _{23} = 78$, $x _{24} = 50, y _{24} = 90$
Требуется
найти следующие условные эмпирические характеристики: 1) ковариацию $X$ и $Y$ при условии, что одновременно $X \geqslant 50$
 и $Y \geqslant 50$; 2) коэффициент корреляции $X$ и $Y$ при том же условии.




1) Ковариация = $-876.6667$
2) Коэффициент корреляции = $-4.7659$


\item

    
    	Распределение результатов экзамена в некоторой стране с $11$-балльной системой оценивания задано следующим образом:
    	$\left\{ 1 : 13, \  2 : 3, \  3 : 14, \  4 : 9, \  5 : 6, \  6 : 15, \  7 : 1, \  8 : 22, \  9 : 17, \  10 : 10, \  11 : 16\right\}$

	Работы будут перепроверять $6$ преподавателей, которые разделили все имеющиеся работы между собой случайным образом. Пусть $\overline{X}$ - средний балл (по перепроверки) работ, попавших к одному преподавателю.

	Требуется найти матожидание и стандартное отклонение среднего балла работ, попавших к одному преподавателю, до перепроверки.
    


    


    k = len(marks) // k

    ex = np.sum([marks[m] * m for m in marks]) / n

    varx = np.var([ m for m in marks for temp in range(marks[m])]) / k * (n - k) / (n - 1)

    sigmax = varx**(0.5)
    Ответы: $6.57937, 0.64259$.

    

\item


(10) Пусть $X _{1}$, $X _{2}$, $X _{3}$, $X _{4}$ выборка из $N(\theta, \sigma ^{2})$. Рассмотрим две оценки параметра $\theta$:
\[\hat \theta _{1} = \frac{3X _{1} + X _{2} + 4X _{3} + 2X _{4}}{10}, \hat \theta _{1} = \frac{X _{1} + 6X _{2} + 2X _{3} + X _{4}}{10}\]
a) Покажите, что обе оценки несмещенные.
б) Какая из оценок оптимальная?




Обе они несмещенные, потому что в числителе выходит в сумме 10.
Какая-то точно должна быть, а может и нет....



\end{enumerate}

\section{Билет 112}

\begin{enumerate}


\item

Дайте определение случайной величины, которая имеет гамма-распределение $\Gamma(\alpha,  \lambda)$, и выведите основные свойства гамма-расределения. Запишите формулы для математичсекого ожидания
$\mathbb{E}(X)$ и дисперсии $\mathbb{V}ar(X)$ гамма-распределения




Здесь написанно много всего интересного и полезного о гамма-распределении


\item



Случайные величины $X$ и $Y$ независимы и имеют равномерное
распределение на отрезках $[0;7]$ и $[0;8]$ соответственно. Для случайной величины $Z=\frac{Y}{X}$ найдите: 
1) функцию распределения $F_Z(x)$;
2) плотность распределения $f_Z(x)$ и постройте график плотности;
3) вероятность $\P(1,\!072\leqslant Z\leqslant 1,\!953)$.




%\folder 2_53d22.png
1) Функция распределения $F_Z(x)$ имеет вид:
$
F_Z(x)=\left\{
\begin{array}{l}
0, x\leqslant 0;\\
\frac{7 x}{16}, 0\leqslant x\leqslant \frac{8}{7}\approx 1,\!143;\\
1 - \frac{4}{7 x}, x\geqslant\frac{8}{7};
\end{array}.
\right.
$
2) Плотность распределения $f_Z(x)$ имеет вид:
$
f_Z(x)=\left\{
\begin{array}{l}
0, x<0;\\
\frac{7}{16}, 0\leqslant x\leqslant \frac{8}{7}\approx 1,\!143;\\
\frac{4}{7 x^{2}}, x\geqslant\frac{8}{7};
\end{array}.
\right.
$


\begin{figure}[H]
    \includegraphics[width=0.9\textwidth]{2_53d22}
\end{figure}


3) вероятность равна:
$
\P(1,\!072\leqslant Z\leqslant 1,\!953)=
0,\!23843.
$


\item

    
	Случайная величина Y принимает только значения из множества $\{1, 10\}$, при этом $P(Y=1) = 0.7$.
	Распределение случайной величины X определено следующим образом:
	\begin{equation*}
		X | Y =
		\begin{cases}
			$5$ * y, с вероятностью $ 0.11$ \\
			$8$ * y, с вероятностью $ 1 - 0.11$
		\end{cases}
	\end{equation*}

	Юный аналитик Дарья нашла матожидание и дисперсию $X$.

	Помогите Дарье найти матожидание и дисперсию величины $X$
	


	

	Первым этапом надо найти характеристики случайной величины $Y$

	$E(Y) = 1 * 0.7 + 10 * (1 - 0.7)$

	$Var(Y) = E(Y^2) - [E(Y)]^2 = 1^2 * 0.7 + 10^2 * (1 - 0.7) - [E(Y)]^2$


	Перейдем к рассмотрению характеристик условной случайно величины X

	$E(X) = E(E(X|Y)) = E[E(5 * Y) * 0.11 + E(8 * Y) * (1 - 0.11)] = E(Y) * (5 * 0.11 + 8 * (1 - 0.11)) = 28.379$

	$E(Var(X|Y)) = E[b * Var(c3 * Y) + (1 - b) * Var(c4 * Y)] = Var(Y) * (c3^2 * b + c4^2 * (1- b)) $

	$Var(E(X|Y)) = E(X^2|Y) - [E(X)]^2 = [E(Y)]^2 * (b * c3^2 + (1-b)*c4^2) - E(X)]^2$

	$Var(X) = E(Var(X|Y)) + Var(E(X|Y)) = 1027.72936$
	

\item


(10) В группе $\Omega$ учатся студенты:$\omega _{1}...\omega _{25}$ . Пусть $X$ и $Y$ – 100-балльные экзаменационные оценки по
математическому анализу и теории вероятностей. Оценки $\omega _{i}$ студента обозначаются: $x _{i} = X(\omega _{i})$ и $y _{i} = Y(\omega _{i})$, $i = 1...25$. Все оценки известны
$x _{0} = 55, y _{0} = 55$, $x _{1} = 88, y _{1} = 86$, $x _{2} = 42, y _{2} = 96$, $x _{3} = 69, y _{3} = 93$, $x _{4} = 43, y _{4} = 64$, $x _{5} = 42, y _{5} = 86$, $x _{6} = 35, y _{6} = 45$, $x _{7} = 60, y _{7} = 55$, $x _{8} = 41, y _{8} = 90$, $x _{9} = 62, y _{9} = 57$, $x _{10} = 52, y _{10} = 53$, $x _{11} = 67, y _{11} = 32$, $x _{12} = 72, y _{12} = 98$, $x _{13} = 42, y _{13} = 84$, $x _{14} = 97, y _{14} = 51$, $x _{15} = 32, y _{15} = 89$, $x _{16} = 38, y _{16} = 84$, $x _{17} = 42, y _{17} = 84$, $x _{18} = 61, y _{18} = 94$, $x _{19} = 96, y _{19} = 31$, $x _{20} = 67, y _{20} = 56$, $x _{21} = 66, y _{21} = 67$, $x _{22} = 41, y _{22} = 95$, $x _{23} = 54, y _{23} = 95$, $x _{24} = 36, y _{24} = 80$
Требуется
найти следующие условные эмпирические характеристики: 1) ковариацию $X$ и $Y$ при условии, что одновременно $X \geqslant 50$
 и $Y \geqslant 50$; 2) коэффициент корреляции $X$ и $Y$ при том же условии.




1) Ковариация = $92.6667$
2) Коэффициент корреляции = $0.3814$


\item


(10) Эмпирическое распределение признаков $X$ и $Y$ на генеральной совокупности $\Omega$ задано таблицей частот  
 
\begin{tabular}{ | c | c | c | c | }
\hline
 & $Y = 2$ & $Y = 4$ & $Y = 5$  \\ \hline
$X = 200$ & $25$ & $26$ & $10$\\ \hline
$X = 300$ & $10$ & $10$ & $19$\\
\hline
\end{tabular}

Из $\Omega$ случайным образом без возвращения извлекаются $12$ элементов. 
Пусть $\bar X$ и $\bar Y$ – средние значения признаков на выбранных элементах. 
Требуется найти: 1) математическое ожидание $\mathbb{E}(\bar Y)$; 2) стандартное отклонение $\sigma(\bar X)$ ; 
3) ковариацию $Cov(\bar X, \bar Y)$




1) математическое ожидание $\mathbb{E}(\bar Y)$: $3.59$ 
2) стандартное отклонение $\sigma(\bar X)$: $228.8693$
3) ковариацию $Cov(\bar X, \bar Y)$: $1.3324$


\item


(10) Пусть $X _{1}$, $X _{2}$, $X _{3}$, $X _{4}$ выборка из $N(\theta, \sigma ^{2})$. Рассмотрим две оценки параметра $\theta$:
\[\hat \theta _{1} = \frac{X _{1} + 4X _{2} + X _{3} + 4X _{4}}{10}, \hat \theta _{1} = \frac{2X _{1} + 3X _{2} + 3X _{3} + 2X _{4}}{10}\]
a) Покажите, что обе оценки несмещенные.
б) Какая из оценок оптимальная?




Обе они несмещенные, потому что в числителе выходит в сумме 10.
Какая-то точно должна быть, а может и нет....



\end{enumerate}

\section{Билет 113}

\begin{enumerate}


\item


Сформулируйте определение случайной выборки из конечной генеральной совокупности. Какие
виды выборок вам известны? Перечислите (с указанием формул) основные характеристики выборочной и генеральной совокупностей




Здесь очень много исчерпывающей информации о выборках из генеральной совокупности и про различные виды выборок


\item

Случайные величины $X$ и $Y$ независимы и имеют равномерное
распределение на отрезках $[0;10]$ и $[0;3]$ соответственно. Для случайной величины $Z=\frac{Y}{X}$ найдите: 
1) функцию распределения $F_Z(x)$;
2) плотность распределения $f_Z(x)$ и постройте график плотности;
3) вероятность $\P(0,\!057\leqslant Z\leqslant 0,\!556)$.




%\folder 2_53d1.png
1) Функция распределения $F_Z(x)$ имеет вид:
$
F_Z(x)=\left\{
\begin{array}{l}
0, x\leqslant 0;\\
\frac{5 x}{3}, 0\leqslant x\leqslant \frac{3}{10}\approx 0,\!3;\\
1 - \frac{3}{20 x}, x\geqslant\frac{3}{10};
\end{array}.
\right.
$
2) Плотность распределения $f_Z(x)$ имеет вид:
$
f_Z(x)=\left\{
\begin{array}{l}
0, x<0;\\
\frac{5}{3}, 0\leqslant x\leqslant \frac{3}{10}\approx 0,\!3;\\
\frac{3}{20 x^{2}}, x\geqslant\frac{3}{10};
\end{array}.
\right.
$


\begin{figure}[H]
    \includegraphics[width=0.9\textwidth]{2_53d1}
\end{figure}


3) вероятность равна:
$
\P(0,\!057\leqslant Z\leqslant 0,\!556)=
0,\!63552.
$


\item


(10) Известно, что доля возвратов по кредитам в банке имеет распределение $F(x) = x ^{\beta}, 0 \leqslant x \leqslant 1$.
Наблюдения показали, что в среднем она составляет $75,0\%$. Методом моментов оцените параметр $\beta$ и
вероятность того, что она опуститься ниже $20\%$




Найдём плотность рапределения как интеграл от ФР, а дальше всё и вовсе простою Ответ: $8000$


\item

    
    Создайте эмперические совокупности  $\mathtt{\text{exp}}$ и $\mathtt{\text{log}}$ вида $\mathtt{\text{exp}}(1),\mathtt{\text{exp}}(2), ..., \mathtt{\text{exp}}(77) $ и $\mathtt{\text{log}}(1),\mathtt{\text{log}}(2), ..., \mathtt{\text{log}}(77). $

    Найдите эмпирическое среднее и эмпирическое стандартное отклонение совокупности $\mathtt{\text{exp}}$, её четвёртый эмпирический центральный момент и эмпирический эксцесс.

    Кроме того, найдите эмпирический коэффициент корреляции признаков $\mathtt{\text{exp}}$ и $\mathtt{\text{log}}$ на совокупности натуральных чисел от $1$ до $77$.
    


    
    Используя

	$E(X) = sum(X) / n$

	$Var(X) = E(X^2) - [E(X)]^2$

	$\mu_4(X) = E((X-E(X))^4)$

	$Ex = \frac{\mu_4(X)}{[\sigma(X)]^4} - 3$

	$r_{xy} = \frac{E(XY) - E(X) * E(Y)}{\sigma(X) * \sigma(Y)}$

    рассчитаем искомые значения.

    Ответы: $5.66740783200168 \cdot 10^{31}, 3.33285124990578 \cdot 10^{32}, 7.03150966623892 \cdot 10^{131}, 53.98819, 0.0006$.

    

\item

    
    	Распределение результатов экзамена в некоторой стране с $14$-балльной системой оценивания задано следующим образом:
    	$\left\{ 1 : 3, \  2 : 7, \  3 : 5, \  4 : 2, \  5 : 11, \  6 : 9, \  7 : 2, \  8 : 19, \  9 : 23, \  10 : 26, \  11 : 15, \  12 : 9, \  13 : 20, \  14 : 41\right\}$

	Работы будут перепроверять $16$ преподавателей, которые разделили все имеющиеся работы между собой случайным образом. Пусть $\overline{X}$ - средний балл (по перепроверки) работ, попавших к одному преподавателю.

	Требуется найти матожидание и стандартное отклонение среднего балла работ, попавших к одному преподавателю, до перепроверки.
    


    


    k = len(marks) // k

    ex = np.sum([marks[m] * m for m in marks]) / n

    varx = np.var([ m for m in marks for temp in range(marks[m])]) / k * (n - k) / (n - 1)

    sigmax = varx**(0.5)
    Ответы: $9.83854, 0.99615$.

    

\item


(10) Пусть $X _{1}$, $X _{2}$, $X _{3}$, $X _{4}$ выборка из $N(\theta, \sigma ^{2})$. Рассмотрим две оценки параметра $\theta$:
\[\hat \theta _{1} = \frac{X _{1} + X _{2} + X _{3} + 7X _{4}}{10}, \hat \theta _{1} = \frac{3X _{1} + 5X _{2} + X _{3} + X _{4}}{10}\]
a) Покажите, что обе оценки несмещенные.
б) Какая из оценок оптимальная?




Обе они несмещенные, потому что в числителе выходит в сумме 10.
Какая-то точно должна быть, а может и нет....



\end{enumerate}

\section{Билет 114}

\begin{enumerate}


\item


Дайте определение случайной величины, которая имеет $\chi ^{2}$-распределение с n степенями свободы.
Запишите плотность $\chi ^{2}$- распределения. Выведите формулы для математического ожидания $\mathbb{E}(X)$ и дисперсии $\mathbb{V}ar(X)$ $\chi ^{2}$-распределение с n степенями свободы. Найдите а) $\mathbb{P}(\chi _{20}^{2} > 10.9)$, где $\chi _{20}^{2}$–случайная величина, которая имеет $\chi ^{2}$– распределение с 20 степенями свободы; б) найдите 93\%
(верхнюю) точку $\chi _{0.93}^{2} (5)$ хи-квадрат распределения с 5 степенями свободы




$\mathbb{P}(\chi _{20}^{2} > 10.9) =  0.948775$; $\chi _{0.93}^{2} (5) = 1.34721$.


\item



Случайные величины $X$ и $Y$ независимы и имеют равномерное
распределение на отрезках $[0;2]$ и $[0;6]$ соответственно. Для случайной величины $Z=\frac{Y}{X}$ найдите: 
1) функцию распределения $F_Z(x)$;
2) плотность распределения $f_Z(x)$ и постройте график плотности;
3) вероятность $\P(2,\!532\leqslant Z\leqslant 4,\!716)$.




%\folder 2_53d20.png
1) Функция распределения $F_Z(x)$ имеет вид:
$
F_Z(x)=\left\{
\begin{array}{l}
0, x\leqslant 0;\\
\frac{x}{6}, 0\leqslant x\leqslant 3\approx 3,\!0;\\
1 - \frac{3}{2 x}, x\geqslant3;
\end{array}.
\right.
$
2) Плотность распределения $f_Z(x)$ имеет вид:
$
f_Z(x)=\left\{
\begin{array}{l}
0, x<0;\\
\frac{1}{6}, 0\leqslant x\leqslant 3\approx 3,\!0;\\
\frac{3}{2 x^{2}}, x\geqslant3;
\end{array}.
\right.
$


\begin{figure}[H]
    \includegraphics[width=0.9\textwidth]{2_53d20}
\end{figure}


3) вероятность равна:
$
\P(2,\!532\leqslant Z\leqslant 4,\!716)=
0,\!25993.
$


\item


(10) Известно, что доля возвратов по кредитам в банке имеет распределение $F(x) = x ^{\beta}, 0 \leqslant x \leqslant 1$.
Наблюдения показали, что в среднем она составляет $75,0\%$. Методом моментов оцените параметр $\beta$ и
вероятность того, что она опуститься ниже $52\%$




Найдём плотность рапределения как интеграл от ФР, а дальше всё и вовсе простою Ответ: $140608$


\item


(10) В группе $\Omega$ учатся студенты:$\omega _{1}...\omega _{25}$ . Пусть $X$ и $Y$ – 100-балльные экзаменационные оценки по
математическому анализу и теории вероятностей. Оценки $\omega _{i}$ студента обозначаются: $x _{i} = X(\omega _{i})$ и $y _{i} = Y(\omega _{i})$, $i = 1...25$. Все оценки известны
$x _{0} = 33, y _{0} = 72$, $x _{1} = 94, y _{1} = 94$, $x _{2} = 91, y _{2} = 52$, $x _{3} = 47, y _{3} = 59$, $x _{4} = 53, y _{4} = 45$, $x _{5} = 96, y _{5} = 54$, $x _{6} = 60, y _{6} = 99$, $x _{7} = 70, y _{7} = 44$, $x _{8} = 50, y _{8} = 81$, $x _{9} = 57, y _{9} = 40$, $x _{10} = 99, y _{10} = 61$, $x _{11} = 94, y _{11} = 43$, $x _{12} = 85, y _{12} = 96$, $x _{13} = 30, y _{13} = 91$, $x _{14} = 57, y _{14} = 37$, $x _{15} = 42, y _{15} = 35$, $x _{16} = 84, y _{16} = 75$, $x _{17} = 96, y _{17} = 97$, $x _{18} = 69, y _{18} = 92$, $x _{19} = 91, y _{19} = 93$, $x _{20} = 45, y _{20} = 30$, $x _{21} = 35, y _{21} = 94$, $x _{22} = 83, y _{22} = 53$, $x _{23} = 53, y _{23} = 60$, $x _{24} = 36, y _{24} = 69$
Требуется
найти следующие условные эмпирические характеристики: 1) ковариацию $X$ и $Y$ при условии, что одновременно $X \geqslant 50$
 и $Y \geqslant 50$; 2) коэффициент корреляции $X$ и $Y$ при том же условии.




1) Ковариация = $-350.8333$
2) Коэффициент корреляции = $-1.2925$


\item


(10) Эмпирическое распределение признаков $X$ и $Y$ на генеральной совокупности $\Omega$ задано таблицей частот  
 
\begin{tabular}{ | c | c | c | c | }
\hline
 & $Y = 2$ & $Y = 4$ & $Y = 5$  \\ \hline
$X = 200$ & $16$ & $16$ & $22$\\ \hline
$X = 300$ & $7$ & $26$ & $13$\\
\hline
\end{tabular}

Из $\Omega$ случайным образом без возвращения извлекаются $9$ элементов. 
Пусть $\bar X$ и $\bar Y$ – средние значения признаков на выбранных элементах. 
Требуется найти: 1) математическое ожидание $\mathbb{E}(\bar Y)$; 2) стандартное отклонение $\sigma(\bar X)$ ; 
3) ковариацию $Cov(\bar X, \bar Y)$




1) математическое ожидание $\mathbb{E}(\bar Y)$: $3.89$ 
2) стандартное отклонение $\sigma(\bar X)$: $239.4845$
3) ковариацию $Cov(\bar X, \bar Y)$: $0.3732$


\item

    
	Известно, что доля возвратов по кредитам в банке имеет распределение $F(x) = x^{\beta}, 0 \le x \le 1$. Наблюдения показали, что в среднем она составила $67.0$\%. Методом моментов оцените параметр $\beta$ и вероятность того, что она опуститься ниже $52.0$\%.
	


	

	$f(x) = F'(x) = \beta \cdot x^{\beta - 1}$

	$\mu_{1} = E(X) = \int_{-\inf}^{\inf}x \cdot f(x) = \int_{-\inf}^{\inf} \beta \cdot x^{\beta} = \beta \cdot \frac{x^{\beta + 1}}{\beta + 1}\bigg|_0^1 = \frac{\beta}{\beta + 1}$

	$\beta = (\beta + 1) \cdot 67.0$

	$\beta = \frac{67.0}{1 - 67.0}$

	$ P(x \le 52.0) = F(52.0) = 52.0^{2.03} $

    Ответ: $2.03, 0.27$
	


\end{enumerate}

\section{Билет 115}

\begin{enumerate}


\item


Дайте определение случайной величины, которая имеет $\chi ^{2}$-распределение с n степенями свободы.
Запишите плотность $\chi ^{2}$- распределения. Выведите формулы для математического ожидания $\mathbb{E}(X)$ и дисперсии $\mathbb{V}ar(X)$ $\chi ^{2}$-распределение с n степенями свободы. Найдите а) $\mathbb{P}(\chi _{20}^{2} > 10.9)$, где $\chi _{20}^{2}$–случайная величина, которая имеет $\chi ^{2}$– распределение с 20 степенями свободы; б) найдите 93\%
(верхнюю) точку $\chi _{0.93}^{2} (5)$ хи-квадрат распределения с 5 степенями свободы




$\mathbb{P}(\chi _{20}^{2} > 10.9) =  0.948775$; $\chi _{0.93}^{2} (5) = 1.34721$.


\item



Случайные величины $X$ и $Y$ независимы и имеют равномерное
распределение на отрезках $[0;4]$ и $[0;7]$ соответственно. Для случайной величины $Z=\frac{Y}{X}$ найдите: 
1) функцию распределения $F_Z(x)$;
2) плотность распределения $f_Z(x)$ и постройте график плотности;
3) вероятность $\P(0,\!035\leqslant Z\leqslant 2,\!775)$.




%\folder 2_53d12.png
1) Функция распределения $F_Z(x)$ имеет вид:
$
F_Z(x)=\left\{
\begin{array}{l}
0, x\leqslant 0;\\
\frac{2 x}{7}, 0\leqslant x\leqslant \frac{7}{4}\approx 1,\!75;\\
1 - \frac{7}{8 x}, x\geqslant\frac{7}{4};
\end{array}.
\right.
$
2) Плотность распределения $f_Z(x)$ имеет вид:
$
f_Z(x)=\left\{
\begin{array}{l}
0, x<0;\\
\frac{2}{7}, 0\leqslant x\leqslant \frac{7}{4}\approx 1,\!75;\\
\frac{7}{8 x^{2}}, x\geqslant\frac{7}{4};
\end{array}.
\right.
$


\begin{figure}[H]
    \includegraphics[width=0.9\textwidth]{2_53d12}
\end{figure}


3) вероятность равна:
$
\P(0,\!035\leqslant Z\leqslant 2,\!775)=
0,\!67474.
$


\item


%\folder 2.pdf
(10) Известно, что доля возвратов по кредитам в банке имеет распределение $F(x) = x ^{\beta}, 0 \leqslant x \leqslant 1$.
Наблюдения показали, что в среднем она составляет $75,0\%$. Методом моментов оцените параметр $\beta$ и
вероятность того, что она опуститься ниже $20\%$




Найдём плотность рапределения как интеграл от ФР, а дальше всё и вовсе простою Ответ: $8000$


\item


(10) В группе $\Omega$ учатся студенты:$\omega _{1}...\omega _{25}$ . Пусть $X$ и $Y$ – 100-балльные экзаменационные оценки по
математическому анализу и теории вероятностей. Оценки $\omega _{i}$ студента обозначаются: $x _{i} = X(\omega _{i})$ и $y _{i} = Y(\omega _{i})$, $i = 1...25$. Все оценки известны
$x _{0} = 73, y _{0} = 44$, $x _{1} = 44, y _{1} = 83$, $x _{2} = 49, y _{2} = 41$, $x _{3} = 36, y _{3} = 32$, $x _{4} = 48, y _{4} = 60$, $x _{5} = 53, y _{5} = 37$, $x _{6} = 70, y _{6} = 86$, $x _{7} = 61, y _{7} = 82$, $x _{8} = 42, y _{8} = 57$, $x _{9} = 94, y _{9} = 40$, $x _{10} = 44, y _{10} = 78$, $x _{11} = 85, y _{11} = 78$, $x _{12} = 48, y _{12} = 66$, $x _{13} = 88, y _{13} = 82$, $x _{14} = 31, y _{14} = 39$, $x _{15} = 84, y _{15} = 68$, $x _{16} = 49, y _{16} = 51$, $x _{17} = 84, y _{17} = 55$, $x _{18} = 65, y _{18} = 67$, $x _{19} = 37, y _{19} = 99$, $x _{20} = 46, y _{20} = 31$, $x _{21} = 84, y _{21} = 46$, $x _{22} = 40, y _{22} = 67$, $x _{23} = 86, y _{23} = 54$, $x _{24} = 89, y _{24} = 32$
Требуется
найти следующие условные эмпирические характеристики: 1) ковариацию $X$ и $Y$ при условии, что одновременно $X \geqslant 50$
 и $Y \geqslant 50$; 2) коэффициент корреляции $X$ и $Y$ при том же условии.




1) Ковариация = $-345.5$
2) Коэффициент корреляции = $-2.9554$


\item


(10) Эмпирическое распределение признаков $X$ и $Y$ на генеральной совокупности $\Omega$ задано таблицей частот  
 
\begin{tabular}{ | c | c | c | c | }
\hline
 & $Y = 2$ & $Y = 4$ & $Y = 5$  \\ \hline
$X = 200$ & $1$ & $6$ & $23$\\ \hline
$X = 300$ & $13$ & $30$ & $27$\\
\hline
\end{tabular}

Из $\Omega$ случайным образом без возвращения извлекаются $13$ элементов. 
Пусть $\bar X$ и $\bar Y$ – средние значения признаков на выбранных элементах. 
Требуется найти: 1) математическое ожидание $\mathbb{E}(\bar Y)$; 2) стандартное отклонение $\sigma(\bar X)$ ; 
3) ковариацию $Cov(\bar X, \bar Y)$




1) математическое ожидание $\mathbb{E}(\bar Y)$: $4.22$ 
2) стандартное отклонение $\sigma(\bar X)$: $255.4769$
3) ковариацию $Cov(\bar X, \bar Y)$: $-1.2655$


\item


(10) Пусть $X _{1}$, $X _{2}$, $X _{3}$, $X _{4}$ выборка из $N(\theta, \sigma ^{2})$. Рассмотрим две оценки параметра $\theta$:
\[\hat \theta _{1} = \frac{2X _{1} + 6X _{2} + X _{3} + X _{4}}{10}, \hat \theta _{1} = \frac{5X _{1} + X _{2} + X _{3} + 3X _{4}}{10}\]
a) Покажите, что обе оценки несмещенные.
б) Какая из оценок оптимальная?




Обе они несмещенные, потому что в числителе выходит в сумме 10.
Какая-то точно должна быть, а может и нет....



\end{enumerate}

\section{Билет 116}

\begin{enumerate}


\item


Дайте определение случайной величины, которая имеет $\chi ^{2}$-распределение с n степенями свободы.
Запишите плотность $\chi ^{2}$- распределения. Выведите формулы для математического ожидания $\mathbb{E}(X)$ и дисперсии $\mathbb{V}ar(X)$ $\chi ^{2}$-распределение с n степенями свободы. Найдите а) $\mathbb{P}(\chi _{20}^{2} > 10.9)$, где $\chi _{20}^{2}$–случайная величина, которая имеет $\chi ^{2}$– распределение с 20 степенями свободы; б) найдите 93\%
(верхнюю) точку $\chi _{0.93}^{2} (5)$ хи-квадрат распределения с 5 степенями свободы




$\mathbb{P}(\chi _{20}^{2} > 10.9) =  0.948775$; $\chi _{0.93}^{2} (5) = 1.34721$.


\item



Случайные величины $X$ и $Y$ независимы и имеют равномерное
распределение на отрезках $[0;7]$ и $[0;3]$ соответственно. Для случайной величины $Z=\frac{Y}{X}$ найдите: 
1) функцию распределения $F_Z(x)$;
2) плотность распределения $f_Z(x)$ и постройте график плотности;
3) вероятность $\P(0,\!006\leqslant Z\leqslant 0,\!519)$.




%\folder 2_53d18.png
1) Функция распределения $F_Z(x)$ имеет вид:
$
F_Z(x)=\left\{
\begin{array}{l}
0, x\leqslant 0;\\
\frac{7 x}{6}, 0\leqslant x\leqslant \frac{3}{7}\approx 0,\!429;\\
1 - \frac{3}{14 x}, x\geqslant\frac{3}{7};
\end{array}.
\right.
$
2) Плотность распределения $f_Z(x)$ имеет вид:
$
f_Z(x)=\left\{
\begin{array}{l}
0, x<0;\\
\frac{7}{6}, 0\leqslant x\leqslant \frac{3}{7}\approx 0,\!429;\\
\frac{3}{14 x^{2}}, x\geqslant\frac{3}{7};
\end{array}.
\right.
$


\begin{figure}[H]
    \includegraphics[width=0.9\textwidth]{2_53d18}
\end{figure}


3) вероятность равна:
$
\P(0,\!006\leqslant Z\leqslant 0,\!519)=
0,\!57962.
$


\item

    
	Случайная величина Y принимает только значения из множества $\{3, 4\}$, при этом $P(Y=3) = 0.33$.
	Распределение случайной величины X определено следующим образом:
	\begin{equation*}
		X | Y =
		\begin{cases}
			$9$ * y, с вероятностью $ 0.34$ \\
			$7$ * y, с вероятностью $ 1 - 0.34$
		\end{cases}
	\end{equation*}

	Юный аналитик Дарья нашла матожидание и дисперсию $X$.

	Помогите Дарье найти матожидание и дисперсию величины $X$
	


	

	Первым этапом надо найти характеристики случайной величины $Y$

	$E(Y) = 3 * 0.33 + 4 * (1 - 0.33)$

	$Var(Y) = E(Y^2) - [E(Y)]^2 = 3^2 * 0.33 + 4^2 * (1 - 0.33) - [E(Y)]^2$


	Перейдем к рассмотрению характеристик условной случайно величины X

	$E(X) = E(E(X|Y)) = E[E(9 * Y) * 0.34 + E(7 * Y) * (1 - 0.34)] = E(Y) * (9 * 0.34 + 7 * (1 - 0.34)) = 28.1856$

	$E(Var(X|Y)) = E[b * Var(c3 * Y) + (1 - b) * Var(c4 * Y)] = Var(Y) * (c3^2 * b + c4^2 * (1- b)) $

	$Var(E(X|Y)) = E(X^2|Y) - [E(X)]^2 = [E(Y)]^2 * (b * c3^2 + (1-b)*c4^2) - E(X)]^2$

	$Var(X) = E(Var(X|Y)) + Var(E(X|Y)) = 25.32915$
	

\item


(10) В группе $\Omega$ учатся студенты:$\omega _{1}...\omega _{25}$ . Пусть $X$ и $Y$ – 100-балльные экзаменационные оценки по
математическому анализу и теории вероятностей. Оценки $\omega _{i}$ студента обозначаются: $x _{i} = X(\omega _{i})$ и $y _{i} = Y(\omega _{i})$, $i = 1...25$. Все оценки известны
$x _{0} = 55, y _{0} = 54$, $x _{1} = 64, y _{1} = 68$, $x _{2} = 34, y _{2} = 51$, $x _{3} = 48, y _{3} = 73$, $x _{4} = 81, y _{4} = 69$, $x _{5} = 62, y _{5} = 69$, $x _{6} = 76, y _{6} = 59$, $x _{7} = 84, y _{7} = 45$, $x _{8} = 97, y _{8} = 77$, $x _{9} = 76, y _{9} = 87$, $x _{10} = 43, y _{10} = 67$, $x _{11} = 33, y _{11} = 55$, $x _{12} = 71, y _{12} = 96$, $x _{13} = 62, y _{13} = 97$, $x _{14} = 84, y _{14} = 37$, $x _{15} = 41, y _{15} = 70$, $x _{16} = 92, y _{16} = 41$, $x _{17} = 60, y _{17} = 54$, $x _{18} = 71, y _{18} = 44$, $x _{19} = 39, y _{19} = 70$, $x _{20} = 98, y _{20} = 75$, $x _{21} = 99, y _{21} = 32$, $x _{22} = 58, y _{22} = 42$, $x _{23} = 61, y _{23} = 92$, $x _{24} = 58, y _{24} = 32$
Требуется
найти следующие условные эмпирические характеристики: 1) ковариацию $X$ и $Y$ при условии, что одновременно $X \geqslant 50$
 и $Y \geqslant 50$; 2) коэффициент корреляции $X$ и $Y$ при том же условии.




1) Ковариация = $276.75$
2) Коэффициент корреляции = $1.373$


\item


(10) Эмпирическое распределение признаков $X$ и $Y$ на генеральной совокупности $\Omega$ задано таблицей частот  
 
\begin{tabular}{ | c | c | c | c | }
\hline
 & $Y = 2$ & $Y = 4$ & $Y = 5$  \\ \hline
$X = 200$ & $1$ & $18$ & $12$\\ \hline
$X = 300$ & $31$ & $26$ & $12$\\
\hline
\end{tabular}

Из $\Omega$ случайным образом без возвращения извлекаются $12$ элементов. 
Пусть $\bar X$ и $\bar Y$ – средние значения признаков на выбранных элементах. 
Требуется найти: 1) математическое ожидание $\mathbb{E}(\bar Y)$; 2) стандартное отклонение $\sigma(\bar X)$ ; 
3) ковариацию $Cov(\bar X, \bar Y)$




1) математическое ожидание $\mathbb{E}(\bar Y)$: $3.6$ 
2) стандартное отклонение $\sigma(\bar X)$: $256.084$
3) ковариацию $Cov(\bar X, \bar Y)$: $-1.9911$


\item


(10) Пусть $X _{1}$, $X _{2}$, $X _{3}$, $X _{4}$ выборка из $N(\theta, \sigma ^{2})$. Рассмотрим две оценки параметра $\theta$:
\[\hat \theta _{1} = \frac{X _{1} + 6X _{2} + X _{3} + 2X _{4}}{10}, \hat \theta _{1} = \frac{3X _{1} + X _{2} + 3X _{3} + 3X _{4}}{10}\]
a) Покажите, что обе оценки несмещенные.
б) Какая из оценок оптимальная?




Обе они несмещенные, потому что в числителе выходит в сумме 10.
Какая-то точно должна быть, а может и нет....



\end{enumerate}

\section{Билет 117}

\begin{enumerate}


\item


Дайте определение случайной величины, которая имеет $\chi ^{2}$-распределение с n степенями свободы.
Запишите плотность $\chi ^{2}$- распределения. Выведите формулы для математического ожидания $\mathbb{E}(X)$ и дисперсии $\mathbb{V}ar(X)$ $\chi ^{2}$-распределение с n степенями свободы. Найдите а) $\mathbb{P}(\chi _{20}^{2} > 10.9)$, где $\chi _{20}^{2}$–случайная величина, которая имеет $\chi ^{2}$– распределение с 20 степенями свободы; б) найдите 93\%
(верхнюю) точку $\chi _{0.93}^{2} (5)$ хи-квадрат распределения с 5 степенями свободы




$\mathbb{P}(\chi _{20}^{2} > 10.9) =  0.948775$; $\chi _{0.93}^{2} (5) = 1.34721$.


\item


(10) Сформулируйте критерий независимости $\chi ^ {2}$ – Пирсона. Приведите (с выводом и
необходимыми пояснениями в обозначениях) явный вид статистики критерия в случае, когда 
таблица сопряженности двух признаков $X$ и $Y$ имеет вид

\begin{tabular}[b]{ | c | c | c | }
\hline
$ $ & $Y = y _{1}$ & $Y = y _{2}$  \\ \hline
$X = x _{1}$ & $a$ & $b$ \\ \hline
$X = x _{2}$ & $c$ & $d$ \\
\hline
\end{tabular}




Здесь формулировки критерия независимости Пирсона и приводится пример


\item


%\folder 2.pdf
(10) Известно, что доля возвратов по кредитам в банке имеет распределение $F(x) = x ^{\beta}, 0 \leqslant x \leqslant 1$.
Наблюдения показали, что в среднем она составляет $75,0\%$. Методом моментов оцените параметр $\beta$ и
вероятность того, что она опуститься ниже $20\%$




Найдём плотность рапределения как интеграл от ФР, а дальше всё и вовсе простою Ответ: $8000$


\item


(10) В группе $\Omega$ учатся студенты:$\omega _{1}...\omega _{25}$ . Пусть $X$ и $Y$ – 100-балльные экзаменационные оценки по
математическому анализу и теории вероятностей. Оценки $\omega _{i}$ студента обозначаются: $x _{i} = X(\omega _{i})$ и $y _{i} = Y(\omega _{i})$, $i = 1...25$. Все оценки известны
$x _{0} = 73, y _{0} = 44$, $x _{1} = 44, y _{1} = 83$, $x _{2} = 49, y _{2} = 41$, $x _{3} = 36, y _{3} = 32$, $x _{4} = 48, y _{4} = 60$, $x _{5} = 53, y _{5} = 37$, $x _{6} = 70, y _{6} = 86$, $x _{7} = 61, y _{7} = 82$, $x _{8} = 42, y _{8} = 57$, $x _{9} = 94, y _{9} = 40$, $x _{10} = 44, y _{10} = 78$, $x _{11} = 85, y _{11} = 78$, $x _{12} = 48, y _{12} = 66$, $x _{13} = 88, y _{13} = 82$, $x _{14} = 31, y _{14} = 39$, $x _{15} = 84, y _{15} = 68$, $x _{16} = 49, y _{16} = 51$, $x _{17} = 84, y _{17} = 55$, $x _{18} = 65, y _{18} = 67$, $x _{19} = 37, y _{19} = 99$, $x _{20} = 46, y _{20} = 31$, $x _{21} = 84, y _{21} = 46$, $x _{22} = 40, y _{22} = 67$, $x _{23} = 86, y _{23} = 54$, $x _{24} = 89, y _{24} = 32$
Требуется
найти следующие условные эмпирические характеристики: 1) ковариацию $X$ и $Y$ при условии, что одновременно $X \geqslant 50$
 и $Y \geqslant 50$; 2) коэффициент корреляции $X$ и $Y$ при том же условии.




1) Ковариация = $-345.5$
2) Коэффициент корреляции = $-2.9554$


\item


(10) Эмпирическое распределение признаков $X$ и $Y$ на генеральной совокупности $\Omega$ задано таблицей частот  
 
\begin{tabular}{ | c | c | c | c | }
\hline
 & $Y = 2$ & $Y = 4$ & $Y = 5$  \\ \hline
$X = 200$ & $16$ & $19$ & $5$\\ \hline
$X = 300$ & $25$ & $10$ & $25$\\
\hline
\end{tabular}

Из $\Omega$ случайным образом без возвращения извлекаются $6$ элементов. 
Пусть $\bar X$ и $\bar Y$ – средние значения признаков на выбранных элементах. 
Требуется найти: 1) математическое ожидание $\mathbb{E}(\bar Y)$; 2) стандартное отклонение $\sigma(\bar X)$ ; 
3) ковариацию $Cov(\bar X, \bar Y)$




1) математическое ожидание $\mathbb{E}(\bar Y)$: $3.48$ 
2) стандартное отклонение $\sigma(\bar X)$: $256.5595$
3) ковариацию $Cov(\bar X, \bar Y)$: $0.5887$


\item

    
	Известно, что доля возвратов по кредитам в банке имеет распределение $F(x) = x^{\beta}, 0 \le x \le 1$. Наблюдения показали, что в среднем она составила $67.0$\%. Методом моментов оцените параметр $\beta$ и вероятность того, что она опуститься ниже $52.0$\%.
	


	

	$f(x) = F'(x) = \beta \cdot x^{\beta - 1}$

	$\mu_{1} = E(X) = \int_{-\inf}^{\inf}x \cdot f(x) = \int_{-\inf}^{\inf} \beta \cdot x^{\beta} = \beta \cdot \frac{x^{\beta + 1}}{\beta + 1}\bigg|_0^1 = \frac{\beta}{\beta + 1}$

	$\beta = (\beta + 1) \cdot 67.0$

	$\beta = \frac{67.0}{1 - 67.0}$

	$ P(x \le 52.0) = F(52.0) = 52.0^{2.03} $

    Ответ: $2.03, 0.27$
	


\end{enumerate}

\section{Билет 118}

\begin{enumerate}


\item


Дайте определение случайной величины, которая имеет $\chi ^{2}$-распределение с n степенями свободы.
Запишите плотность $\chi ^{2}$- распределения. Выведите формулы для математического ожидания $\mathbb{E}(X)$ и дисперсии $\mathbb{V}ar(X)$ $\chi ^{2}$-распределение с n степенями свободы. Найдите а) $\mathbb{P}(\chi _{20}^{2} > 10.9)$, где $\chi _{20}^{2}$–случайная величина, которая имеет $\chi ^{2}$– распределение с 20 степенями свободы; б) найдите 93\%
(верхнюю) точку $\chi _{0.93}^{2} (5)$ хи-квадрат распределения с 5 степенями свободы




$\mathbb{P}(\chi _{20}^{2} > 10.9) =  0.948775$; $\chi _{0.93}^{2} (5) = 1.34721$.


\item



Случайные величины $X$ и $Y$ независимы и имеют равномерное
распределение на отрезках $[0;3]$ и $[0;8]$ соответственно. Для случайной величины $Z=\frac{Y}{X}$ найдите: 
1) функцию распределения $F_Z(x)$;
2) плотность распределения $f_Z(x)$ и постройте график плотности;
3) вероятность $\P(2,\!475\leqslant Z\leqslant 4,\!811)$.




%\folder 2_53d8.png
1) Функция распределения $F_Z(x)$ имеет вид:
$
F_Z(x)=\left\{
\begin{array}{l}
0, x\leqslant 0;\\
\frac{3 x}{16}, 0\leqslant x\leqslant \frac{8}{3}\approx 2,\!667;\\
1 - \frac{4}{3 x}, x\geqslant\frac{8}{3};
\end{array}.
\right.
$
2) Плотность распределения $f_Z(x)$ имеет вид:
$
f_Z(x)=\left\{
\begin{array}{l}
0, x<0;\\
\frac{3}{16}, 0\leqslant x\leqslant \frac{8}{3}\approx 2,\!667;\\
\frac{4}{3 x^{2}}, x\geqslant\frac{8}{3};
\end{array}.
\right.
$


\begin{figure}[H]
    \includegraphics[width=0.9\textwidth]{2_53d8}
\end{figure}


3) вероятность равна:
$
\P(2,\!475\leqslant Z\leqslant 4,\!811)=
0,\!25884.
$


\item


(10) Известно, что доля возвратов по кредитам в банке имеет распределение $F(x) = x ^{\beta}, 0 \leqslant x \leqslant 1$.
Наблюдения показали, что в среднем она составляет $91,6667\%$. Методом моментов оцените параметр $\beta$ и
вероятность того, что она опуститься ниже $59\%$




Найдём плотность рапределения как интеграл от ФР, а дальше всё и вовсе простою Ответ: $30155888444737842659$


\item


(10) В группе $\Omega$ учатся студенты:$\omega _{1}...\omega _{25}$ . Пусть $X$ и $Y$ – 100-балльные экзаменационные оценки по
математическому анализу и теории вероятностей. Оценки $\omega _{i}$ студента обозначаются: $x _{i} = X(\omega _{i})$ и $y _{i} = Y(\omega _{i})$, $i = 1...25$. Все оценки известны
$x _{0} = 64, y _{0} = 84$, $x _{1} = 82, y _{1} = 42$, $x _{2} = 51, y _{2} = 99$, $x _{3} = 68, y _{3} = 57$, $x _{4} = 90, y _{4} = 71$, $x _{5} = 89, y _{5} = 55$, $x _{6} = 55, y _{6} = 55$, $x _{7} = 90, y _{7} = 58$, $x _{8} = 61, y _{8} = 78$, $x _{9} = 38, y _{9} = 84$, $x _{10} = 56, y _{10} = 95$, $x _{11} = 86, y _{11} = 69$, $x _{12} = 71, y _{12} = 72$, $x _{13} = 35, y _{13} = 99$, $x _{14} = 82, y _{14} = 67$, $x _{15} = 79, y _{15} = 59$, $x _{16} = 83, y _{16} = 88$, $x _{17} = 45, y _{17} = 75$, $x _{18} = 70, y _{18} = 79$, $x _{19} = 89, y _{19} = 80$, $x _{20} = 33, y _{20} = 30$, $x _{21} = 63, y _{21} = 73$, $x _{22} = 55, y _{22} = 53$, $x _{23} = 31, y _{23} = 78$, $x _{24} = 50, y _{24} = 90$
Требуется
найти следующие условные эмпирические характеристики: 1) ковариацию $X$ и $Y$ при условии, что одновременно $X \geqslant 50$
 и $Y \geqslant 50$; 2) коэффициент корреляции $X$ и $Y$ при том же условии.




1) Ковариация = $-876.6667$
2) Коэффициент корреляции = $-4.7659$


\item


(10) Эмпирическое распределение признаков $X$ и $Y$ на генеральной совокупности $\Omega$ задано таблицей частот  
 
\begin{tabular}{ | c | c | c | c | }
\hline
 & $Y = 2$ & $Y = 4$ & $Y = 5$  \\ \hline
$X = 200$ & $17$ & $3$ & $13$\\ \hline
$X = 300$ & $21$ & $23$ & $23$\\
\hline
\end{tabular}

Из $\Omega$ случайным образом без возвращения извлекаются $10$ элементов. 
Пусть $\bar X$ и $\bar Y$ – средние значения признаков на выбранных элементах. 
Требуется найти: 1) математическое ожидание $\mathbb{E}(\bar Y)$; 2) стандартное отклонение $\sigma(\bar X)$ ; 
3) ковариацию $Cov(\bar X, \bar Y)$




1) математическое ожидание $\mathbb{E}(\bar Y)$: $3.6$ 
2) стандартное отклонение $\sigma(\bar X)$: $257.2355$
3) ковариацию $Cov(\bar X, \bar Y)$: $0.7091$


\item


(10) Пусть $X _{1}$, $X _{2}$, $X _{3}$, $X _{4}$ выборка из $N(\theta, \sigma ^{2})$. Рассмотрим две оценки параметра $\theta$:
\[\hat \theta _{1} = \frac{X _{1} + 4X _{2} + X _{3} + 4X _{4}}{10}, \hat \theta _{1} = \frac{2X _{1} + 3X _{2} + 3X _{3} + 2X _{4}}{10}\]
a) Покажите, что обе оценки несмещенные.
б) Какая из оценок оптимальная?




Обе они несмещенные, потому что в числителе выходит в сумме 10.
Какая-то точно должна быть, а может и нет....



\end{enumerate}

\section{Билет 119}

\begin{enumerate}


\item

Дайте определение случайной величины, которая имеет гамма-распределение $\Gamma(\alpha,  \lambda)$, и выведите основные свойства гамма-расределения. Запишите формулы для математичсекого ожидания
$\mathbb{E}(X)$ и дисперсии $\mathbb{V}ar(X)$ гамма-распределения




Здесь написанно много всего интересного и полезного о гамма-распределении


\item



Случайные величины $X$ и $Y$ независимы и имеют равномерное
распределение на отрезках $[0;10]$ и $[0;9]$ соответственно. Для случайной величины $Z=\frac{Y}{X}$ найдите: 
1) функцию распределения $F_Z(x)$;
2) плотность распределения $f_Z(x)$ и постройте график плотности;
3) вероятность $\P(0,\!719\leqslant Z\leqslant 1,\!005)$.




%\folder 2_53d14.png
1) Функция распределения $F_Z(x)$ имеет вид:
$
F_Z(x)=\left\{
\begin{array}{l}
0, x\leqslant 0;\\
\frac{5 x}{9}, 0\leqslant x\leqslant \frac{9}{10}\approx 0,\!9;\\
1 - \frac{9}{20 x}, x\geqslant\frac{9}{10};
\end{array}.
\right.
$
2) Плотность распределения $f_Z(x)$ имеет вид:
$
f_Z(x)=\left\{
\begin{array}{l}
0, x<0;\\
\frac{5}{9}, 0\leqslant x\leqslant \frac{9}{10}\approx 0,\!9;\\
\frac{9}{20 x^{2}}, x\geqslant\frac{9}{10};
\end{array}.
\right.
$


\begin{figure}[H]
    \includegraphics[width=0.9\textwidth]{2_53d14}
\end{figure}


3) вероятность равна:
$
\P(0,\!719\leqslant Z\leqslant 1,\!005)=
0,\!15287.
$


\item


(10) Известно, что доля возвратов по кредитам в банке имеет распределение $F(x) = x ^{\beta}, 0 \leqslant x \leqslant 1$.
Наблюдения показали, что в среднем она составляет $93,3333\%$. Методом моментов оцените параметр $\beta$ и
вероятность того, что она опуститься ниже $19\%$




Найдём плотность рапределения как интеграл от ФР, а дальше всё и вовсе простою Ответ: $799006685782884121$


\item

    
    Создайте эмперические совокупности  $\mathtt{\text{exp}}$ и $\mathtt{\text{sin}}$ вида $\mathtt{\text{exp}}(1),\mathtt{\text{exp}}(2), ..., \mathtt{\text{exp}}(85) $ и $\mathtt{\text{sin}}(1),\mathtt{\text{sin}}(2), ..., \mathtt{\text{sin}}(85). $

    Найдите эмпирическое среднее и эмпирическое стандартное отклонение совокупности $\mathtt{\text{exp}}$, её четвёртый эмпирический центральный момент и эмпирический эксцесс.

    Кроме того, найдите эмпирический коэффициент корреляции признаков $\mathtt{\text{exp}}$ и $\mathtt{\text{sin}}$ на совокупности натуральных чисел от $1$ до $85$.
    


    
    Используя

	$E(X) = sum(X) / n$

	$Var(X) = E(X^2) - [E(X)]^2$

	$\mu_4(X) = E((X-E(X))^4)$

	$Ex = \frac{\mu_4(X)}{[\sigma(X)]^4} - 3$

	$r_{xy} = \frac{E(XY) - E(X) * E(Y)}{\sigma(X) * \sigma(Y)}$

    рассчитаем искомые значения.

    Ответы: $1.53042524409691 \cdot 10^{35}, 9.46886335007349 \cdot 10^{35}, 5.07073544919377 \cdot 10^{145}, 60.07824, 0.00032$.

    

\item


(10) Эмпирическое распределение признаков $X$ и $Y$ на генеральной совокупности $\Omega$ задано таблицей частот  
 
\begin{tabular}{ | c | c | c | c | }
\hline
 & $Y = 2$ & $Y = 4$ & $Y = 5$  \\ \hline
$X = 200$ & $28$ & $23$ & $3$\\ \hline
$X = 300$ & $2$ & $12$ & $32$\\
\hline
\end{tabular}

Из $\Omega$ случайным образом без возвращения извлекаются $5$ элементов. 
Пусть $\bar X$ и $\bar Y$ – средние значения признаков на выбранных элементах. 
Требуется найти: 1) математическое ожидание $\mathbb{E}(\bar Y)$; 2) стандартное отклонение $\sigma(\bar X)$ ; 
3) ковариацию $Cov(\bar X, \bar Y)$




1) математическое ожидание $\mathbb{E}(\bar Y)$: $3.75$ 
2) стандартное отклонение $\sigma(\bar X)$: $244.6913$
3) ковариацию $Cov(\bar X, \bar Y)$: $3.7904$


\item


(10) Пусть $X _{1}$, $X _{2}$, $X _{3}$, $X _{4}$ выборка из $N(\theta, \sigma ^{2})$. Рассмотрим две оценки параметра $\theta$:
\[\hat \theta _{1} = \frac{2X _{1} + 3X _{2} + 4X _{3} + X _{4}}{10}, \hat \theta _{1} = \frac{2X _{1} + 3X _{2} + 2X _{3} + 3X _{4}}{10}\]
a) Покажите, что обе оценки несмещенные.
б) Какая из оценок оптимальная?




Обе они несмещенные, потому что в числителе выходит в сумме 10.
Какая-то точно должна быть, а может и нет....



\end{enumerate}

\section{Билет 120}

\begin{enumerate}


\item


Дайте определение случайной величины, которая имеет $\chi ^{2}$-распределение с n степенями свободы.
Запишите плотность $\chi ^{2}$- распределения. Выведите формулы для математического ожидания $\mathbb{E}(X)$ и дисперсии $\mathbb{V}ar(X)$ $\chi ^{2}$-распределение с n степенями свободы. Найдите а) $\mathbb{P}(\chi _{20}^{2} > 10.9)$, где $\chi _{20}^{2}$–случайная величина, которая имеет $\chi ^{2}$– распределение с 20 степенями свободы; б) найдите 93\%
(верхнюю) точку $\chi _{0.93}^{2} (5)$ хи-квадрат распределения с 5 степенями свободы




$\mathbb{P}(\chi _{20}^{2} > 10.9) =  0.948775$; $\chi _{0.93}^{2} (5) = 1.34721$.


\item



Случайные величины $X$ и $Y$ независимы и имеют равномерное
распределение на отрезках $[0;4]$ и $[0;7]$ соответственно. Для случайной величины $Z=\frac{Y}{X}$ найдите: 
1) функцию распределения $F_Z(x)$;
2) плотность распределения $f_Z(x)$ и постройте график плотности;
3) вероятность $\P(0,\!035\leqslant Z\leqslant 2,\!775)$.




%\folder 2_53d12.png
1) Функция распределения $F_Z(x)$ имеет вид:
$
F_Z(x)=\left\{
\begin{array}{l}
0, x\leqslant 0;\\
\frac{2 x}{7}, 0\leqslant x\leqslant \frac{7}{4}\approx 1,\!75;\\
1 - \frac{7}{8 x}, x\geqslant\frac{7}{4};
\end{array}.
\right.
$
2) Плотность распределения $f_Z(x)$ имеет вид:
$
f_Z(x)=\left\{
\begin{array}{l}
0, x<0;\\
\frac{2}{7}, 0\leqslant x\leqslant \frac{7}{4}\approx 1,\!75;\\
\frac{7}{8 x^{2}}, x\geqslant\frac{7}{4};
\end{array}.
\right.
$


\begin{figure}[H]
    \includegraphics[width=0.9\textwidth]{2_53d12}
\end{figure}


3) вероятность равна:
$
\P(0,\!035\leqslant Z\leqslant 2,\!775)=
0,\!67474.
$


\item

    
	Случайная величина Y принимает только значения из множества $\{2, 1\}$, при этом $P(Y=2) = 0.61$.
	Распределение случайной величины X определено следующим образом:
	\begin{equation*}
		X | Y =
		\begin{cases}
			$8$ * y, с вероятностью $ 0.15$ \\
			$6$ * y, с вероятностью $ 1 - 0.15$
		\end{cases}
	\end{equation*}

	Юный аналитик Дарья нашла матожидание и дисперсию $X$.

	Помогите Дарье найти матожидание и дисперсию величины $X$
	


	

	Первым этапом надо найти характеристики случайной величины $Y$

	$E(Y) = 2 * 0.61 + 1 * (1 - 0.61)$

	$Var(Y) = E(Y^2) - [E(Y)]^2 = 2^2 * 0.61 + 1^2 * (1 - 0.61) - [E(Y)]^2$


	Перейдем к рассмотрению характеристик условной случайно величины X

	$E(X) = E(E(X|Y)) = E[E(8 * Y) * 0.15 + E(6 * Y) * (1 - 0.15)] = E(Y) * (8 * 0.15 + 6 * (1 - 0.15)) = 10.143$

	$E(Var(X|Y)) = E[b * Var(c3 * Y) + (1 - b) * Var(c4 * Y)] = Var(Y) * (c3^2 * b + c4^2 * (1- b)) $

	$Var(E(X|Y)) = E(X^2|Y) - [E(X)]^2 = [E(Y)]^2 * (b * c3^2 + (1-b)*c4^2) - E(X)]^2$

	$Var(X) = E(Var(X|Y)) + Var(E(X|Y)) = 10.88555$
	

\item


(10) В группе $\Omega$ учатся студенты:$\omega _{1}...\omega _{25}$ . Пусть $X$ и $Y$ – 100-балльные экзаменационные оценки по
математическому анализу и теории вероятностей. Оценки $\omega _{i}$ студента обозначаются: $x _{i} = X(\omega _{i})$ и $y _{i} = Y(\omega _{i})$, $i = 1...25$. Все оценки известны
$x _{0} = 40, y _{0} = 84$, $x _{1} = 83, y _{1} = 71$, $x _{2} = 85, y _{2} = 64$, $x _{3} = 77, y _{3} = 32$, $x _{4} = 86, y _{4} = 59$, $x _{5} = 99, y _{5} = 77$, $x _{6} = 91, y _{6} = 74$, $x _{7} = 46, y _{7} = 48$, $x _{8} = 73, y _{8} = 42$, $x _{9} = 82, y _{9} = 89$, $x _{10} = 40, y _{10} = 43$, $x _{11} = 60, y _{11} = 31$, $x _{12} = 81, y _{12} = 57$, $x _{13} = 88, y _{13} = 50$, $x _{14} = 34, y _{14} = 31$, $x _{15} = 45, y _{15} = 63$, $x _{16} = 38, y _{16} = 45$, $x _{17} = 34, y _{17} = 92$, $x _{18} = 92, y _{18} = 83$, $x _{19} = 88, y _{19} = 56$, $x _{20} = 60, y _{20} = 36$, $x _{21} = 85, y _{21} = 59$, $x _{22} = 60, y _{22} = 87$, $x _{23} = 30, y _{23} = 53$, $x _{24} = 56, y _{24} = 73$
Требуется
найти следующие условные эмпирические характеристики: 1) ковариацию $X$ и $Y$ при условии, что одновременно $X \geqslant 50$
 и $Y \geqslant 50$; 2) коэффициент корреляции $X$ и $Y$ при том же условии.




1) Ковариация = $-335.0$
2) Коэффициент корреляции = $-2.4919$


\item


(10) Эмпирическое распределение признаков $X$ и $Y$ на генеральной совокупности $\Omega$ задано таблицей частот  
 
\begin{tabular}{ | c | c | c | c | }
\hline
 & $Y = 2$ & $Y = 4$ & $Y = 5$  \\ \hline
$X = 200$ & $28$ & $13$ & $10$\\ \hline
$X = 300$ & $1$ & $12$ & $35$\\
\hline
\end{tabular}

Из $\Omega$ случайным образом без возвращения извлекаются $7$ элементов. 
Пусть $\bar X$ и $\bar Y$ – средние значения признаков на выбранных элементах. 
Требуется найти: 1) математическое ожидание $\mathbb{E}(\bar Y)$; 2) стандартное отклонение $\sigma(\bar X)$ ; 
3) ковариацию $Cov(\bar X, \bar Y)$




1) математическое ожидание $\mathbb{E}(\bar Y)$: $3.85$ 
2) стандартное отклонение $\sigma(\bar X)$: $244.0153$
3) ковариацию $Cov(\bar X, \bar Y)$: $3.7764$


\item


(10) Пусть $X _{1}$, $X _{2}$, $X _{3}$, $X _{4}$ выборка из $N(\theta, \sigma ^{2})$. Рассмотрим две оценки параметра $\theta$:
\[\hat \theta _{1} = \frac{2X _{1} + 6X _{2} + X _{3} + X _{4}}{10}, \hat \theta _{1} = \frac{5X _{1} + X _{2} + X _{3} + 3X _{4}}{10}\]
a) Покажите, что обе оценки несмещенные.
б) Какая из оценок оптимальная?




Обе они несмещенные, потому что в числителе выходит в сумме 10.
Какая-то точно должна быть, а может и нет....



\end{enumerate}

\section{Билет 121}

\begin{enumerate}


\item


Дайте определение случайной величины, которая имеет $\chi ^{2}$-распределение с n степенями свободы.
Запишите плотность $\chi ^{2}$- распределения. Выведите формулы для математического ожидания $\mathbb{E}(X)$ и дисперсии $\mathbb{V}ar(X)$ $\chi ^{2}$-распределение с n степенями свободы. Найдите а) $\mathbb{P}(\chi _{20}^{2} > 10.9)$, где $\chi _{20}^{2}$–случайная величина, которая имеет $\chi ^{2}$– распределение с 20 степенями свободы; б) найдите 93\%
(верхнюю) точку $\chi _{0.93}^{2} (5)$ хи-квадрат распределения с 5 степенями свободы




$\mathbb{P}(\chi _{20}^{2} > 10.9) =  0.948775$; $\chi _{0.93}^{2} (5) = 1.34721$.


\item



Случайные величины $X$ и $Y$ независимы и имеют равномерное
распределение на отрезках $[0;3]$ и $[0;10]$ соответственно. Для случайной величины $Z=\frac{Y}{X}$ найдите: 
1) функцию распределения $F_Z(x)$;
2) плотность распределения $f_Z(x)$ и постройте график плотности;
3) вероятность $\P(3,\!263\leqslant Z\leqslant 5,\!35)$.




%\folder 2_53d23.png
1) Функция распределения $F_Z(x)$ имеет вид:
$
F_Z(x)=\left\{
\begin{array}{l}
0, x\leqslant 0;\\
\frac{3 x}{20}, 0\leqslant x\leqslant \frac{10}{3}\approx 3,\!333;\\
1 - \frac{5}{3 x}, x\geqslant\frac{10}{3};
\end{array}.
\right.
$
2) Плотность распределения $f_Z(x)$ имеет вид:
$
f_Z(x)=\left\{
\begin{array}{l}
0, x<0;\\
\frac{3}{20}, 0\leqslant x\leqslant \frac{10}{3}\approx 3,\!333;\\
\frac{5}{3 x^{2}}, x\geqslant\frac{10}{3};
\end{array}.
\right.
$


\begin{figure}[H]
    \includegraphics[width=0.9\textwidth]{2_53d23}
\end{figure}


3) вероятность равна:
$
\P(3,\!263\leqslant Z\leqslant 5,\!35)=
0,\!19897.
$


\item


(10) Известно, что доля возвратов по кредитам в банке имеет распределение $F(x) = x ^{\beta}, 0 \leqslant x \leqslant 1$.
Наблюдения показали, что в среднем она составляет $87,5\%$. Методом моментов оцените параметр $\beta$ и
вероятность того, что она опуститься ниже $53\%$




Найдём плотность рапределения как интеграл от ФР, а дальше всё и вовсе простою Ответ: $1174711139837$


\item


(10) В группе $\Omega$ учатся студенты:$\omega _{1}...\omega _{25}$ . Пусть $X$ и $Y$ – 100-балльные экзаменационные оценки по
математическому анализу и теории вероятностей. Оценки $\omega _{i}$ студента обозначаются: $x _{i} = X(\omega _{i})$ и $y _{i} = Y(\omega _{i})$, $i = 1...25$. Все оценки известны
$x _{0} = 33, y _{0} = 72$, $x _{1} = 94, y _{1} = 94$, $x _{2} = 91, y _{2} = 52$, $x _{3} = 47, y _{3} = 59$, $x _{4} = 53, y _{4} = 45$, $x _{5} = 96, y _{5} = 54$, $x _{6} = 60, y _{6} = 99$, $x _{7} = 70, y _{7} = 44$, $x _{8} = 50, y _{8} = 81$, $x _{9} = 57, y _{9} = 40$, $x _{10} = 99, y _{10} = 61$, $x _{11} = 94, y _{11} = 43$, $x _{12} = 85, y _{12} = 96$, $x _{13} = 30, y _{13} = 91$, $x _{14} = 57, y _{14} = 37$, $x _{15} = 42, y _{15} = 35$, $x _{16} = 84, y _{16} = 75$, $x _{17} = 96, y _{17} = 97$, $x _{18} = 69, y _{18} = 92$, $x _{19} = 91, y _{19} = 93$, $x _{20} = 45, y _{20} = 30$, $x _{21} = 35, y _{21} = 94$, $x _{22} = 83, y _{22} = 53$, $x _{23} = 53, y _{23} = 60$, $x _{24} = 36, y _{24} = 69$
Требуется
найти следующие условные эмпирические характеристики: 1) ковариацию $X$ и $Y$ при условии, что одновременно $X \geqslant 50$
 и $Y \geqslant 50$; 2) коэффициент корреляции $X$ и $Y$ при том же условии.




1) Ковариация = $-350.8333$
2) Коэффициент корреляции = $-1.2925$


\item


(10) Эмпирическое распределение признаков $X$ и $Y$ на генеральной совокупности $\Omega$ задано таблицей частот  
 
\begin{tabular}{ | c | c | c | c | }
\hline
 & $Y = 2$ & $Y = 4$ & $Y = 5$  \\ \hline
$X = 200$ & $1$ & $6$ & $23$\\ \hline
$X = 300$ & $13$ & $30$ & $27$\\
\hline
\end{tabular}

Из $\Omega$ случайным образом без возвращения извлекаются $13$ элементов. 
Пусть $\bar X$ и $\bar Y$ – средние значения признаков на выбранных элементах. 
Требуется найти: 1) математическое ожидание $\mathbb{E}(\bar Y)$; 2) стандартное отклонение $\sigma(\bar X)$ ; 
3) ковариацию $Cov(\bar X, \bar Y)$




1) математическое ожидание $\mathbb{E}(\bar Y)$: $4.22$ 
2) стандартное отклонение $\sigma(\bar X)$: $255.4769$
3) ковариацию $Cov(\bar X, \bar Y)$: $-1.2655$


\item

    
    	Юный аналитик Дарья использовала метод Монте-Карло для исследования Дискретного случайного вектора, описанного ниже.

        \begin{tabular}{|c|c|c|c|}
	\hline
	& X=$-3$ & X=$-2$ & X=$-1$ \\
	\hline
	Y = $2$ & $0.29$ & $0.298$  &  $0.234$ \\
	\hline
	Y = $3$ & $0.066$ & $0.03$ & $0.082$  \\
	\hline
\end{tabular}

    	Дарья получила, что E(Y|X + Y = 1) = $2.10982$.
    	Проверьте, можно ли доверять результату Дарьи аналитически. Сформулируйте определение метода Монте-Карло.
    


    
        $E(Y|X+Y=1) = \frac{\sum(P(X=1 - y_i, y=y_i) * y_i)}{\sum(P(X=1 - y_i, y=y_i)}$.

        Ответ: $2.10982$
    


\end{enumerate}

\section{Билет 122}

\begin{enumerate}


\item


Сформулируйте определение случайной выборки из конечной генеральной совокупности. Какие
виды выборок вам известны? Перечислите (с указанием формул) основные характеристики выборочной и генеральной совокупностей




Здесь очень много исчерпывающей информации о выборках из генеральной совокупности и про различные виды выборок


\item



Случайные величины $X$ и $Y$ независимы и имеют равномерное
распределение на отрезках $[0;2]$ и $[0;8]$ соответственно. Для случайной величины $Z=\frac{Y}{X}$ найдите: 
1) функцию распределения $F_Z(x)$;
2) плотность распределения $f_Z(x)$ и постройте график плотности;
3) вероятность $\P(2,\!016\leqslant Z\leqslant 6,\!716)$.




%\folder 2_53d24.png
1) Функция распределения $F_Z(x)$ имеет вид:
$
F_Z(x)=\left\{
\begin{array}{l}
0, x\leqslant 0;\\
\frac{x}{8}, 0\leqslant x\leqslant 4\approx 4,\!0;\\
1 - \frac{2}{x}, x\geqslant4;
\end{array}.
\right.
$
2) Плотность распределения $f_Z(x)$ имеет вид:
$
f_Z(x)=\left\{
\begin{array}{l}
0, x<0;\\
\frac{1}{8}, 0\leqslant x\leqslant 4\approx 4,\!0;\\
\frac{2}{x^{2}}, x\geqslant4;
\end{array}.
\right.
$


\begin{figure}[H]
    \includegraphics[width=0.9\textwidth]{2_53d24}
\end{figure}


3) вероятность равна:
$
\P(2,\!016\leqslant Z\leqslant 6,\!716)=
0,\!4502.
$


\item

    
	Случайная величина Y принимает только значения из множества $\{2, 1\}$, при этом $P(Y=2) = 0.61$.
	Распределение случайной величины X определено следующим образом:
	\begin{equation*}
		X | Y =
		\begin{cases}
			$8$ * y, с вероятностью $ 0.15$ \\
			$6$ * y, с вероятностью $ 1 - 0.15$
		\end{cases}
	\end{equation*}

	Юный аналитик Дарья нашла матожидание и дисперсию $X$.

	Помогите Дарье найти матожидание и дисперсию величины $X$
	


	

	Первым этапом надо найти характеристики случайной величины $Y$

	$E(Y) = 2 * 0.61 + 1 * (1 - 0.61)$

	$Var(Y) = E(Y^2) - [E(Y)]^2 = 2^2 * 0.61 + 1^2 * (1 - 0.61) - [E(Y)]^2$


	Перейдем к рассмотрению характеристик условной случайно величины X

	$E(X) = E(E(X|Y)) = E[E(8 * Y) * 0.15 + E(6 * Y) * (1 - 0.15)] = E(Y) * (8 * 0.15 + 6 * (1 - 0.15)) = 10.143$

	$E(Var(X|Y)) = E[b * Var(c3 * Y) + (1 - b) * Var(c4 * Y)] = Var(Y) * (c3^2 * b + c4^2 * (1- b)) $

	$Var(E(X|Y)) = E(X^2|Y) - [E(X)]^2 = [E(Y)]^2 * (b * c3^2 + (1-b)*c4^2) - E(X)]^2$

	$Var(X) = E(Var(X|Y)) + Var(E(X|Y)) = 10.88555$
	

\item

    
    Создайте эмперические совокупности  $\mathtt{\text{sin}}$ и $\mathtt{\text{cos}}$ вида $\mathtt{\text{sin}}(1),\mathtt{\text{sin}}(2), ..., \mathtt{\text{sin}}(60) $ и $\mathtt{\text{cos}}(1),\mathtt{\text{cos}}(2), ..., \mathtt{\text{cos}}(60). $

    Найдите эмпирическое среднее и эмпирическое стандартное отклонение совокупности $\mathtt{\text{sin}}$, её четвёртый эмпирический центральный момент и эмпирический эксцесс.

    Кроме того, найдите эмпирический коэффициент корреляции признаков $\mathtt{\text{sin}}$ и $\mathtt{\text{cos}}$ на совокупности натуральных чисел от $1$ до $60$.
    


    
    Используя

	$E(X) = sum(X) / n$

	$Var(X) = E(X^2) - [E(X)]^2$

	$\mu_4(X) = E((X-E(X))^4)$

	$Ex = \frac{\mu_4(X)}{[\sigma(X)]^4} - 3$

	$r_{xy} = \frac{E(XY) - E(X) * E(Y)}{\sigma(X) * \sigma(Y)}$

    рассчитаем искомые значения.

    Ответы: $0.02724, 0.70603, 0.37291, -1.49926, 0.00012$.

    

\item


(10) Эмпирическое распределение признаков $X$ и $Y$ на генеральной совокупности $\Omega$ задано таблицей частот  
 
\begin{tabular}{ | c | c | c | c | }
\hline
 & $Y = 2$ & $Y = 4$ & $Y = 5$  \\ \hline
$X = 200$ & $11$ & $26$ & $27$\\ \hline
$X = 300$ & $5$ & $10$ & $21$\\
\hline
\end{tabular}

Из $\Omega$ случайным образом без возвращения извлекаются $6$ элементов. 
Пусть $\bar X$ и $\bar Y$ – средние значения признаков на выбранных элементах. 
Требуется найти: 1) математическое ожидание $\mathbb{E}(\bar Y)$; 2) стандартное отклонение $\sigma(\bar X)$ ; 
3) ковариацию $Cov(\bar X, \bar Y)$




1) математическое ожидание $\mathbb{E}(\bar Y)$: $4.16$ 
2) стандартное отклонение $\sigma(\bar X)$: $233.542$
3) ковариацию $Cov(\bar X, \bar Y)$: $0.4975$


\item

    
	Известно, что доля возвратов по кредитам в банке имеет распределение $F(x) = x^{\beta}, 0 \le x \le 1$. Наблюдения показали, что в среднем она составила $62.0$\%. Методом моментов оцените параметр $\beta$ и вероятность того, что она опуститься ниже $59.0$\%.
	


	

	$f(x) = F'(x) = \beta \cdot x^{\beta - 1}$

	$\mu_{1} = E(X) = \int_{-\inf}^{\inf}x \cdot f(x) = \int_{-\inf}^{\inf} \beta \cdot x^{\beta} = \beta \cdot \frac{x^{\beta + 1}}{\beta + 1}\bigg|_0^1 = \frac{\beta}{\beta + 1}$

	$\beta = (\beta + 1) \cdot 62.0$

	$\beta = \frac{62.0}{1 - 62.0}$

	$ P(x \le 59.0) = F(59.0) = 59.0^{1.63} $

    Ответ: $1.63, 0.42$
	


\end{enumerate}

\section{Билет 123}

\begin{enumerate}


\item

Дайте определение случайной величины, которая имеет гамма-распределение $\Gamma(\alpha,  \lambda)$, и выведите основные свойства гамма-расределения. Запишите формулы для математичсекого ожидания
$\mathbb{E}(X)$ и дисперсии $\mathbb{V}ar(X)$ гамма-распределения




Здесь написанно много всего интересного и полезного о гамма-распределении


\item



Случайные величины $X$ и $Y$ независимы и имеют равномерное
распределение на отрезках $[0;9]$ и $[0;3]$ соответственно. Для случайной величины $Z=\frac{Y}{X}$ найдите: 
1) функцию распределения $F_Z(x)$;
2) плотность распределения $f_Z(x)$ и постройте график плотности;
3) вероятность $\P(0,\!059\leqslant Z\leqslant 0,\!348)$.




%\folder 2_53d19.png
1) Функция распределения $F_Z(x)$ имеет вид:
$
F_Z(x)=\left\{
\begin{array}{l}
0, x\leqslant 0;\\
\frac{3 x}{2}, 0\leqslant x\leqslant \frac{1}{3}\approx 0,\!333;\\
1 - \frac{1}{6 x}, x\geqslant\frac{1}{3};
\end{array}.
\right.
$
2) Плотность распределения $f_Z(x)$ имеет вид:
$
f_Z(x)=\left\{
\begin{array}{l}
0, x<0;\\
\frac{3}{2}, 0\leqslant x\leqslant \frac{1}{3}\approx 0,\!333;\\
\frac{1}{6 x^{2}}, x\geqslant\frac{1}{3};
\end{array}.
\right.
$


\begin{figure}[H]
    \includegraphics[width=0.9\textwidth]{2_53d19}
\end{figure}


3) вероятность равна:
$
\P(0,\!059\leqslant Z\leqslant 0,\!348)=
0,\!43307.
$


\item


(10) Известно, что доля возвратов по кредитам в банке имеет распределение $F(x) = x ^{\beta}, 0 \leqslant x \leqslant 1$.
Наблюдения показали, что в среднем она составляет $85,7143\%$. Методом моментов оцените параметр $\beta$ и
вероятность того, что она опуститься ниже $96\%$




Найдём плотность рапределения как интеграл от ФР, а дальше всё и вовсе простою Ответ: $782757789696$


\item


(10) В группе $\Omega$ учатся студенты:$\omega _{1}...\omega _{25}$ . Пусть $X$ и $Y$ – 100-балльные экзаменационные оценки по
математическому анализу и теории вероятностей. Оценки $\omega _{i}$ студента обозначаются: $x _{i} = X(\omega _{i})$ и $y _{i} = Y(\omega _{i})$, $i = 1...25$. Все оценки известны
$x _{0} = 55, y _{0} = 55$, $x _{1} = 88, y _{1} = 86$, $x _{2} = 42, y _{2} = 96$, $x _{3} = 69, y _{3} = 93$, $x _{4} = 43, y _{4} = 64$, $x _{5} = 42, y _{5} = 86$, $x _{6} = 35, y _{6} = 45$, $x _{7} = 60, y _{7} = 55$, $x _{8} = 41, y _{8} = 90$, $x _{9} = 62, y _{9} = 57$, $x _{10} = 52, y _{10} = 53$, $x _{11} = 67, y _{11} = 32$, $x _{12} = 72, y _{12} = 98$, $x _{13} = 42, y _{13} = 84$, $x _{14} = 97, y _{14} = 51$, $x _{15} = 32, y _{15} = 89$, $x _{16} = 38, y _{16} = 84$, $x _{17} = 42, y _{17} = 84$, $x _{18} = 61, y _{18} = 94$, $x _{19} = 96, y _{19} = 31$, $x _{20} = 67, y _{20} = 56$, $x _{21} = 66, y _{21} = 67$, $x _{22} = 41, y _{22} = 95$, $x _{23} = 54, y _{23} = 95$, $x _{24} = 36, y _{24} = 80$
Требуется
найти следующие условные эмпирические характеристики: 1) ковариацию $X$ и $Y$ при условии, что одновременно $X \geqslant 50$
 и $Y \geqslant 50$; 2) коэффициент корреляции $X$ и $Y$ при том же условии.




1) Ковариация = $92.6667$
2) Коэффициент корреляции = $0.3814$


\item


(10) Эмпирическое распределение признаков $X$ и $Y$ на генеральной совокупности $\Omega$ задано таблицей частот  
 
\begin{tabular}{ | c | c | c | c | }
\hline
 & $Y = 2$ & $Y = 4$ & $Y = 5$  \\ \hline
$X = 200$ & $28$ & $23$ & $3$\\ \hline
$X = 300$ & $2$ & $12$ & $32$\\
\hline
\end{tabular}

Из $\Omega$ случайным образом без возвращения извлекаются $5$ элементов. 
Пусть $\bar X$ и $\bar Y$ – средние значения признаков на выбранных элементах. 
Требуется найти: 1) математическое ожидание $\mathbb{E}(\bar Y)$; 2) стандартное отклонение $\sigma(\bar X)$ ; 
3) ковариацию $Cov(\bar X, \bar Y)$




1) математическое ожидание $\mathbb{E}(\bar Y)$: $3.75$ 
2) стандартное отклонение $\sigma(\bar X)$: $244.6913$
3) ковариацию $Cov(\bar X, \bar Y)$: $3.7904$


\item

    
    	Юный аналитик Дарья использовала метод Монте-Карло для исследования Дискретного случайного вектора, описанного ниже.

        \begin{tabular}{|c|c|c|c|}
	\hline
	& X=$-6$ & X=$-5$ & X=$-4$ \\
	\hline
	Y = $5$ & $0.039$ & $0.207$  &  $0.054$ \\
	\hline
	Y = $6$ & $0.035$ & $0.255$ & $0.41$  \\
	\hline
\end{tabular}

    	Дарья получила, что E(Y|X + Y = 1) = $5.82286$.
    	Проверьте, можно ли доверять результату Дарьи аналитически. Сформулируйте определение метода Монте-Карло.
    


    
        $E(Y|X+Y=1) = \frac{\sum(P(X=1 - y_i, y=y_i) * y_i)}{\sum(P(X=1 - y_i, y=y_i)}$.

        Ответ: $5.82286$
    


\end{enumerate}

\section{Билет 124}

\begin{enumerate}


\item


Дайте определение случайной величины, которая имеет $\chi ^{2}$-распределение с n степенями свободы.
Запишите плотность $\chi ^{2}$- распределения. Выведите формулы для математического ожидания $\mathbb{E}(X)$ и дисперсии $\mathbb{V}ar(X)$ $\chi ^{2}$-распределение с n степенями свободы. Найдите а) $\mathbb{P}(\chi _{20}^{2} > 10.9)$, где $\chi _{20}^{2}$–случайная величина, которая имеет $\chi ^{2}$– распределение с 20 степенями свободы; б) найдите 93\%
(верхнюю) точку $\chi _{0.93}^{2} (5)$ хи-квадрат распределения с 5 степенями свободы




$\mathbb{P}(\chi _{20}^{2} > 10.9) =  0.948775$; $\chi _{0.93}^{2} (5) = 1.34721$.


\item



Случайные величины $X$ и $Y$ независимы и имеют равномерное
распределение на отрезках $[0;10]$ и $[0;9]$ соответственно. Для случайной величины $Z=\frac{Y}{X}$ найдите: 
1) функцию распределения $F_Z(x)$;
2) плотность распределения $f_Z(x)$ и постройте график плотности;
3) вероятность $\P(0,\!719\leqslant Z\leqslant 1,\!005)$.




%\folder 2_53d14.png
1) Функция распределения $F_Z(x)$ имеет вид:
$
F_Z(x)=\left\{
\begin{array}{l}
0, x\leqslant 0;\\
\frac{5 x}{9}, 0\leqslant x\leqslant \frac{9}{10}\approx 0,\!9;\\
1 - \frac{9}{20 x}, x\geqslant\frac{9}{10};
\end{array}.
\right.
$
2) Плотность распределения $f_Z(x)$ имеет вид:
$
f_Z(x)=\left\{
\begin{array}{l}
0, x<0;\\
\frac{5}{9}, 0\leqslant x\leqslant \frac{9}{10}\approx 0,\!9;\\
\frac{9}{20 x^{2}}, x\geqslant\frac{9}{10};
\end{array}.
\right.
$


\begin{figure}[H]
    \includegraphics[width=0.9\textwidth]{2_53d14}
\end{figure}


3) вероятность равна:
$
\P(0,\!719\leqslant Z\leqslant 1,\!005)=
0,\!15287.
$


\item

    
	Случайная величина Y принимает только значения из множества $\{2, 1\}$, при этом $P(Y=2) = 0.61$.
	Распределение случайной величины X определено следующим образом:
	\begin{equation*}
		X | Y =
		\begin{cases}
			$8$ * y, с вероятностью $ 0.15$ \\
			$6$ * y, с вероятностью $ 1 - 0.15$
		\end{cases}
	\end{equation*}

	Юный аналитик Дарья нашла матожидание и дисперсию $X$.

	Помогите Дарье найти матожидание и дисперсию величины $X$
	


	

	Первым этапом надо найти характеристики случайной величины $Y$

	$E(Y) = 2 * 0.61 + 1 * (1 - 0.61)$

	$Var(Y) = E(Y^2) - [E(Y)]^2 = 2^2 * 0.61 + 1^2 * (1 - 0.61) - [E(Y)]^2$


	Перейдем к рассмотрению характеристик условной случайно величины X

	$E(X) = E(E(X|Y)) = E[E(8 * Y) * 0.15 + E(6 * Y) * (1 - 0.15)] = E(Y) * (8 * 0.15 + 6 * (1 - 0.15)) = 10.143$

	$E(Var(X|Y)) = E[b * Var(c3 * Y) + (1 - b) * Var(c4 * Y)] = Var(Y) * (c3^2 * b + c4^2 * (1- b)) $

	$Var(E(X|Y)) = E(X^2|Y) - [E(X)]^2 = [E(Y)]^2 * (b * c3^2 + (1-b)*c4^2) - E(X)]^2$

	$Var(X) = E(Var(X|Y)) + Var(E(X|Y)) = 10.88555$
	

\item


(10) В группе $\Omega$ учатся студенты:$\omega _{1}...\omega _{25}$ . Пусть $X$ и $Y$ – 100-балльные экзаменационные оценки по
математическому анализу и теории вероятностей. Оценки $\omega _{i}$ студента обозначаются: $x _{i} = X(\omega _{i})$ и $y _{i} = Y(\omega _{i})$, $i = 1...25$. Все оценки известны
$x _{0} = 33, y _{0} = 72$, $x _{1} = 94, y _{1} = 94$, $x _{2} = 91, y _{2} = 52$, $x _{3} = 47, y _{3} = 59$, $x _{4} = 53, y _{4} = 45$, $x _{5} = 96, y _{5} = 54$, $x _{6} = 60, y _{6} = 99$, $x _{7} = 70, y _{7} = 44$, $x _{8} = 50, y _{8} = 81$, $x _{9} = 57, y _{9} = 40$, $x _{10} = 99, y _{10} = 61$, $x _{11} = 94, y _{11} = 43$, $x _{12} = 85, y _{12} = 96$, $x _{13} = 30, y _{13} = 91$, $x _{14} = 57, y _{14} = 37$, $x _{15} = 42, y _{15} = 35$, $x _{16} = 84, y _{16} = 75$, $x _{17} = 96, y _{17} = 97$, $x _{18} = 69, y _{18} = 92$, $x _{19} = 91, y _{19} = 93$, $x _{20} = 45, y _{20} = 30$, $x _{21} = 35, y _{21} = 94$, $x _{22} = 83, y _{22} = 53$, $x _{23} = 53, y _{23} = 60$, $x _{24} = 36, y _{24} = 69$
Требуется
найти следующие условные эмпирические характеристики: 1) ковариацию $X$ и $Y$ при условии, что одновременно $X \geqslant 50$
 и $Y \geqslant 50$; 2) коэффициент корреляции $X$ и $Y$ при том же условии.




1) Ковариация = $-350.8333$
2) Коэффициент корреляции = $-1.2925$


\item


(10) Эмпирическое распределение признаков $X$ и $Y$ на генеральной совокупности $\Omega$ задано таблицей частот  
 
\begin{tabular}{ | c | c | c | c | }
\hline
 & $Y = 2$ & $Y = 4$ & $Y = 5$  \\ \hline
$X = 200$ & $16$ & $16$ & $22$\\ \hline
$X = 300$ & $7$ & $26$ & $13$\\
\hline
\end{tabular}

Из $\Omega$ случайным образом без возвращения извлекаются $9$ элементов. 
Пусть $\bar X$ и $\bar Y$ – средние значения признаков на выбранных элементах. 
Требуется найти: 1) математическое ожидание $\mathbb{E}(\bar Y)$; 2) стандартное отклонение $\sigma(\bar X)$ ; 
3) ковариацию $Cov(\bar X, \bar Y)$




1) математическое ожидание $\mathbb{E}(\bar Y)$: $3.89$ 
2) стандартное отклонение $\sigma(\bar X)$: $239.4845$
3) ковариацию $Cov(\bar X, \bar Y)$: $0.3732$


\item

    
	Известно, что доля возвратов по кредитам в банке имеет распределение $F(x) = x^{\beta}, 0 \le x \le 1$. Наблюдения показали, что в среднем она составила $62.0$\%. Методом моментов оцените параметр $\beta$ и вероятность того, что она опуститься ниже $59.0$\%.
	


	

	$f(x) = F'(x) = \beta \cdot x^{\beta - 1}$

	$\mu_{1} = E(X) = \int_{-\inf}^{\inf}x \cdot f(x) = \int_{-\inf}^{\inf} \beta \cdot x^{\beta} = \beta \cdot \frac{x^{\beta + 1}}{\beta + 1}\bigg|_0^1 = \frac{\beta}{\beta + 1}$

	$\beta = (\beta + 1) \cdot 62.0$

	$\beta = \frac{62.0}{1 - 62.0}$

	$ P(x \le 59.0) = F(59.0) = 59.0^{1.63} $

    Ответ: $1.63, 0.42$
	


\end{enumerate}

\section{Билет 125}

\begin{enumerate}


\item


Сформулируйте определение случайной выборки из конечной генеральной совокупности. Какие
виды выборок вам известны? Перечислите (с указанием формул) основные характеристики выборочной и генеральной совокупностей




Здесь очень много исчерпывающей информации о выборках из генеральной совокупности и про различные виды выборок


\item



Случайные величины $X$ и $Y$ независимы и имеют равномерное
распределение на отрезках $[0;7]$ и $[0;3]$ соответственно. Для случайной величины $Z=\frac{Y}{X}$ найдите: 
1) функцию распределения $F_Z(x)$;
2) плотность распределения $f_Z(x)$ и постройте график плотности;
3) вероятность $\P(0,\!006\leqslant Z\leqslant 0,\!519)$.




%\folder 2_53d18.png
1) Функция распределения $F_Z(x)$ имеет вид:
$
F_Z(x)=\left\{
\begin{array}{l}
0, x\leqslant 0;\\
\frac{7 x}{6}, 0\leqslant x\leqslant \frac{3}{7}\approx 0,\!429;\\
1 - \frac{3}{14 x}, x\geqslant\frac{3}{7};
\end{array}.
\right.
$
2) Плотность распределения $f_Z(x)$ имеет вид:
$
f_Z(x)=\left\{
\begin{array}{l}
0, x<0;\\
\frac{7}{6}, 0\leqslant x\leqslant \frac{3}{7}\approx 0,\!429;\\
\frac{3}{14 x^{2}}, x\geqslant\frac{3}{7};
\end{array}.
\right.
$


\begin{figure}[H]
    \includegraphics[width=0.9\textwidth]{2_53d18}
\end{figure}


3) вероятность равна:
$
\P(0,\!006\leqslant Z\leqslant 0,\!519)=
0,\!57962.
$


\item


(10) Известно, что доля возвратов по кредитам в банке имеет распределение $F(x) = x ^{\beta}, 0 \leqslant x \leqslant 1$.
Наблюдения показали, что в среднем она составляет $87,5\%$. Методом моментов оцените параметр $\beta$ и
вероятность того, что она опуститься ниже $53\%$




Найдём плотность рапределения как интеграл от ФР, а дальше всё и вовсе простою Ответ: $1174711139837$


\item


(10) В группе $\Omega$ учатся студенты:$\omega _{1}...\omega _{25}$ . Пусть $X$ и $Y$ – 100-балльные экзаменационные оценки по
математическому анализу и теории вероятностей. Оценки $\omega _{i}$ студента обозначаются: $x _{i} = X(\omega _{i})$ и $y _{i} = Y(\omega _{i})$, $i = 1...25$. Все оценки известны
$x _{0} = 73, y _{0} = 44$, $x _{1} = 44, y _{1} = 83$, $x _{2} = 49, y _{2} = 41$, $x _{3} = 36, y _{3} = 32$, $x _{4} = 48, y _{4} = 60$, $x _{5} = 53, y _{5} = 37$, $x _{6} = 70, y _{6} = 86$, $x _{7} = 61, y _{7} = 82$, $x _{8} = 42, y _{8} = 57$, $x _{9} = 94, y _{9} = 40$, $x _{10} = 44, y _{10} = 78$, $x _{11} = 85, y _{11} = 78$, $x _{12} = 48, y _{12} = 66$, $x _{13} = 88, y _{13} = 82$, $x _{14} = 31, y _{14} = 39$, $x _{15} = 84, y _{15} = 68$, $x _{16} = 49, y _{16} = 51$, $x _{17} = 84, y _{17} = 55$, $x _{18} = 65, y _{18} = 67$, $x _{19} = 37, y _{19} = 99$, $x _{20} = 46, y _{20} = 31$, $x _{21} = 84, y _{21} = 46$, $x _{22} = 40, y _{22} = 67$, $x _{23} = 86, y _{23} = 54$, $x _{24} = 89, y _{24} = 32$
Требуется
найти следующие условные эмпирические характеристики: 1) ковариацию $X$ и $Y$ при условии, что одновременно $X \geqslant 50$
 и $Y \geqslant 50$; 2) коэффициент корреляции $X$ и $Y$ при том же условии.




1) Ковариация = $-345.5$
2) Коэффициент корреляции = $-2.9554$


\item


(10) Эмпирическое распределение признаков $X$ и $Y$ на генеральной совокупности $\Omega$ задано таблицей частот  
 
\begin{tabular}{ | c | c | c | c | }
\hline
 & $Y = 2$ & $Y = 4$ & $Y = 5$  \\ \hline
$X = 200$ & $16$ & $19$ & $5$\\ \hline
$X = 300$ & $25$ & $10$ & $25$\\
\hline
\end{tabular}

Из $\Omega$ случайным образом без возвращения извлекаются $6$ элементов. 
Пусть $\bar X$ и $\bar Y$ – средние значения признаков на выбранных элементах. 
Требуется найти: 1) математическое ожидание $\mathbb{E}(\bar Y)$; 2) стандартное отклонение $\sigma(\bar X)$ ; 
3) ковариацию $Cov(\bar X, \bar Y)$




1) математическое ожидание $\mathbb{E}(\bar Y)$: $3.48$ 
2) стандартное отклонение $\sigma(\bar X)$: $256.5595$
3) ковариацию $Cov(\bar X, \bar Y)$: $0.5887$


\item


(10) Пусть $X _{1}$, $X _{2}$, $X _{3}$, $X _{4}$ выборка из $N(\theta, \sigma ^{2})$. Рассмотрим две оценки параметра $\theta$:
\[\hat \theta _{1} = \frac{3X _{1} + X _{2} + 4X _{3} + 2X _{4}}{10}, \hat \theta _{1} = \frac{X _{1} + 6X _{2} + 2X _{3} + X _{4}}{10}\]
a) Покажите, что обе оценки несмещенные.
б) Какая из оценок оптимальная?




Обе они несмещенные, потому что в числителе выходит в сумме 10.
Какая-то точно должна быть, а может и нет....



\end{enumerate}

\section{Билет 126}

\begin{enumerate}


\item


Дайте определение случайной величины, которая имеет $\chi ^{2}$-распределение с n степенями свободы.
Запишите плотность $\chi ^{2}$- распределения. Выведите формулы для математического ожидания $\mathbb{E}(X)$ и дисперсии $\mathbb{V}ar(X)$ $\chi ^{2}$-распределение с n степенями свободы. Найдите а) $\mathbb{P}(\chi _{20}^{2} > 10.9)$, где $\chi _{20}^{2}$–случайная величина, которая имеет $\chi ^{2}$– распределение с 20 степенями свободы; б) найдите 93\%
(верхнюю) точку $\chi _{0.93}^{2} (5)$ хи-квадрат распределения с 5 степенями свободы




$\mathbb{P}(\chi _{20}^{2} > 10.9) =  0.948775$; $\chi _{0.93}^{2} (5) = 1.34721$.


\item



Случайные величины $X$ и $Y$ независимы и имеют равномерное
распределение на отрезках $[0;3]$ и $[0;10]$ соответственно. Для случайной величины $Z=\frac{Y}{X}$ найдите: 
1) функцию распределения $F_Z(x)$;
2) плотность распределения $f_Z(x)$ и постройте график плотности;
3) вероятность $\P(3,\!263\leqslant Z\leqslant 5,\!35)$.




%\folder 2_53d23.png
1) Функция распределения $F_Z(x)$ имеет вид:
$
F_Z(x)=\left\{
\begin{array}{l}
0, x\leqslant 0;\\
\frac{3 x}{20}, 0\leqslant x\leqslant \frac{10}{3}\approx 3,\!333;\\
1 - \frac{5}{3 x}, x\geqslant\frac{10}{3};
\end{array}.
\right.
$
2) Плотность распределения $f_Z(x)$ имеет вид:
$
f_Z(x)=\left\{
\begin{array}{l}
0, x<0;\\
\frac{3}{20}, 0\leqslant x\leqslant \frac{10}{3}\approx 3,\!333;\\
\frac{5}{3 x^{2}}, x\geqslant\frac{10}{3};
\end{array}.
\right.
$


\begin{figure}[H]
    \includegraphics[width=0.9\textwidth]{2_53d23}
\end{figure}


3) вероятность равна:
$
\P(3,\!263\leqslant Z\leqslant 5,\!35)=
0,\!19897.
$


\item


(10) Известно, что доля возвратов по кредитам в банке имеет распределение $F(x) = x ^{\beta}, 0 \leqslant x \leqslant 1$.
Наблюдения показали, что в среднем она составляет $85,7143\%$. Методом моментов оцените параметр $\beta$ и
вероятность того, что она опуститься ниже $96\%$




Найдём плотность рапределения как интеграл от ФР, а дальше всё и вовсе простою Ответ: $782757789696$


\item


(10) В группе $\Omega$ учатся студенты:$\omega _{1}...\omega _{25}$ . Пусть $X$ и $Y$ – 100-балльные экзаменационные оценки по
математическому анализу и теории вероятностей. Оценки $\omega _{i}$ студента обозначаются: $x _{i} = X(\omega _{i})$ и $y _{i} = Y(\omega _{i})$, $i = 1...25$. Все оценки известны
$x _{0} = 33, y _{0} = 72$, $x _{1} = 94, y _{1} = 94$, $x _{2} = 91, y _{2} = 52$, $x _{3} = 47, y _{3} = 59$, $x _{4} = 53, y _{4} = 45$, $x _{5} = 96, y _{5} = 54$, $x _{6} = 60, y _{6} = 99$, $x _{7} = 70, y _{7} = 44$, $x _{8} = 50, y _{8} = 81$, $x _{9} = 57, y _{9} = 40$, $x _{10} = 99, y _{10} = 61$, $x _{11} = 94, y _{11} = 43$, $x _{12} = 85, y _{12} = 96$, $x _{13} = 30, y _{13} = 91$, $x _{14} = 57, y _{14} = 37$, $x _{15} = 42, y _{15} = 35$, $x _{16} = 84, y _{16} = 75$, $x _{17} = 96, y _{17} = 97$, $x _{18} = 69, y _{18} = 92$, $x _{19} = 91, y _{19} = 93$, $x _{20} = 45, y _{20} = 30$, $x _{21} = 35, y _{21} = 94$, $x _{22} = 83, y _{22} = 53$, $x _{23} = 53, y _{23} = 60$, $x _{24} = 36, y _{24} = 69$
Требуется
найти следующие условные эмпирические характеристики: 1) ковариацию $X$ и $Y$ при условии, что одновременно $X \geqslant 50$
 и $Y \geqslant 50$; 2) коэффициент корреляции $X$ и $Y$ при том же условии.




1) Ковариация = $-350.8333$
2) Коэффициент корреляции = $-1.2925$


\item

    
    	Распределение результатов экзамена в некоторой стране с $10$-балльной системой оценивания задано следующим образом:
    	$\left\{ 1 : 6, \  2 : 16, \  3 : 9, \  4 : 16, \  5 : 14, \  6 : 4, \  7 : 25, \  8 : 26, \  9 : 24, \  10 : 10\right\}$

	Работы будут перепроверять $10$ преподавателей, которые разделили все имеющиеся работы между собой случайным образом. Пусть $\overline{X}$ - средний балл (по перепроверки) работ, попавших к одному преподавателю.

	Требуется найти матожидание и стандартное отклонение среднего балла работ, попавших к одному преподавателю, до перепроверки.
    


    


    k = len(marks) // k

    ex = np.sum([marks[m] * m for m in marks]) / n

    varx = np.var([ m for m in marks for temp in range(marks[m])]) / k * (n - k) / (n - 1)

    sigmax = varx**(0.5)
    Ответы: $6.14667, 0.65542$.

    

\item

    
	Известно, что доля возвратов по кредитам в банке имеет распределение $F(x) = x^{\beta}, 0 \le x \le 1$. Наблюдения показали, что в среднем она составила $57.0$\%. Методом моментов оцените параметр $\beta$ и вероятность того, что она опуститься ниже $51.0$\%.
	


	

	$f(x) = F'(x) = \beta \cdot x^{\beta - 1}$

	$\mu_{1} = E(X) = \int_{-\inf}^{\inf}x \cdot f(x) = \int_{-\inf}^{\inf} \beta \cdot x^{\beta} = \beta \cdot \frac{x^{\beta + 1}}{\beta + 1}\bigg|_0^1 = \frac{\beta}{\beta + 1}$

	$\beta = (\beta + 1) \cdot 57.0$

	$\beta = \frac{57.0}{1 - 57.0}$

	$ P(x \le 51.0) = F(51.0) = 51.0^{1.33} $

    Ответ: $1.33, 0.41$
	


\end{enumerate}

\section{Билет 127}

\begin{enumerate}


\item

Дайте определение случайной величины, которая имеет гамма-распределение $\Gamma(\alpha,  \lambda)$, и выведите основные свойства гамма-расределения. Запишите формулы для математичсекого ожидания
$\mathbb{E}(X)$ и дисперсии $\mathbb{V}ar(X)$ гамма-распределения




Здесь написанно много всего интересного и полезного о гамма-распределении


\item



Случайные величины $X$ и $Y$ независимы и имеют равномерное
распределение на отрезках $[0;1]$ и $[0;3]$ соответственно. Для случайной величины $Z=\frac{Y}{X}$ найдите: 
1) функцию распределения $F_Z(x)$;
2) плотность распределения $f_Z(x)$ и постройте график плотности;
3) вероятность $\P(0,\!039\leqslant Z\leqslant 5,\!208)$.




%\folder 2_53d21.png
1) Функция распределения $F_Z(x)$ имеет вид:
$
F_Z(x)=\left\{
\begin{array}{l}
0, x\leqslant 0;\\
\frac{x}{6}, 0\leqslant x\leqslant 3\approx 3,\!0;\\
1 - \frac{3}{2 x}, x\geqslant3;
\end{array}.
\right.
$
2) Плотность распределения $f_Z(x)$ имеет вид:
$
f_Z(x)=\left\{
\begin{array}{l}
0, x<0;\\
\frac{1}{6}, 0\leqslant x\leqslant 3\approx 3,\!0;\\
\frac{3}{2 x^{2}}, x\geqslant3;
\end{array}.
\right.
$


\begin{figure}[H]
    \includegraphics[width=0.9\textwidth]{2_53d21}
\end{figure}


3) вероятность равна:
$
\P(0,\!039\leqslant Z\leqslant 5,\!208)=
0,\!70548.
$


\item


(10) Известно, что доля возвратов по кредитам в банке имеет распределение $F(x) = x ^{\beta}, 0 \leqslant x \leqslant 1$.
Наблюдения показали, что в среднем она составляет $85,7143\%$. Методом моментов оцените параметр $\beta$ и
вероятность того, что она опуститься ниже $96\%$




Найдём плотность рапределения как интеграл от ФР, а дальше всё и вовсе простою Ответ: $782757789696$


\item


(10) В группе $\Omega$ учатся студенты:$\omega _{1}...\omega _{25}$ . Пусть $X$ и $Y$ – 100-балльные экзаменационные оценки по
математическому анализу и теории вероятностей. Оценки $\omega _{i}$ студента обозначаются: $x _{i} = X(\omega _{i})$ и $y _{i} = Y(\omega _{i})$, $i = 1...25$. Все оценки известны
$x _{0} = 97, y _{0} = 80$, $x _{1} = 45, y _{1} = 92$, $x _{2} = 41, y _{2} = 62$, $x _{3} = 56, y _{3} = 75$, $x _{4} = 88, y _{4} = 53$, $x _{5} = 45, y _{5} = 93$, $x _{6} = 91, y _{6} = 71$, $x _{7} = 31, y _{7} = 62$, $x _{8} = 57, y _{8} = 69$, $x _{9} = 48, y _{9} = 84$, $x _{10} = 33, y _{10} = 82$, $x _{11} = 95, y _{11} = 34$, $x _{12} = 94, y _{12} = 40$, $x _{13} = 58, y _{13} = 78$, $x _{14} = 64, y _{14} = 60$, $x _{15} = 81, y _{15} = 47$, $x _{16} = 57, y _{16} = 55$, $x _{17} = 30, y _{17} = 93$, $x _{18} = 51, y _{18} = 52$, $x _{19} = 99, y _{19} = 88$, $x _{20} = 47, y _{20} = 60$, $x _{21} = 78, y _{21} = 31$, $x _{22} = 61, y _{22} = 37$, $x _{23} = 91, y _{23} = 81$, $x _{24} = 39, y _{24} = 98$
Требуется
найти следующие условные эмпирические характеристики: 1) ковариацию $X$ и $Y$ при условии, что одновременно $X \geqslant 50$
 и $Y \geqslant 50$; 2) коэффициент корреляции $X$ и $Y$ при том же условии.




1) Ковариация = $1210.3636$
2) Коэффициент корреляции = $5.5178$


\item


(10) Эмпирическое распределение признаков $X$ и $Y$ на генеральной совокупности $\Omega$ задано таблицей частот  
 
\begin{tabular}{ | c | c | c | c | }
\hline
 & $Y = 2$ & $Y = 4$ & $Y = 5$  \\ \hline
$X = 200$ & $28$ & $13$ & $10$\\ \hline
$X = 300$ & $1$ & $12$ & $35$\\
\hline
\end{tabular}

Из $\Omega$ случайным образом без возвращения извлекаются $7$ элементов. 
Пусть $\bar X$ и $\bar Y$ – средние значения признаков на выбранных элементах. 
Требуется найти: 1) математическое ожидание $\mathbb{E}(\bar Y)$; 2) стандартное отклонение $\sigma(\bar X)$ ; 
3) ковариацию $Cov(\bar X, \bar Y)$




1) математическое ожидание $\mathbb{E}(\bar Y)$: $3.85$ 
2) стандартное отклонение $\sigma(\bar X)$: $244.0153$
3) ковариацию $Cov(\bar X, \bar Y)$: $3.7764$


\item

    
	Известно, что доля возвратов по кредитам в банке имеет распределение $F(x) = x^{\beta}, 0 \le x \le 1$. Наблюдения показали, что в среднем она составила $74.0$\%. Методом моментов оцените параметр $\beta$ и вероятность того, что она опуститься ниже $73.0$\%.
	


	

	$f(x) = F'(x) = \beta \cdot x^{\beta - 1}$

	$\mu_{1} = E(X) = \int_{-\inf}^{\inf}x \cdot f(x) = \int_{-\inf}^{\inf} \beta \cdot x^{\beta} = \beta \cdot \frac{x^{\beta + 1}}{\beta + 1}\bigg|_0^1 = \frac{\beta}{\beta + 1}$

	$\beta = (\beta + 1) \cdot 74.0$

	$\beta = \frac{74.0}{1 - 74.0}$

	$ P(x \le 73.0) = F(73.0) = 73.0^{2.85} $

    Ответ: $2.85, 0.41$
	


\end{enumerate}

\section{Билет 128}

\begin{enumerate}


\item


Сформулируйте определение случайной выборки из конечной генеральной совокупности. Какие
виды выборок вам известны? Перечислите (с указанием формул) основные характеристики выборочной и генеральной совокупностей




Здесь очень много исчерпывающей информации о выборках из генеральной совокупности и про различные виды выборок


\item



Случайные величины $X$ и $Y$ независимы и имеют равномерное
распределение на отрезках $[0;2]$ и $[0;7]$ соответственно. Для случайной величины $Z=\frac{Y}{X}$ найдите: 
1) функцию распределения $F_Z(x)$;
2) плотность распределения $f_Z(x)$ и постройте график плотности;
3) вероятность $\P(2,\!019\leqslant Z\leqslant 3,\!843)$.




%\folder 2_53d15.png
1) Функция распределения $F_Z(x)$ имеет вид:
$
F_Z(x)=\left\{
\begin{array}{l}
0, x\leqslant 0;\\
\frac{x}{7}, 0\leqslant x\leqslant \frac{7}{2}\approx 3,\!5;\\
1 - \frac{7}{4 x}, x\geqslant\frac{7}{2};
\end{array}.
\right.
$
2) Плотность распределения $f_Z(x)$ имеет вид:
$
f_Z(x)=\left\{
\begin{array}{l}
0, x<0;\\
\frac{1}{7}, 0\leqslant x\leqslant \frac{7}{2}\approx 3,\!5;\\
\frac{7}{4 x^{2}}, x\geqslant\frac{7}{2};
\end{array}.
\right.
$


\begin{figure}[H]
    \includegraphics[width=0.9\textwidth]{2_53d15}
\end{figure}


3) вероятность равна:
$
\P(2,\!019\leqslant Z\leqslant 3,\!843)=
0,\!25613.
$


\item


(10) Известно, что доля возвратов по кредитам в банке имеет распределение $F(x) = x ^{\beta}, 0 \leqslant x \leqslant 1$.
Наблюдения показали, что в среднем она составляет $75,0\%$. Методом моментов оцените параметр $\beta$ и
вероятность того, что она опуститься ниже $20\%$




Найдём плотность рапределения как интеграл от ФР, а дальше всё и вовсе простою Ответ: $8000$


\item


(10) В группе $\Omega$ учатся студенты:$\omega _{1}...\omega _{25}$ . Пусть $X$ и $Y$ – 100-балльные экзаменационные оценки по
математическому анализу и теории вероятностей. Оценки $\omega _{i}$ студента обозначаются: $x _{i} = X(\omega _{i})$ и $y _{i} = Y(\omega _{i})$, $i = 1...25$. Все оценки известны
$x _{0} = 33, y _{0} = 72$, $x _{1} = 94, y _{1} = 94$, $x _{2} = 91, y _{2} = 52$, $x _{3} = 47, y _{3} = 59$, $x _{4} = 53, y _{4} = 45$, $x _{5} = 96, y _{5} = 54$, $x _{6} = 60, y _{6} = 99$, $x _{7} = 70, y _{7} = 44$, $x _{8} = 50, y _{8} = 81$, $x _{9} = 57, y _{9} = 40$, $x _{10} = 99, y _{10} = 61$, $x _{11} = 94, y _{11} = 43$, $x _{12} = 85, y _{12} = 96$, $x _{13} = 30, y _{13} = 91$, $x _{14} = 57, y _{14} = 37$, $x _{15} = 42, y _{15} = 35$, $x _{16} = 84, y _{16} = 75$, $x _{17} = 96, y _{17} = 97$, $x _{18} = 69, y _{18} = 92$, $x _{19} = 91, y _{19} = 93$, $x _{20} = 45, y _{20} = 30$, $x _{21} = 35, y _{21} = 94$, $x _{22} = 83, y _{22} = 53$, $x _{23} = 53, y _{23} = 60$, $x _{24} = 36, y _{24} = 69$
Требуется
найти следующие условные эмпирические характеристики: 1) ковариацию $X$ и $Y$ при условии, что одновременно $X \geqslant 50$
 и $Y \geqslant 50$; 2) коэффициент корреляции $X$ и $Y$ при том же условии.




1) Ковариация = $-350.8333$
2) Коэффициент корреляции = $-1.2925$


\item


(10) Эмпирическое распределение признаков $X$ и $Y$ на генеральной совокупности $\Omega$ задано таблицей частот  
 
\begin{tabular}{ | c | c | c | c | }
\hline
 & $Y = 2$ & $Y = 4$ & $Y = 5$  \\ \hline
$X = 200$ & $24$ & $17$ & $3$\\ \hline
$X = 300$ & $13$ & $24$ & $19$\\
\hline
\end{tabular}

Из $\Omega$ случайным образом без возвращения извлекаются $9$ элементов. 
Пусть $\bar X$ и $\bar Y$ – средние значения признаков на выбранных элементах. 
Требуется найти: 1) математическое ожидание $\mathbb{E}(\bar Y)$; 2) стандартное отклонение $\sigma(\bar X)$ ; 
3) ковариацию $Cov(\bar X, \bar Y)$




1) математическое ожидание $\mathbb{E}(\bar Y)$: $3.48$ 
2) стандартное отклонение $\sigma(\bar X)$: $248.8024$
3) ковариацию $Cov(\bar X, \bar Y)$: $2.0333$


\item

    
	Известно, что доля возвратов по кредитам в банке имеет распределение $F(x) = x^{\beta}, 0 \le x \le 1$. Наблюдения показали, что в среднем она составила $74.0$\%. Методом моментов оцените параметр $\beta$ и вероятность того, что она опуститься ниже $73.0$\%.
	


	

	$f(x) = F'(x) = \beta \cdot x^{\beta - 1}$

	$\mu_{1} = E(X) = \int_{-\inf}^{\inf}x \cdot f(x) = \int_{-\inf}^{\inf} \beta \cdot x^{\beta} = \beta \cdot \frac{x^{\beta + 1}}{\beta + 1}\bigg|_0^1 = \frac{\beta}{\beta + 1}$

	$\beta = (\beta + 1) \cdot 74.0$

	$\beta = \frac{74.0}{1 - 74.0}$

	$ P(x \le 73.0) = F(73.0) = 73.0^{2.85} $

    Ответ: $2.85, 0.41$
	


\end{enumerate}

\section{Билет 129}

\begin{enumerate}


\item

Дайте определение случайной величины, которая имеет гамма-распределение $\Gamma(\alpha,  \lambda)$, и выведите основные свойства гамма-расределения. Запишите формулы для математичсекого ожидания
$\mathbb{E}(X)$ и дисперсии $\mathbb{V}ar(X)$ гамма-распределения




Здесь написанно много всего интересного и полезного о гамма-распределении


\item



Случайные величины $X$ и $Y$ независимы и имеют равномерное
распределение на отрезках $[0;4]$ и $[0;3]$ соответственно. Для случайной величины $Z=\frac{Y}{X}$ найдите: 
1) функцию распределения $F_Z(x)$;
2) плотность распределения $f_Z(x)$ и постройте график плотности;
3) вероятность $\P(0,\!182\leqslant Z\leqslant 1,\!21)$.




%\folder 2_53d11.png
1) Функция распределения $F_Z(x)$ имеет вид:
$
F_Z(x)=\left\{
\begin{array}{l}
0, x\leqslant 0;\\
\frac{2 x}{3}, 0\leqslant x\leqslant \frac{3}{4}\approx 0,\!75;\\
1 - \frac{3}{8 x}, x\geqslant\frac{3}{4};
\end{array}.
\right.
$
2) Плотность распределения $f_Z(x)$ имеет вид:
$
f_Z(x)=\left\{
\begin{array}{l}
0, x<0;\\
\frac{2}{3}, 0\leqslant x\leqslant \frac{3}{4}\approx 0,\!75;\\
\frac{3}{8 x^{2}}, x\geqslant\frac{3}{4};
\end{array}.
\right.
$


\begin{figure}[H]
    \includegraphics[width=0.9\textwidth]{2_53d11}
\end{figure}


3) вероятность равна:
$
\P(0,\!182\leqslant Z\leqslant 1,\!21)=
0,\!56852.
$


\item


(10) Известно, что доля возвратов по кредитам в банке имеет распределение $F(x) = x ^{\beta}, 0 \leqslant x \leqslant 1$.
Наблюдения показали, что в среднем она составляет $85,7143\%$. Методом моментов оцените параметр $\beta$ и
вероятность того, что она опуститься ниже $96\%$




Найдём плотность рапределения как интеграл от ФР, а дальше всё и вовсе простою Ответ: $782757789696$


\item


(10) В группе $\Omega$ учатся студенты:$\omega _{1}...\omega _{25}$ . Пусть $X$ и $Y$ – 100-балльные экзаменационные оценки по
математическому анализу и теории вероятностей. Оценки $\omega _{i}$ студента обозначаются: $x _{i} = X(\omega _{i})$ и $y _{i} = Y(\omega _{i})$, $i = 1...25$. Все оценки известны
$x _{0} = 64, y _{0} = 84$, $x _{1} = 82, y _{1} = 42$, $x _{2} = 51, y _{2} = 99$, $x _{3} = 68, y _{3} = 57$, $x _{4} = 90, y _{4} = 71$, $x _{5} = 89, y _{5} = 55$, $x _{6} = 55, y _{6} = 55$, $x _{7} = 90, y _{7} = 58$, $x _{8} = 61, y _{8} = 78$, $x _{9} = 38, y _{9} = 84$, $x _{10} = 56, y _{10} = 95$, $x _{11} = 86, y _{11} = 69$, $x _{12} = 71, y _{12} = 72$, $x _{13} = 35, y _{13} = 99$, $x _{14} = 82, y _{14} = 67$, $x _{15} = 79, y _{15} = 59$, $x _{16} = 83, y _{16} = 88$, $x _{17} = 45, y _{17} = 75$, $x _{18} = 70, y _{18} = 79$, $x _{19} = 89, y _{19} = 80$, $x _{20} = 33, y _{20} = 30$, $x _{21} = 63, y _{21} = 73$, $x _{22} = 55, y _{22} = 53$, $x _{23} = 31, y _{23} = 78$, $x _{24} = 50, y _{24} = 90$
Требуется
найти следующие условные эмпирические характеристики: 1) ковариацию $X$ и $Y$ при условии, что одновременно $X \geqslant 50$
 и $Y \geqslant 50$; 2) коэффициент корреляции $X$ и $Y$ при том же условии.




1) Ковариация = $-876.6667$
2) Коэффициент корреляции = $-4.7659$


\item


(10) Эмпирическое распределение признаков $X$ и $Y$ на генеральной совокупности $\Omega$ задано таблицей частот  
 
\begin{tabular}{ | c | c | c | c | }
\hline
 & $Y = 2$ & $Y = 4$ & $Y = 5$  \\ \hline
$X = 200$ & $17$ & $3$ & $13$\\ \hline
$X = 300$ & $21$ & $23$ & $23$\\
\hline
\end{tabular}

Из $\Omega$ случайным образом без возвращения извлекаются $10$ элементов. 
Пусть $\bar X$ и $\bar Y$ – средние значения признаков на выбранных элементах. 
Требуется найти: 1) математическое ожидание $\mathbb{E}(\bar Y)$; 2) стандартное отклонение $\sigma(\bar X)$ ; 
3) ковариацию $Cov(\bar X, \bar Y)$




1) математическое ожидание $\mathbb{E}(\bar Y)$: $3.6$ 
2) стандартное отклонение $\sigma(\bar X)$: $257.2355$
3) ковариацию $Cov(\bar X, \bar Y)$: $0.7091$


\item

    
	Известно, что доля возвратов по кредитам в банке имеет распределение $F(x) = x^{\beta}, 0 \le x \le 1$. Наблюдения показали, что в среднем она составила $76.0$\%. Методом моментов оцените параметр $\beta$ и вероятность того, что она опуститься ниже $74.0$\%.
	


	

	$f(x) = F'(x) = \beta \cdot x^{\beta - 1}$

	$\mu_{1} = E(X) = \int_{-\inf}^{\inf}x \cdot f(x) = \int_{-\inf}^{\inf} \beta \cdot x^{\beta} = \beta \cdot \frac{x^{\beta + 1}}{\beta + 1}\bigg|_0^1 = \frac{\beta}{\beta + 1}$

	$\beta = (\beta + 1) \cdot 76.0$

	$\beta = \frac{76.0}{1 - 76.0}$

	$ P(x \le 74.0) = F(74.0) = 74.0^{3.17} $

    Ответ: $3.17, 0.39$
	


\end{enumerate}

\section{Билет 130}

\begin{enumerate}


\item

Дайте определение случайной величины, которая имеет гамма-распределение $\Gamma(\alpha,  \lambda)$, и выведите основные свойства гамма-расределения. Запишите формулы для математичсекого ожидания
$\mathbb{E}(X)$ и дисперсии $\mathbb{V}ar(X)$ гамма-распределения




Здесь написанно много всего интересного и полезного о гамма-распределении


\item



Случайные величины $X$ и $Y$ независимы и имеют равномерное
распределение на отрезках $[0;3]$ и $[0;8]$ соответственно. Для случайной величины $Z=\frac{Y}{X}$ найдите: 
1) функцию распределения $F_Z(x)$;
2) плотность распределения $f_Z(x)$ и постройте график плотности;
3) вероятность $\P(2,\!475\leqslant Z\leqslant 4,\!811)$.




%\folder 2_53d8.png
1) Функция распределения $F_Z(x)$ имеет вид:
$
F_Z(x)=\left\{
\begin{array}{l}
0, x\leqslant 0;\\
\frac{3 x}{16}, 0\leqslant x\leqslant \frac{8}{3}\approx 2,\!667;\\
1 - \frac{4}{3 x}, x\geqslant\frac{8}{3};
\end{array}.
\right.
$
2) Плотность распределения $f_Z(x)$ имеет вид:
$
f_Z(x)=\left\{
\begin{array}{l}
0, x<0;\\
\frac{3}{16}, 0\leqslant x\leqslant \frac{8}{3}\approx 2,\!667;\\
\frac{4}{3 x^{2}}, x\geqslant\frac{8}{3};
\end{array}.
\right.
$


\begin{figure}[H]
    \includegraphics[width=0.9\textwidth]{2_53d8}
\end{figure}


3) вероятность равна:
$
\P(2,\!475\leqslant Z\leqslant 4,\!811)=
0,\!25884.
$


\item

    
	Случайная величина Y принимает только значения из множества $\{7, 5\}$, при этом $P(Y=7) = 0.08$.
	Распределение случайной величины X определено следующим образом:
	\begin{equation*}
		X | Y =
		\begin{cases}
			$9$ * y, с вероятностью $ 0.24$ \\
			$8$ * y, с вероятностью $ 1 - 0.24$
		\end{cases}
	\end{equation*}

	Юный аналитик Дарья нашла матожидание и дисперсию $X$.

	Помогите Дарье найти матожидание и дисперсию величины $X$
	


	

	Первым этапом надо найти характеристики случайной величины $Y$

	$E(Y) = 7 * 0.08 + 5 * (1 - 0.08)$

	$Var(Y) = E(Y^2) - [E(Y)]^2 = 7^2 * 0.08 + 5^2 * (1 - 0.08) - [E(Y)]^2$


	Перейдем к рассмотрению характеристик условной случайно величины X

	$E(X) = E(E(X|Y)) = E[E(9 * Y) * 0.24 + E(8 * Y) * (1 - 0.24)] = E(Y) * (9 * 0.24 + 8 * (1 - 0.24)) = 42.5184$

	$E(Var(X|Y)) = E[b * Var(c3 * Y) + (1 - b) * Var(c4 * Y)] = Var(Y) * (c3^2 * b + c4^2 * (1- b)) $

	$Var(E(X|Y)) = E(X^2|Y) - [E(X)]^2 = [E(Y)]^2 * (b * c3^2 + (1-b)*c4^2) - E(X)]^2$

	$Var(X) = E(Var(X|Y)) + Var(E(X|Y)) = 24.89926$
	

\item


(10) В группе $\Omega$ учатся студенты:$\omega _{1}...\omega _{25}$ . Пусть $X$ и $Y$ – 100-балльные экзаменационные оценки по
математическому анализу и теории вероятностей. Оценки $\omega _{i}$ студента обозначаются: $x _{i} = X(\omega _{i})$ и $y _{i} = Y(\omega _{i})$, $i = 1...25$. Все оценки известны
$x _{0} = 55, y _{0} = 54$, $x _{1} = 64, y _{1} = 68$, $x _{2} = 34, y _{2} = 51$, $x _{3} = 48, y _{3} = 73$, $x _{4} = 81, y _{4} = 69$, $x _{5} = 62, y _{5} = 69$, $x _{6} = 76, y _{6} = 59$, $x _{7} = 84, y _{7} = 45$, $x _{8} = 97, y _{8} = 77$, $x _{9} = 76, y _{9} = 87$, $x _{10} = 43, y _{10} = 67$, $x _{11} = 33, y _{11} = 55$, $x _{12} = 71, y _{12} = 96$, $x _{13} = 62, y _{13} = 97$, $x _{14} = 84, y _{14} = 37$, $x _{15} = 41, y _{15} = 70$, $x _{16} = 92, y _{16} = 41$, $x _{17} = 60, y _{17} = 54$, $x _{18} = 71, y _{18} = 44$, $x _{19} = 39, y _{19} = 70$, $x _{20} = 98, y _{20} = 75$, $x _{21} = 99, y _{21} = 32$, $x _{22} = 58, y _{22} = 42$, $x _{23} = 61, y _{23} = 92$, $x _{24} = 58, y _{24} = 32$
Требуется
найти следующие условные эмпирические характеристики: 1) ковариацию $X$ и $Y$ при условии, что одновременно $X \geqslant 50$
 и $Y \geqslant 50$; 2) коэффициент корреляции $X$ и $Y$ при том же условии.




1) Ковариация = $276.75$
2) Коэффициент корреляции = $1.373$


\item

    
    	Распределение результатов экзамена в некоторой стране с $10$-балльной системой оценивания задано следующим образом:
    	$\left\{ 1 : 6, \  2 : 16, \  3 : 9, \  4 : 16, \  5 : 14, \  6 : 4, \  7 : 25, \  8 : 26, \  9 : 24, \  10 : 10\right\}$

	Работы будут перепроверять $10$ преподавателей, которые разделили все имеющиеся работы между собой случайным образом. Пусть $\overline{X}$ - средний балл (по перепроверки) работ, попавших к одному преподавателю.

	Требуется найти матожидание и стандартное отклонение среднего балла работ, попавших к одному преподавателю, до перепроверки.
    


    


    k = len(marks) // k

    ex = np.sum([marks[m] * m for m in marks]) / n

    varx = np.var([ m for m in marks for temp in range(marks[m])]) / k * (n - k) / (n - 1)

    sigmax = varx**(0.5)
    Ответы: $6.14667, 0.65542$.

    

\item


(10) Пусть $X _{1}$, $X _{2}$, $X _{3}$, $X _{4}$ выборка из $N(\theta, \sigma ^{2})$. Рассмотрим две оценки параметра $\theta$:
\[\hat \theta _{1} = \frac{3X _{1} + X _{2} + 4X _{3} + 2X _{4}}{10}, \hat \theta _{1} = \frac{X _{1} + 6X _{2} + 2X _{3} + X _{4}}{10}\]
a) Покажите, что обе оценки несмещенные.
б) Какая из оценок оптимальная?




Обе они несмещенные, потому что в числителе выходит в сумме 10.
Какая-то точно должна быть, а может и нет....



\end{enumerate}

\begin{figure}[H]
	Подготовил
	\hfill
	\includegraphics[width=2cm]{Prepared}
	П.Е. Рябов
\end{figure}


\begin{figure}[H]
	Утверждаю:\\
	Первый заместитель\\
	руководителя департамента\\
	Дата 01.06.2021
	\hfill
	\includegraphics[width=2cm]{Approved}
	Феклин В.Г.
\end{figure}

\end{document}