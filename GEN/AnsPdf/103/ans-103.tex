\documentclass[a4paper,14pt]{article}
\usepackage[14pt]{extsizes}




\usepackage{cmap}					% поиск в PDF
\usepackage{mathtext} 				% русские буквы в формулах
\usepackage[T2A]{fontenc}			% кодировка
\usepackage[utf8]{inputenc}			% кодировка исходного текста
\usepackage[english,russian]{babel}	% локализация и переносы
\usepackage{ulem}                   % зачеркнутый текст
\usepackage{amssymb}			% пакет математики
\usepackage{float}
\usepackage{amsmath}
\usepackage{graphicx}
\DeclareGraphicsExtensions{.png}

%%% Страница
%\usepackage{extsizes} % Возможность сделать 14-й шрифт
\usepackage[left=1cm,right=1cm,top=1cm,bottom=1cm]{geometry} % Простой способ задавать поля
\pagestyle{empty}

\begin{document}


\begin{center}
ФЕДЕРАЛЬНОЕ ГОСУДАРСТВЕННОЕ ОБРАЗОВАТЕЛЬНОЕ БЮДЖЕТНОЕ УЧРЕЖДЕНИЕ ВЫСШЕГО ОБРАЗОВАНИЯ

    \textbf{«ФИНАНСОВЫЙ УНИВЕРСИТЕТ ПРИ ПРАВИТЕЛЬСТВЕ РОССИЙСКОЙ ФЕДЕРАЦИИ»}

Факультет информационных технологий и анализа больших данных

Департамент анализа данных и машинного обучения

\textit{
	\textbf{Дисциплина: «Теория вероятностей и математическая статистика»}}

\textit{Направление подготовки: 01.03.02 «Прикладная математика и информатика»}

\textit{Профиль: «Анализ данных и принятие решений в экономике и финансах»}

\textit{Форма обучения очная, учебный 2020/2021 год, 4 семестр}

\textbf{Билет 103}

\end{center}

\begin{enumerate}


\item


Сформулируйте определение случайной выборки из конечной генеральной совокупности. Какие
виды выборок вам известны? Перечислите (с указанием формул) основные характеристики выборочной и генеральной совокупностей




Здесь очень много исчерпывающей информации о выборках из генеральной совокупности и про различные виды выборок


\item



Случайные величины $X$ и $Y$ независимы и имеют равномерное
распределение на отрезках $[0;1]$ и $[0;10]$ соответственно. Для случайной величины $Z=\frac{Y}{X}$ найдите: 
1) функцию распределения $F_Z(x)$;
2) плотность распределения $f_Z(x)$ и постройте график плотности;
3) вероятность $\P(2,\!96\leqslant Z\leqslant 17,\!91)$.




%\folder 2_53d13.png
1) Функция распределения $F_Z(x)$ имеет вид:
$
F_Z(x)=\left\{
\begin{array}{l}
0, x\leqslant 0;\\
\frac{x}{20}, 0\leqslant x\leqslant 10\approx 10,\!0;\\
1 - \frac{5}{x}, x\geqslant10;
\end{array}.
\right.
$
2) Плотность распределения $f_Z(x)$ имеет вид:
$
f_Z(x)=\left\{
\begin{array}{l}
0, x<0;\\
\frac{1}{20}, 0\leqslant x\leqslant 10\approx 10,\!0;\\
\frac{5}{x^{2}}, x\geqslant10;
\end{array}.
\right.
$


\begin{figure}[H]
    \includegraphics[width=0.9\textwidth]{2_53d13}
\end{figure}


3) вероятность равна:
$
\P(2,\!96\leqslant Z\leqslant 17,\!91)=
0,\!57283.
$


\item

%\folder 1.pdf
(10) Известно, что доля возвратов по кредитам в банке имеет распределение $F(x) = x ^{\beta}, 0 \leqslant x \leqslant 1$.
Наблюдения показали, что в среднем она составляет $91,6667\%$. Методом моментов оцените параметр $\beta$ и
вероятность того, что она опуститься ниже $59\%$




Найдём плотность рапределения как интеграл от ФР, а дальше всё и вовсе простою Ответ: $30155888444737842659$


\item


(10) В группе $\Omega$ учатся студенты:$\omega _{1}...\omega _{25}$ . Пусть $X$ и $Y$ – 100-балльные экзаменационные оценки по
математическому анализу и теории вероятностей. Оценки $\omega _{i}$ студента обозначаются: $x _{i} = X(\omega _{i})$ и $y _{i} = Y(\omega _{i})$, $i = 1...25$. Все оценки известны
$x _{0} = 73, y _{0} = 44$, $x _{1} = 44, y _{1} = 83$, $x _{2} = 49, y _{2} = 41$, $x _{3} = 36, y _{3} = 32$, $x _{4} = 48, y _{4} = 60$, $x _{5} = 53, y _{5} = 37$, $x _{6} = 70, y _{6} = 86$, $x _{7} = 61, y _{7} = 82$, $x _{8} = 42, y _{8} = 57$, $x _{9} = 94, y _{9} = 40$, $x _{10} = 44, y _{10} = 78$, $x _{11} = 85, y _{11} = 78$, $x _{12} = 48, y _{12} = 66$, $x _{13} = 88, y _{13} = 82$, $x _{14} = 31, y _{14} = 39$, $x _{15} = 84, y _{15} = 68$, $x _{16} = 49, y _{16} = 51$, $x _{17} = 84, y _{17} = 55$, $x _{18} = 65, y _{18} = 67$, $x _{19} = 37, y _{19} = 99$, $x _{20} = 46, y _{20} = 31$, $x _{21} = 84, y _{21} = 46$, $x _{22} = 40, y _{22} = 67$, $x _{23} = 86, y _{23} = 54$, $x _{24} = 89, y _{24} = 32$
Требуется
найти следующие условные эмпирические характеристики: 1) ковариацию $X$ и $Y$ при условии, что одновременно $X \geqslant 50$
 и $Y \geqslant 50$; 2) коэффициент корреляции $X$ и $Y$ при том же условии.




1) Ковариация = $-345.5$
2) Коэффициент корреляции = $-2.9554$


\item

    
    	Распределение результатов экзамена в некоторой стране с $10$-балльной системой оценивания задано следующим образом:
    	$\left\{ 1 : 6, \  2 : 16, \  3 : 9, \  4 : 16, \  5 : 14, \  6 : 4, \  7 : 25, \  8 : 26, \  9 : 24, \  10 : 10\right\}$

	Работы будут перепроверять $10$ преподавателей, которые разделили все имеющиеся работы между собой случайным образом. Пусть $\overline{X}$ - средний балл (по перепроверки) работ, попавших к одному преподавателю.

	Требуется найти матожидание и стандартное отклонение среднего балла работ, попавших к одному преподавателю, до перепроверки.
    


    


    k = len(marks) // k

    ex = np.sum([marks[m] * m for m in marks]) / n

    varx = np.var([ m for m in marks for temp in range(marks[m])]) / k * (n - k) / (n - 1)

    sigmax = varx**(0.5)
    Ответы: $6.14667, 0.65542$.

    

\item

    
	Известно, что доля возвратов по кредитам в банке имеет распределение $F(x) = x^{\beta}, 0 \le x \le 1$. Наблюдения показали, что в среднем она составила $60.0$\%. Методом моментов оцените параметр $\beta$ и вероятность того, что она опуститься ниже $52.0$\%.
	


	

	$f(x) = F'(x) = \beta \cdot x^{\beta - 1}$

	$\mu_{1} = E(X) = \int_{-\inf}^{\inf}x \cdot f(x) = \int_{-\inf}^{\inf} \beta \cdot x^{\beta} = \beta \cdot \frac{x^{\beta + 1}}{\beta + 1}\bigg|_0^1 = \frac{\beta}{\beta + 1}$

	$\beta = (\beta + 1) \cdot 60.0$

	$\beta = \frac{60.0}{1 - 60.0}$

	$ P(x \le 52.0) = F(52.0) = 52.0^{1.5} $

    Ответ: $1.5, 0.37$
	

\end{enumerate}

\begin{figure}[H]
	Подготовил
	\hfill
	\includegraphics[width=2cm]{Prepared}
	П.Е. Рябов
\end{figure}


\begin{figure}[H]
	Утверждаю:\\
	Первый заместитель\\
	руководителя департамента\\
	Дата 01.06.2021
	\hfill
	\includegraphics[width=2cm]{Approved}
	Феклин В.Г.
\end{figure}

\end{document}

