\documentclass[a4paper,14pt]{article}
\usepackage[14pt]{extsizes}




\usepackage{cmap}					% поиск в PDF
\usepackage{mathtext} 				% русские буквы в формулах
\usepackage[T2A]{fontenc}			% кодировка
\usepackage[utf8]{inputenc}			% кодировка исходного текста
\usepackage[english,russian]{babel}	% локализация и переносы
\usepackage{ulem}                   % зачеркнутый текст
\usepackage{amssymb}			% пакет математики
\usepackage{float}
\usepackage{amsmath}
\usepackage{graphicx}
\DeclareGraphicsExtensions{.png}

%%% Страница
%\usepackage{extsizes} % Возможность сделать 14-й шрифт
\usepackage[left=1cm,right=1cm,top=1cm,bottom=1cm]{geometry} % Простой способ задавать поля
\pagestyle{empty}

\begin{document}


\begin{center}
ФЕДЕРАЛЬНОЕ ГОСУДАРСТВЕННОЕ ОБРАЗОВАТЕЛЬНОЕ БЮДЖЕТНОЕ УЧРЕЖДЕНИЕ ВЫСШЕГО ОБРАЗОВАНИЯ

    \textbf{«ФИНАНСОВЫЙ УНИВЕРСИТЕТ ПРИ ПРАВИТЕЛЬСТВЕ РОССИЙСКОЙ ФЕДЕРАЦИИ»}

Факультет информационных технологий и анализа больших данных

Департамент анализа данных и машинного обучения

\textit{
	\textbf{Дисциплина: «Теория вероятностей и математическая статистика»}}

\textit{Направление подготовки: 01.03.02 «Прикладная математика и информатика»}

\textit{Профиль: «Анализ данных и принятие решений в экономике и финансах»}

\textit{Форма обучения очная, учебный 2020/2021 год, 4 семестр}

\textbf{Билет 105}

\end{center}

\begin{enumerate}


\item


Сформулируйте определение случайной выборки из конечной генеральной совокупности. Какие
виды выборок вам известны? Перечислите (с указанием формул) основные характеристики выборочной и генеральной совокупностей




Здесь очень много исчерпывающей информации о выборках из генеральной совокупности и про различные виды выборок


\item


(10) Сформулируйте критерий независимости $\chi ^ {2}$ – Пирсона. Приведите (с выводом и
необходимыми пояснениями в обозначениях) явный вид статистики критерия в случае, когда 
таблица сопряженности двух признаков $X$ и $Y$ имеет вид

\begin{tabular}[b]{ | c | c | c | }
\hline
$ $ & $Y = y _{1}$ & $Y = y _{2}$  \\ \hline
$X = x _{1}$ & $a$ & $b$ \\ \hline
$X = x _{2}$ & $c$ & $d$ \\
\hline
\end{tabular}




Здесь формулировки критерия независимости Пирсона и приводится пример


\item

    
	Случайная величина Y принимает только значения из множества $\{10, 7\}$, при этом $P(Y=10) = 0.24$.
	Распределение случайной величины X определено следующим образом:
	\begin{equation*}
		X | Y =
		\begin{cases}
			$4$ * y, с вероятностью $ 0.53$ \\
			$9$ * y, с вероятностью $ 1 - 0.53$
		\end{cases}
	\end{equation*}

	Юный аналитик Дарья нашла матожидание и дисперсию $X$.

	Помогите Дарье найти матожидание и дисперсию величины $X$
	


	

	Первым этапом надо найти характеристики случайной величины $Y$

	$E(Y) = 10 * 0.24 + 7 * (1 - 0.24)$

	$Var(Y) = E(Y^2) - [E(Y)]^2 = 10^2 * 0.24 + 7^2 * (1 - 0.24) - [E(Y)]^2$


	Перейдем к рассмотрению характеристик условной случайно величины X

	$E(X) = E(E(X|Y)) = E[E(4 * Y) * 0.53 + E(9 * Y) * (1 - 0.53)] = E(Y) * (4 * 0.53 + 9 * (1 - 0.53)) = 49.022$

	$E(Var(X|Y)) = E[b * Var(c3 * Y) + (1 - b) * Var(c4 * Y)] = Var(Y) * (c3^2 * b + c4^2 * (1- b)) $

	$Var(E(X|Y)) = E(X^2|Y) - [E(X)]^2 = [E(Y)]^2 * (b * c3^2 + (1-b)*c4^2) - E(X)]^2$

	$Var(X) = E(Var(X|Y)) + Var(E(X|Y)) = 447.56552$
	

\item


(10) В группе $\Omega$ учатся студенты:$\omega _{1}...\omega _{25}$ . Пусть $X$ и $Y$ – 100-балльные экзаменационные оценки по
математическому анализу и теории вероятностей. Оценки $\omega _{i}$ студента обозначаются: $x _{i} = X(\omega _{i})$ и $y _{i} = Y(\omega _{i})$, $i = 1...25$. Все оценки известны
$x _{0} = 55, y _{0} = 54$, $x _{1} = 64, y _{1} = 68$, $x _{2} = 34, y _{2} = 51$, $x _{3} = 48, y _{3} = 73$, $x _{4} = 81, y _{4} = 69$, $x _{5} = 62, y _{5} = 69$, $x _{6} = 76, y _{6} = 59$, $x _{7} = 84, y _{7} = 45$, $x _{8} = 97, y _{8} = 77$, $x _{9} = 76, y _{9} = 87$, $x _{10} = 43, y _{10} = 67$, $x _{11} = 33, y _{11} = 55$, $x _{12} = 71, y _{12} = 96$, $x _{13} = 62, y _{13} = 97$, $x _{14} = 84, y _{14} = 37$, $x _{15} = 41, y _{15} = 70$, $x _{16} = 92, y _{16} = 41$, $x _{17} = 60, y _{17} = 54$, $x _{18} = 71, y _{18} = 44$, $x _{19} = 39, y _{19} = 70$, $x _{20} = 98, y _{20} = 75$, $x _{21} = 99, y _{21} = 32$, $x _{22} = 58, y _{22} = 42$, $x _{23} = 61, y _{23} = 92$, $x _{24} = 58, y _{24} = 32$
Требуется
найти следующие условные эмпирические характеристики: 1) ковариацию $X$ и $Y$ при условии, что одновременно $X \geqslant 50$
 и $Y \geqslant 50$; 2) коэффициент корреляции $X$ и $Y$ при том же условии.




1) Ковариация = $276.75$
2) Коэффициент корреляции = $1.373$


\item

    
    	Распределение результатов экзамена в некоторой стране с $11$-балльной системой оценивания задано следующим образом:
    	$\left\{ 1 : 13, \  2 : 3, \  3 : 14, \  4 : 9, \  5 : 6, \  6 : 15, \  7 : 1, \  8 : 22, \  9 : 17, \  10 : 10, \  11 : 16\right\}$

	Работы будут перепроверять $6$ преподавателей, которые разделили все имеющиеся работы между собой случайным образом. Пусть $\overline{X}$ - средний балл (по перепроверки) работ, попавших к одному преподавателю.

	Требуется найти матожидание и стандартное отклонение среднего балла работ, попавших к одному преподавателю, до перепроверки.
    


    


    k = len(marks) // k

    ex = np.sum([marks[m] * m for m in marks]) / n

    varx = np.var([ m for m in marks for temp in range(marks[m])]) / k * (n - k) / (n - 1)

    sigmax = varx**(0.5)
    Ответы: $6.57937, 0.64259$.

    

\item

    
    	Юный аналитик Дарья использовала метод Монте-Карло для исследования Дискретного случайного вектора, описанного ниже.

        \begin{tabular}{|c|c|c|c|}
	\hline
	& X=$-3$ & X=$-2$ & X=$-1$ \\
	\hline
	Y = $2$ & $0.29$ & $0.298$  &  $0.234$ \\
	\hline
	Y = $3$ & $0.066$ & $0.03$ & $0.082$  \\
	\hline
\end{tabular}

    	Дарья получила, что E(Y|X + Y = 1) = $2.10982$.
    	Проверьте, можно ли доверять результату Дарьи аналитически. Сформулируйте определение метода Монте-Карло.
    


    
        $E(Y|X+Y=1) = \frac{\sum(P(X=1 - y_i, y=y_i) * y_i)}{\sum(P(X=1 - y_i, y=y_i)}$.

        Ответ: $2.10982$
    

\end{enumerate}

\begin{figure}[H]
	Подготовил
	\hfill
	\includegraphics[width=2cm]{Prepared}
	П.Е. Рябов
\end{figure}


\begin{figure}[H]
	Утверждаю:\\
	Первый заместитель\\
	руководителя департамента\\
	Дата 01.06.2021
	\hfill
	\includegraphics[width=2cm]{Approved}
	Феклин В.Г.
\end{figure}

\end{document}

